\dhatu{aṅk} \pres{aṅkayati; -te \gan{X PĀ}}, \pppp{aṅkita} \artha{kennzeichnen} 
\forms{
}

\dhatu{añc} \pres{añcati \gan{I PĀ}}, \pppp{añcita} \artha{biegen, verehren}
\forms{
\CS añcayati 
}
\Zusatz{añcita \artha{gebogen, verehrt}, (sam-)ud-akta \artha{[Wasser aus einem Brunnen] hochgehoben, geleert}}

\dhatu{añj} \pres{anakti \gan{VII P}}, \pppp{akta} \artha{salben, schmücken, ehren} 
\forms{
\OP añjyāt
\IPV anaktu
\PS ajyate
\IMP ānak
\GDV añjya, \mbox{-añjanīya}
\CS añjayati \CS \PPP añjita
}
\Zusatz{vy-akta \artha{deutlich}, vy-añjita \artha{deutlich gemacht}}

\dhatu{aṭ} \pres{aṭati; -te \gan{I P}}, \pppp{aṭita} \artha{herumstreifen} 
\forms{
\INFIN aṭitum
\ABS aṭitvā
}

\dhatu{ad} \pres{atti \gan{II P}}, \pppp{[jagdha]} \artha{essen} 
\forms{
\IND admi, atsi, atti, adanti
\OP adyāt
\IPV adāni, addhi, attu, adantu
\IMP ādam, ādaḥ, ādat, ādan
\AO [aghasat]
\FT atsyati
\PS adyate
\GDV \shortlonga{}dya, attavya, adanīya
\INFIN attum[/jagdhum]
\ABS [jagdhvā, \mbox{-jagdhya}]
\CS ādayati
\DS [jighatsati]
}
\Zusatz{unvollständiges Paradigma; fehlende Formen werden durch die der gleichbedeutenden Wurzel \vw{ghas/jakṣ} ergänzt}

\dhatu{adhī} \vw{adhi-i}

\dhatu{an} \pres{aniti \gan{II  P}}, \pppp{---} \artha{atmen}
\forms{
\OP anyāt
\IPV anāni, anihi
\IMP ānam, ānīḥ/ānaḥ, ānīt/ānat
\CS ānayati
}
\Zusatz{prāṇa m.\ \artha{Atem, Leben}}

\dhatu{anviṣ/anveṣ} \vw{iṣ [1]}

\dhatu{abhilaṣ} \vw{laṣ}

\dhatu{ay} \vw{i}

\dhatu{argh} \pres{arghati \gan{I P}}, \pppp{arghita} \artha{wert sein}
\forms{
\GDV arghya
}
\Zusatz{arghya n.\ \artha{Fußwasser}}

\dhatu{arc} \pres{arcati \gan{I P}}, \pppp{arcita} \artha{verehren} 
\forms{
\PF ānarca, ānarcuḥ
\AO ārcicat
\FT arciṣyati
\PS arcyate
\GDV arcya, arcanīya
\INFIN arcitum
\ABS arcitvā, \mbox{-arcya}
\CS arcayati 
}

\dhatu{arj} \vw{ṛj}

\dhatu{arpay} \vw{ṛ}

\dhatu{arth} \pres{arthayate \gan{X Ā}}, \pppp{arthita} \artha{verlangen, bitten}
\forms{
\PS arthyate
\GDV arthitavya, arthanīya
\INFIN  arthayitum
\ABS -arthya
}


\dhatu{arh} \pres{arhati; -te \gan{I P}}, \pppp{arhita} \artha{dürfen, müssen, können} 
\forms{
\GDV \mbox{-arhitavya}, arhaṇīya
\CS arhayati \artha{verehren} 
}
\Zusatz{kartum arhati \artha{er\altern sie\altern es muss\altern kann tun}, na tat śaṃsitum arhati \artha{das verdient nicht gepriesen zu werden}}

\dhatu{av} \pres{avati \gan{I P}}, \pppp{avita} \artha{fördern, schützen}
\forms{
\IPV avatu\altern avatāt
\PF āva
\INFIN avitum
}

\dhatu{aś [1]} \pres{aśnoti, aśnute \gan{V Ā}}, \pppp{aṣṭa} \artha{erreichen} 
\forms{
\OP aśnuvīta
\IPV aśnavai, aśnuṣva, aśnutām
\IMP āśnuvi, āśnuthāḥ, āśnuta
\PF ānaṃśa, ānaśuḥ; ānaśe
\INFIN aśitum
\ABS aṣṭvā/aśitvā
}

\dhatu{aś [2]}  \pres{aśnāti \gan{IX P}}, \pppp{aśita} \artha{essen} 
\forms{
\OP aśnīyāt
\IPV aśnāni, aśāna, aśnātu
\PF āśa
\AO āśīt
\FT aśiṣyati
\PS aśyate
\GDV āśya, aśitavya, aśanīya
\INFIN aśitum
\ABS aśitvā, \mbox{-aśya}
\CS āśayati \CS \PPP āśita
\DS aśiśiṣati
}

\dhatu{as [1]} \pres{asti \gan{II P}}, \pppp{---} \artha{sein} 
\forms{
\IND asmi, asi, asti, svaḥ, sthaḥ, staḥ, smaḥ, stha, santi
\OP syām, syāḥ, syāt, syāva, syātam, syātām, syāma,
syāta, syuḥ
\IPV asāni, edhi, astu, asāva, stam, stām, asāma, sta, santu
\IMP āsam, āsīḥ, āsīt, āsva, āstam, āstām, āsma, āsta, āsan
\PF āsa, āsitha, āsa, āsiva, āsathuḥ, āsatuḥ, āsima, āsa, āsuḥ
}

\dhatu{as [2]} \pres{asyati \gan{IV P}}, \pppp{asta} \artha{werfen} 
\forms{
\PF āsa, āsitha, āsa, āsiva, āsathuḥ, āsatuḥ, āsima, āsa, āsuḥ
\AO āsthat
\FT asiṣyati
\PS asyate \PS \AO āsi
\GDV astavya, asanīya
\INFIN asitum
\ABS as(i)tvā
\CS āsayati
}

\dhatu{ah} \pres{---}, \pppp{---} \artha{sagen}
\forms{
\PF āttha (2.\,Sg.), āha (3.\,Sg.), āhathuḥ (2.\,Dual), āhatuḥ (3.\,Dual), āhuḥ (3.\,Pl.)
}
\Zusatz{unvollständiges Paradigma; die Formen werden präsentisch und perfektisch gebraucht}

\medskip

\dhatu{ākarṇay} \pres{ākarṇayati} \gand{Denom.} \pppp{ākarṇita} \artha{hören}
\forms{
\INFIN ākarṇayitum
\ABS ākarṇya
}
\Zusatz{karṇa m.\ \artha{Ohr}}

\dhatu{ākulay} \pres{ākulayati} \gand{Denom.} \pppp{ākulita} \artha{verwirren}
\forms{
}
\Zusatz{ākula \artha{verwirrt}}

\dhatu{ādṛ/ādriy} \vw{dṛ [1]}

\dhatu{āp} \pres{āpnoti; āpnute \gan{V P}}, \pppp{āpta} \artha{erreichen} 
\forms{
\IND āpnomi, āpnoṣi, āpnoti, āpnuvaḥ, āpnuthaḥ, āpnutaḥ, āpnumaḥ, āpnutha, āpnuvanti; āpnuve, āpnuṣe, āpnute, āpnuvahe, āpnuvāthe, āpnuvāte, āpnumahe, āpnudhve, āpnuvate
\OP āpnuyāt
\IPV āpnavāni,  āpnuhi, āpnotu
\IMP āpnot
\PF āpa, āpuḥ
\AO āpat
\FT āpsyati
\PS āpyate \PS \AO āpi
\GDV āpya, āptavya, āpanīya
\INFIN āptum
\ABS āptvā, \mbox{(-)āpya}
\CS āpayati  \CS \GDV āpayitavya
\DS īpsati
}
\Zusatz{häufig mit pra-: su-prāpya/-prāpa \artha{leicht zu erreichen}, duṣ-prāpya/-prāpa \artha{schwer zu erreichen}}

\dhatu{āpar/āpṛ/āpri} \vw{pṛ}

\dhatu{āpyā} \vw{pyā}

\dhatu{āmnā} \vw{mnā}

\dhatu{āyas} \vw{yas}

\dhatu{ārabh} \artha{anfangen} \vw{rabh}

\dhatu{āliṅg} \pres{āliṅgati; -te} \gand{Denom.} \pppp{āliṅgita} \artha{umarmen}
\forms{
\GDV āliṅganīya
\INFIN āliṅgitum
\ABS āliṅgitvā/āliṅgya
\CS āliṅgayati
}


\dhatu{ās} \pres{āste \gan{II  Ā}}, \pppp{āsita} \artha{sitzen} 
\forms{
\OP āsīta
\IPV āstām \Zusatz{auch \artha{weg damit!}, \artha{genug davon!}}
\IMP āsta
\PPF āsāṃ cakre; cakāra
\AO āsiṣṭa
\FT āsiṣyate
\PS āsyate
\PPA āsīna
\GDV \mbox{-āsya}, āsitavya, āsanīya
\INFIN āsitum
\ABS āsitvā, \mbox{-āsya}
}
\Zusatz{āsana n.\ \artha{Sitzen, Sitz}}

\medskip

\dhatu{i}  \pres{eti \gan{II  P} | ayati; -te \gan{I P}}, \pppp{ita} \artha{gehen} 
\forms{
\IND emi, eṣi, eti, ivaḥ, ithaḥ, itaḥ, imaḥ, itha, yanti | ayati; -te
\OP iyāt
\IPV ayāni,  ihi,  etu,  ayāva, itam, itām, ayāma, ita, yantu
\IMP āyam, aiḥ, ait, aiva, aitam, aitām, aima, aita, āyan
\PF iyāya,  iyetha,  iyāya,  īyiva, īyuḥ
\PPF \mbox{-ayāṃ} cakre
\AO [agāt]
\FT eṣyati
\PFT etā
\PS īyate
\GDV itya, etavya
\INFIN etum
\ABS itvā, \mbox{-itya}\altern \mbox{-īya}
\CS āyayati
}
\Zusatz{unvollständiges Paradigma; Aorist wird von der gleichbedeutenden Wurzel \vw{gā [1]} ergänzt}

\dhatu{adhi-i}  \pres{adhīte \gan{II  Ā}}, \pppp{adhīta} \artha{lernen, lesen} 
\forms{
\OP adhīyīta
\IPV adhyayai,  adhīṣva,  adhītām,  adhyayāvahai,  adhīyāthām,  adhīyātām,  adhyayāmahai, adhīdhvam, adhīyatām
\IMP adhyaita, adhyaiyātām, adhyaiyata
\AO adhyaiṣṭa,  adhyaiṣātām,  adhyaiṣata
\FT adhyeṣyate
\PS adhīyate
\CS adhyāpayati
}

\dhatu{iṅg} \pres{iṅgati; -te \gan{I P}}, \pppp{iṅgita} \artha{sich bewegen} 
\forms{
\GDV iṅgya
\CS iṅgayati
}

\dhatu{idh/indh} \pres{inddhe \gan{VII  Ā}}, \pppp{iddha} \artha{(ent)flammen} 
\forms{
\IND inddhe, indhate
\OP indhīta
\IPV inadhai,  intsva,  inddhām
\IMP ainddha
\FT indhiṣyate
\PS idhyate
}

\dhatu{iṣ [1]} \pres{iṣyati \gan{IV P}}, \pppp{iṣita} \artha{antreiben, senden} 
\forms{
\AO \mbox{-aiṣīt}
\PS iṣyate
\GDV \mbox{-eṣyam}, \mbox{-eṣitavya}
\ABS \mbox{-iṣya}/pra-iṣya = preṣya
\CS iṣayati; -te \CS \PPF \mbox{-eṣayām āsa}
}
\Zusatz{meist mit pra- oder anu-; iṣu m.\ \artha{Pfeil}; preṣaṇa n.\ \artha{Absenden; Befehl}; anveṣa(ṇa) n.\ \artha{Suchen}}

\dhatu{iṣ [2]} \pres{icchati; -te \gan{VI  P}}, \pppp{iṣṭa} \artha{wünschen} 
\forms{
\IMP aicchat
\PF iyeṣa,  iyeṣitha, iyeṣa,  īṣiva, īṣuḥ; īṣe, īṣire
\AO aiṣīt
\FT eṣiṣyati
\PS iṣyate
\GDV eṣya, eṣitavya/eṣṭavya, eṣaṇīya
\INFIN eṣṭum
\ABS iṣṭvā/eṣitvā, \mbox{-iṣya}
\CS eṣayati \CS \PPP eṣita
}

\medskip

\dhatu{īkṣ} \pres{īkṣate; -ti \gan{I  Ā}}, \pppp{īkṣita} \artha{sehen; erwarten} 
\forms{
\IMP aikṣata
\PPF īkṣāṃ cakre; cakruḥ
\AO aikṣiṣṭa
\FT īkṣiṣyate
\PS īkṣyate  \PS \AO aikṣi
\GDV \mbox{-īkṣya}, īkṣitavya, īkṣaṇīya
\INFIN īkṣitum
\ABS \mbox{(-)īkṣya}
\CS īkṣayati
}

\dhatu{īḍ} \pres{īṭṭe \gan{II Ā}}, \pppp{īḍita} \artha{preisen} 
\forms{
\PF īḍire
\PS īḍyate
\GDV īḍya
\INFIN īḍitum
}

\dhatu{īps} \vw{āp}

\dhatu{īr} \pres{īrte \gan{II Ā}}, \pppp{īrṇa} \artha{bewegen} 
\forms{
\GDV \mbox{-īraṇīya}
\CS īrayati  \CS \PPP īrita
}

\dhatu{īrṣy} \pres{īrṣyati \gan{I P}}, \pppp{īrṣ(y)ita} \artha{neidisch sein} 
\forms{
\GDV īrṣ(y)itavya
}
\Zusatz{īrṣyā f.\ \artha{Neid}}

\dhatu{īś} \pres{īṣṭe \gan{II Ā}}, \pppp{īśita} \artha{besitzen, herrschen} 
\forms{
\GDV īśitavya
\INFIN īśitum
}

\dhatu{īh} \pres{īhate \gan{I Ā}}, \pppp{īhita} \artha{sich bemühen um} 
\forms{
\INFIN īhitum
}

\medskip

\dhatu{ukṣ} \pres{ukṣati; -te \gan{I P}}, \pppp{ukṣita} \artha{besprengen} 
\forms{
\ABS \mbox{-ukṣya}
}

\dhatu{uc} \pres{ucyati \gan{IV P}}, \pppp{ucita} \artha{angemessen, passend sein} 
\forms{
}

\dhatu{ujjh} \pres{ujjhati \gan{VI P}}, \pppp{ujjhita} \artha{verlassen, entsenden} 
\forms{
\IMP aujjhat
\PPF ujjhāṃ cakāra
\AO aujjhīt
\INFIN ujjhitum
\ABS ujjhitvā
}

\dhatu{uttaṃsay} \pres{uttaṃsayati} \gand{Denom.} \pppp{uttaṃsita} \artha{bekränzen}
\forms{
}
\Zusatz{uttaṃsa m.\ \artha{Kranz}}

\dhatu{uñch} \pres{uñchati \gan{I P}}, \pppp{---} \artha{(Ähren) nachlesen} 
\forms{
\INFIN uñchitum
\ABS \mbox{-uñchya}
}

\dhatu{ud/und} \pres{unatti \gan{VII P} | undati}, \pppp{utta/unna} \artha{quellen, benetzen} 
\forms{
}

\dhatu{unmūlay} \pres{unmūlayati} \gand{Denom.} \pppp{unmūlita} \artha{entwurzeln, entthronen}
\forms{
}
\Zusatz{mūla n.\ \artha{Wurzel; Haupttext (im Gegensatz zum Kommentar)}}

\dhatu{ubh/umbh} \pres{umbhati \gan{VI P}}, \pppp{umbhita} \artha{zusammenschnüren, verschließen} 
\forms{
}

\dhatu{uṣ} \pres{oṣati \gan{I  P}}, \pppp{uṣṭa} \artha{verbrennen; züchtigen} 
\forms{
\IMP auṣat
\AO auṣīt
\PS uṣyate
}

\medskip

\dhatu{ūrjay} \pres{ūrjayati} \gand{Denom.} \pppp{ūrjita} \artha{kräftigen}
\forms{
\PPA ūrjayant
}

\dhatu{ūh [1]} \pres{ūhati; -te}, \pppp{ūḍha/ūhita} \artha{schieben, verändern}
\forms{
\PS ūhyate \PS \AO ohi
\GDV ūhya, ūhitavya, ūhanīya
\INFIN oḍhum/ūhitum
\ABS \mbox{-ūhya}
\CS ūhayati
}

\dhatu{ūh [2]} \pres{ūhate; -ti \gan{I Ā}}, \pppp{---} \artha{bemerken, erschließen}
\GDV ūhitavya
\INFIN  ūhitum
\ABS \mbox{-ūhya}
\medskip

\dhatu{ṛ} \pres{ṛcchati; -te \gan{I P}}, \pppp{ṛta} \artha{erreichen} 
\forms{
\IMP ārchat
\PF āra,  āritha,  āra,  āriva
\ABS ṛtvā
\CS arpayati; -te \CS \PPP arpita \CS \PPF arpayām āsa \CS \GDV \mbox{-arpya}, arpayitavya, arpaṇīya
}

\dhatu{ṛc} \vw{arc}

\dhatu{ṛj} \pres{arjati \gan{I P}}, \pppp{arjita} \artha{erwerben}
\forms{
\GDV \mbox{-arjya}, arjanīya
\INFIN arjitum
\CS arjayati; -te
\CS \ABS arjayitvā
}

\dhatu{ṛdh} \pres{ṛdhyati \gan{IV}}, \pppp{ṛddha} \artha{gedeihen} 
\forms{
\PS ṛdhyate
\CS ardhayati
}

\medskip

\dhatu{edh} \pres{edhate; -ti \gan{I Ā}}, \pppp{edhita} \artha{wachsen} 
\forms{
\OP edheta
\IPV edhatām
\IMP aidhata
\PF edhire
\PPF edhām āsa/babhūva
\INFIN edhitum
\CS edhayati
\DS edidhiṣate
}

\medskip

\dhatu{kaṇḍūy} \pres{kaṇḍūyati; -te \gan{I PĀ}}, \pppp{kaṇḍūyita} \artha{kratzen; jucken} 
\forms{
\GDV kaṇḍūyitavya
\INFIN kaṇḍūyitum
}

\dhatu{katth} \pres{katthate; -ti \gan{I Ā}}, \pppp{katthita} \artha{prahlen} 
\forms{
\GDV katthitavya
\INFIN katthitum
\CS katthayati
}

\dhatu{kath} \pres{kathayati \gan{X P}}, \pppp{kathita} \artha{erzählen} 
\forms{
\PPF kathayām āsa/cakāra
\AO acakathat/acīkathat 
\GDV kathya, kathitavya, kathanīya
\INFIN kathayitum
\ABS kathayitvā
}

\dhatu{kadarthay} \pres{kadarthayati} \gand{Denom.} \pppp{kadarthita} \artha{missachten, quälen}
\forms{
\GDV kadarthanīya
\INFIN kad\-arthayitum
}
\Zusatz{v.\ kadartha m.\ \artha{nichtsnutzige Sache}}

\dhatu{kandalay} \pres{kandalayati} \gand{Denom.} \pppp{kandalita} \artha{in Menge hervorbringen}
\forms{
}
\Zusatz{v.\ kandala m.\ \artha{Blüte der Kandalī-Pflanze}}

\dhatu{kam} \pres{--- \gan{I Ā}}, \pppp{kānta} \artha{lieben} 
\forms{
\PF cakame
\CS kāmayate
\CS \PPF kāmayāṃ cakre
\CS \FT kāmayiṣyate
\CS \GDV kāmya, kāmayitavya, k\shortlonga{}manīya
}
\Zusatz{CS ersetzt präs.\ Formen}

\dhatu{kamp} \pres{kampate; -ti \gan{I Ā}}, \pppp{kampita} \artha{zittern, beben} 
\forms{
\PF cakampe
\PS kampyate
\GDV kampya, kampanīya
\INFIN kampitum
\ABS kampitvā, \mbox{-kampya}
\CS kampayati; -te
}

\dhatu{kar} \vw{kṛ}

\dhatu{kal [1]}  \pres{kalayati; -te \gan{X P}}, \pppp{kalita} \artha{tun, antreiben, bemerken, erkennen, zählen} 
\forms{
\PPF kalayām āsa
\AO acakalat
\PS kalyate
\GDV \mbox{-kalya}, \mbox{-kalanīya}
\INFIN kalayitum
}
\Zusatz{°kalita auch \artha{versehen mit \dots}, kālena a-kalita \artha{nicht durch Zeit bestimmt}.}

\dhatu{kal [2]} \pres{kālayati \gan{X P}}, \pppp{kālita} \artha{antreiben} 
\forms{
\PS kālyate
\INFIN  kālayitum
}

\dhatu{kavalay} \pres{kavalayati} \gand{Denom.} \pppp{kavalita} \artha{verschlucken}
\forms{
}
\Zusatz{v.\ kavala m.\ \artha{Bissen}}

\dhatu{kaṣ} \pres{kaṣati; -te \gan{I P}}, \pppp{kaṣita} \artha{schaben, kratzen} 
\forms{
}

\dhatu{kas} \pres{kasati \gan{I P}}, \pppp{kasita} \artha{sich öffnen} 
\forms{
\PF cakase
\CS kāsayati \CS \AO acīkasat \CS \PS kāsyate \CS \PPP kāsita \CS \GDV \mbox{-kāsanīya}
}
\Zusatz{vi-kasita \artha{strahlend, erblüht}, niṣ-kāsita \artha{vertrieben}}


\dhatu{kāṅkṣ} \pres{kāṅkṣati; -te \gan{I P}}, \pppp{kāṅkṣita} \artha{begehren} 
\forms{
\PF cakāṅkṣa
\FT kāṅkṣiṣyati
\GDV \mbox{-kāṅkṣya}, kāṅkṣitavya, kāṅkṣaṇīya
\CS kāṅkṣayate
}

\dhatu{kāś} \pres{kāśate; -ti \gan{I Ā} | -kāśyate \gan{IV Ā}}, \pppp{kāśita} \artha{erscheinen, glänzen} 
\forms{
\PF cakāśe
\GDV \mbox{-kāśya}, \mbox{-kāśayitavya}, \mbox{-kāśanīya}
\ABS \mbox{-kāśya}
\CS kāśayati
}
\Zusatz{oft: pra-kāś}

\dhatu{kās} \pres{kāsate; -ti \gan{I Ā}}, \pppp{---} \artha{husten} 
\Zusatz{ut-kāsana n.\ \artha{Aushusten}}

\dhatu{kit} \vw{cit}

\dhatu{kir} \vw{kṝ}

\dhatu{kisalay} \pres{kisalayati} \gand{Denom.} \pppp{kisalayita} \artha{sprießen lassen}
\Zusatz{kisala(ya) m.\ \artha{Spross}}

\dhatu{kīrtay} \pres{kīrtayati} \gand{Denom.} \pppp{kīrtita} \artha{verkünden, rühmen}
\forms{
\PS kīrtyate
\GDV kīrtayitavya, kīrtanīya
\INFIN kīrtayitum
\ABS kīrtayitvā
}
\Zusatz{v.\ kīrti f.\ \artha{Erwähnung; Ruhm}}

\dhatu{kuc/kuñc} \pres{kuñcate \gan{I P} | kucati \gan{VI P}}, \pppp{ku(ñ)cita} \artha{sich zusammenziehen, krümmen} 
\forms{
\PF cukoca
\PS kucyate
\CS kocayati/kuñcayati
}
\Zusatz{saṃ-kucita \artha{zusammengezogen, geschlossen}}

\dhatu{kuṭṭ} \pres{kuṭṭayati \gan{X PĀ}}, \pppp{kuṭṭita} \artha{zerquetschen} 
\forms{
\PPF kuṭṭayām āsa
\ABS \mbox{-kuṭṭya}
}

\dhatu{kutsay} \pres{kutsayati} \gand{Denom.} \pppp{kutsita} \artha{schmähen, tadeln}
\forms{
\GDV kutsya, kutsitavya, kutsanīya
}
\Zusatz{kutsā f.\ \artha{Tadel}}

\dhatu{kup} \pres{kupyati; -te \gan{IV P}}, \pppp{kupita} \artha{zürnen} 
\forms{
\PF cukopa
\INJ mā kopiṣṭhāḥ
\PS kupyate
\GDV kopya
\ABS kupitvā
\CS kopayati; -te
}

\dhatu{kuṣ} \pres{kuṣṇāti \gan{IX P} | kuṣati}, \pppp{kuṣita} \artha{zerren} 

\dhatu{kūj} \pres{kūjati; -te \gan{I P}}, \pppp{kūjita} \artha{brummen, summen, zwitschern [u.\ ähnliche Laute]}

\forms{
\PF cukūja; cukūje
\PS \AO akūji
\GDV kūjitavya
\ABS kūjitvā
}

\dhatu{kūrd} \pres{kūrdati; -te \gan{I PĀ}}, \pppp{kūrdita} \artha{hüpfen}
\forms{\PF cukūrda}

\dhatu{kṛ} \pres{karoti; kurute \gan{VIII PĀ}}, \pppp{kṛta} \artha{machen, tun} 
\forms{
\IND karomi, karoṣi, karoti, kurvaḥ, kuruthaḥ, kurutaḥ, kurmaḥ, kurutha, kurvanti; kurve, kuruṣe, kurute, kurvahe, kurvate
\OP kuryāt; kurvīta
\IPV karavāṇi, kuru, karotu, karavāva, kurvantu; karavai, kuruṣva, kurutām,  karavāmahai,  kurvatām
\IMP akaravam, akaroḥ, akarot, akurva, akurvan; akurvi, akuruthāḥ, akuruta, akurvahi, akurvata
\PF cakāra, cakruḥ; cakre
\AO akārṣam, akārṣīḥ, akārṣīt, akārṣva, akārṣtam, akārṣṭām, akārṣma, akārṣṭa, akārṣuḥ; akṛṣi, akṛthāḥ, akṛta, akṛṣvahi, akṛṣata
\FT kariṣyati; kariṣyate
\PFT kartā
\PS kriyate \PS \AO akāri
\GDV kṛtya/kārya, kartavya, karaṇīya
\INFIN kartum
\ABS kṛtvā, \mbox{-kṛtya}
\CS kārayati \CS \IMP akārayat \CS \PPF kārayām āsa \CS \AO acīkarat \CS \GDV kārayitavya, \mbox{-kāraṇīya}
\DS cikīrṣati
}
\Zusatz{adhi-kṛtya \artha{in Bezug auf, in Betreff von}; einige Ableitungen auf \vw{S.\,\pageref{mindmap-kr}}}

\dhatu{kṛt} \pres{kṛntati \gan{VI P} | kartati}, \pppp{kṛtta} \artha{schneiden} 
\forms{
\PF cakarta, cakartitha, cakartuḥ; cakartire
\FT kartiṣyati
\PS kṛtyate
\GDV kartya, karttavya
\INFIN kartitum
\ABS kartitvā, \mbox{-kṛtya}
\CS kartayati
\DS cikartiṣati
}

\dhatu{kṛś} \pres{kṛśyati \gan{IV P}}, \pppp{kṛśita} \artha{abmagern} 
\forms{
\CS karśayati \CS \PPP karśita
}
\Zusatz{kṛśa \artha{dünn, schwach}}

\dhatu{kṛṣ} \pres{karṣati; -te \gan{I P} | kṛṣati; -te \gan{VI PĀ}}, \pppp{kṛṣṭa} \artha{ziehen, pflügen} 
\forms{
\PF cakarṣa, cakarṣitha, cakarṣa, cakṛṣiva
\FT krakṣyati
\PS kṛṣyate
\GDV kraṣṭavya, karṣaṇīya
\INFIN kraṣṭum
\ABS kṛṣṭvā, \mbox{-kṛṣya}
\CS karṣayati
}

\dhatu{kṝ} \pres{kirati \gan{VI P}}, \pppp{kīrṇa} \artha{ausstreuen} 
\forms{
\OP kīryāt
\PF cakāra, cakaruḥ; cakre
\FT kariṣyati
\PS kīryate
\GDV \mbox{-kīrya}
\ABS \mbox{-kīrya}
\CS -kīrayati \CS \OP -kīrayet
}

\dhatu{kḷp} \pres{kalpate \gan{I Ā}}, \pppp{kḷpta} \artha{passend/fähig sein}
\forms{
\PF cakḷpe
\FT kalpiṣyate
\GDV kalpya, kalpitavya, kalpanīya
\CS kalpayati; -te \CS \AO acīkḷpat \CS \GDV kalpayitavya
}

\dhatu{krand} \pres{krandati; -te \gan{I PĀ}}, \pppp{krandita} \artha{wiehern, brüllen, schreien, tönen} 
\forms{
\PF cakranda
\AO akrandīt
\PS krandyate
\GDV \mbox{-krandanīya}
\INFIN kranditum
\CS krandayati 
}

\dhatu{kram} \pres{krāmati; kramate \gan{I PĀ}}, \pppp{krānta} \artha{schreiten} 
\forms{
\PF cakrāma, cakramuḥ; cakrame
\AO akramīt
\FT kramiṣyati; -te
\PS kramyate
\GDV \mbox{-kramya}, \mbox{-kramitavya}, \mbox{-kramaṇīya}
\INFIN krāntum/kramitum
\ABS krāntvā/kramitvā, \mbox{-kramya}
\CS kr\shortlonga{}mayati \CS \GDV \mbox{-krāmya}, \mbox{-krāmayitavya}
\DS cikramiṣati
\IS caṅkramīti; caṅkramyate
}

\dhatu{krī} \pres{krīṇāti; krīṇīte \gan{IX PĀ}}, \pppp{krīta} \artha{kaufen} 
\forms{
\PF cikrāya
\FT kreṣyati; -te
\PS krīyate
\GDV kreya/krayya, kretavya
\INFIN kretum
\ABS krītvā, \mbox{-krīya}
\DS  cikrīṣate
}

\dhatu{krīḍ} \pres{krīḍati; -te \gan{I P}}, \pppp{krīḍita} \artha{spielen, scherzen} 
\forms{
\PF cikrīḍa; cikrīḍe
\FT krīḍiṣyati
\INFIN krīḍitum
\ABS krīḍitvā, \mbox{-krīḍya}
\CS krīḍayati/krīḍāpayati
}

\dhatu{krudh} \pres{krudhyati; -te \gan{IV P}}, \pppp{kruddha} \artha{zürnen} 
\forms{
\PF cukrodha, cukrudhuḥ
\PS krudhyate
\GDV krodhanīya
\INFIN kroddhum
}

\dhatu{kruś} \pres{krośati; -te \gan{I P}}, \pppp{kruṣṭa} \artha{schreien, wehklagen} 
\forms{
\PF cukrośa
\FT krokṣyati
\PS kruśyate
\INFIN kroṣṭum
\CS krośayati
}

\dhatu{klam} \pres{klāmyati \gan{IV P}}, \pppp{klānta} \artha{ermüden} 
\forms{
\PF caklame
\CS klāmayati
}

\dhatu{klid} \pres{klidyate; -ti \gan{IV Ā}}, \pppp{klinna} \artha{nass werden} 
\forms{
\GDV kledya
\CS kledayati
}

\dhatu{kliś} \pres{kliśyati \gan{IV P} | kliśnāti \gan{IX P}}, \pppp{kliṣṭa} \artha{quälen} 
\forms{
\PF cikleśa
\PS kliśyate
\ABS kliṣṭvā, \mbox{-kliśya}
\CS kleśayati
}

\dhatu{kvaṇ} \pres{kvaṇati \gan{I P}}, \pppp{kvaṇita} \artha{tönen} 
\forms{
\CS kvaṇayati
}

\dhatu{kvath} \pres{kvathati; -te \gan{I P}}, \pppp{kvathita} \artha{sieden} 
\forms{
\CS kvāthayati \CS \PS kvāthyate \CS \GDV kvāthayitavya
}

\dhatu{kṣan} \pres{kṣaṇoti; kṣaṇute \gan{VIII PĀ}}, \pppp{kṣata} \artha{verwunden} 
\forms{
\INFIN kṣantum
\ABS kṣa(ṇi)tvā
}

\dhatu{kṣap} \pres{kṣapati; -te}, \pppp{---} \artha{sich kasteien}
\forms{
\INFIN kṣapitum
}

\dhatu{kṣam} \pres{kṣamate; -ti \gan{I Ā} | kṣamyate \gan{IV P}}, \pppp{kṣānta/kṣamita} \artha{erdulden; verzeihen} 
\forms{
\PF cakṣame; cakṣāma
\FT kṣaṃsyati/kṣamiṣyati
\PS kṣamyate
\GDV kṣāmya, kṣamitavya/kṣantavya, kṣamaṇīya
\INFIN kṣantum
\CS kṣamayati
}

\dhatu{kṣamāpay} \pres{kṣamāpayati} \gand{Denom.}, \pppp{kṣamāpita} \artha{um Verzeihung bitten}
\forms{
\PPF kṣamāpayām āsa
\INFIN kṣamāpayitum
\ABS kṣamāpayitvā/kṣamāpya
}
\Zusatz{kṣamā f.\ \artha{Duldsamkeit, Nachsicht}}


\dhatu{kṣar} \pres{kṣarati; -te \gan{I P}}, \pppp{kṣarita} \artha{fließen} 
\forms{
\PF cakṣāra
\CS kṣārayati
}

\dhatu{kṣal} \pres{kṣālayati \gan{X PĀ}}, \pppp{kṣālita} \artha{waschen} 
\forms{
\PPF kṣālayām āsa/cakāra
\GDV kṣālya
\INFIN kṣālayitum
\ABS kṣālayitvā, \mbox{-kṣālya}
}

\dhatu{kṣi} \pres{kṣayati \gan{I P} | kṣiṇoti \gan{V P}}, \pppp{kṣita/kṣīṇa} \artha{zerstören, schwinden} 
\forms{
\FT kṣayiṣyati
\PS kṣīyate
\GDV \mbox{(a-)kṣay(y)a},
\INFIN kṣetum
\CS kṣayayati/kṣapayati \CS \PPP kṣayita/kṣapita \CS \GDV kṣayayitavya/kṣapayitavya
}

\dhatu{kṣip} \pres{kṣipati; -te \gan{VI PĀ}}, \pppp{kṣipta} \artha{werfen} 
\forms{
\IPV kṣipāṇi; kṣipai
\PF cikṣepa, cikṣipe
\FT kṣepsyati; -te
\PS kṣipyate
\GDV kṣepya, kṣeptavya
\INFIN kṣeptum
\ABS kṣiptvā, \mbox{-kṣipya}
\CS kṣepayati
\DS cikṣipsati
}

\dhatu{kṣudh} \pres{kṣudhyati \gan{IV P}}, \pppp{kṣudhita} \artha{hungrig sein} 
\forms{
}


\dhatu{kṣubh} \pres{kṣubhyati; -te \gan{IV P}}, \pppp{kṣubdha/kṣubhita} \artha{zittern} 
\forms{
\PF cukṣobha; cukṣubhe
\GDV kṣobhya
\CS kṣobhayati; -te
}

\medskip

\dhatu{khaṇḍ} \pres{khaṇḍayati \gan{X PĀ}}, \pppp{khaṇḍita} \artha{zerstückeln; widerlegen}
\forms{
\GDV khaṇḍya, khaṇḍitavya, khaṇḍanīya
}

\dhatu{khan/khā} \pres{khanati; -te \gan{I PĀ}}, \pppp{khāta} \artha{graben} 
\forms{
\PF cakhāna, cakhnuḥ
\FT khaniṣyati
\PS khanyate/khāyate
\GDV \mbox{-kheya}, khananīya
\INFIN khanitum
\ABS khātvā/khanitvā, \mbox{-khāya}
\CS khānayati
}

\dhatu{khād} \pres{khādati \gan{I P}}, \pppp{khādita} \artha{essen} 
\forms{
\PF cakhāda
\FT khādiṣyate
\PS khādyate
\GDV khādya, khāditavya, khādanīya
\INFIN khāditum
\ABS khāditvā
\CS khādayati
\DS cikhādiṣati
}

\dhatu{khid} \pres{khidyate; -ti \gan{IV Ā}}, \pppp{khinna} \artha{niedergeschlagen sein} 
\forms{
\PS khidyate
\GDV kheditavya
\CS khedayati; -te \CS \GDV khedayitavya
}

\dhatu{khyā} \pres{khyāti \gan{II P}}, \pppp{khyāta} \artha{erzählen; sichtbar werden} 
\forms{
\IPV khyāhi, khyātu
\PF cakhyau, cakhyuḥ
\AO akhyat
\FT khyāsyati; \mbox{-te}
\PS khyāyate
\GDV \mbox{-khyeya}, \mbox{-khyātavya}
\INFIN khyātum
\ABS \mbox{-khyāya}
\CS khyāpayati; -te \CS \PS khyāpyate \CS \GDV khyāpayitavya, khyāpanīya
\DS cikhyāsati \DS \PPP cikhyāsita
}
\Zusatz{ākhyāta \artha{genannt}, ākhyāna n.\ \artha{Erzählung}}

\medskip

\dhatu{gacch} \vw{gam}

\dhatu{gaṇ} \pres{gaṇayati; -te \gan{X PĀ}}, \pppp{gaṇita} \artha{zählen; erwägen; halten für} 
\forms{
\PPF gaṇayām āsa/cakāra
\AO ajīgaṇat
\PS gaṇyate
\GDV gaṇya, gaṇayitavya, gaṇanīya
\INFIN  gaṇayitum
\ABS gaṇayitvā, \mbox{-gaṇayya}
}

\dhatu{gad} \pres{gadati \gan{I P}}, \pppp{gadita} \artha{sprechen} 
\forms{
\PF jagāda; jagade
\FT gadiṣyati; -te
\PS gadyate \PS \AO agādi
\GDV gadya
\INFIN gaditum
\ABS gaditvā
\CS gādayati 
\DS jigadiṣati
\IS jāgadyate
}

\dhatu{gam} \pres{gacchati; -te \gan{I P}}, \pppp{gata} \artha{gehen} 
\forms{
\PF jagāma, jagmuḥ; jagme
\AO agamat
\FT gamiṣyati; -te
\PFT gantā
\PS gamyate \PS \AO agami
\PPFA jagmivāṃs
\GDV gamya, gantavya, gamanīya
\INFIN gantum
\ABS gatvā, \mbox{-gamya}/\mbox{-gatya}
\CS gamayati
\CS \AO ajīgamat
\CS \GDV gamayitavya
\DS jigamiṣati
\IS jaṅganti
\IS \PS jaṅgamyate
}

\dhatu{garj} \pres{garjati; -te \gan{I P}}, \pppp{garjita} \artha{brüllen, donnern} 
\forms{
\PF jagarja
\INFIN garjitum
\ABS garjitvā, \mbox{-garjya}
}

\dhatu{garh} \pres{garhate; -ti \gan{I Ā} | garhayate; -ti \gan{X ĀP}}, \pppp{garhita} \artha{tadeln}

\forms{
\PF jagarhe; jagarha
\FT garhiṣyate; -ti
\PS garhyate
\GDV garhya, garhitavya, garhaṇīya
\INFIN garhitum
\ABS garhitvā, \mbox{-garhya}
}

\dhatu{gal} \pres{galati \gan{I P}}, \pppp{galita} \artha{herabfallen; verschwinden} 
\forms{
\INFIN galitum
\CS gālayati 
\IS galgalyate
}

\dhatu{galbh} \pres{(pra-)galbhate \gan{I Ā}}, \pppp{(pra-)galbhita} \artha{entschlossen sein, wagen} 
\forms{
\PF \mbox{-jagalbhe}
}

\dhatu{gā [1]} \artha{gehen} 
\forms{
  \AO agāt
}
\Zusatz{substituiert die Aorist-Formen der Wurzel \vw{i}}

\dhatu{gā [2]} \vw{gai}

\dhatu{gāh} \pres{gāhate; -ti \gan{I Ā}}, \pppp{gāḍha/gāhita} \artha{eintauchen} 
\forms{
\PF jagāhe
\FT gāhiṣyate
\PS gāhyate
\INFIN gāhitum
\ABS gāhitvā, \mbox{-gāhya}
\CS gāhayati
}

\dhatu{gir/gil} \vw{gṝ}

\dhatu{gup} \pres{gop\shortlonga{}yati | --- \gan{I P}}, \pppp{gupta/gopita/gopāyita} \artha{hüten; verbergen}
\forms{
\PF jugopa, jugupuḥ
\FT gopsyati
\PS gupyate/gopyate
\GDV gopya, goptavya/gop\shortlonga{}yitavya, gopanīya
\INFIN gop(i)tum/gop\shortlonga{}yitum
\ABS guptvā/gop\shortlonga{}yitvā
\DS jugupsate; -ti \artha{sich hüten vor} \DS \PPP jugupsita
}
\Zusatz{gopa m.\ \artha{Kuhhirte}}

\dhatu{guh} \pres{gūhati; -te \gan{I PĀ}}, \pppp{gūḍha} \artha{verbergen} 
\forms{
\PF jugūha; juguhe, jugūhire
\AO aghukṣat
\PS guhyate
\GDV guhya/\mbox{-gohya}, gūhitavya
\INFIN gūhitum
\ABS \mbox{-guhya}
\CS gūhayati
}

\dhatu{gṛdh} \pres{gṛdhyati \gan{IV P}}, \pppp{gṛddha} \artha{gierig sein} 
\forms{
\PF jagṛdhuḥ
\CS gṛddhvā
}

\dhatu{gṛh} \vw{grah}

\dhatu{gṝ [1]} \pres{gṛṇāti; gṛṇīte \gan{IX PĀ} | girate; -ti}, \pppp{gīrṇa} \artha{preisen; verkünden} 
\forms{
}

\dhatu{gṝ [2]} \pres{girati/gilati; girate \gan{VI PĀ}}, \pppp{gīrṇa/gilita} \artha{verschlingen} 
\forms{
\PF jagāra
\FT gariṣyati
\PS gīryate
\INFIN gar\shortlongi{}tum
}

\dhatu{gai} \pres{gāyati; -te \gan{I P}}, \pppp{gīta} \artha{singen} 
\forms{
\PF jagau; jage
\AO agāsīt
\FT gāsyati
\PS gīyate
\GDV geya, gātavya
\INFIN gātum
\ABS gītvā, \mbox{-gīya}
\CS gāpayati; -te
}

\dhatu{gopay/gopāy} \vw{gup} 

\dhatu{grath/granth} \pres{grathnāti \gan{IX P}}, \pppp{grathita} \artha{zusammenbinden, verfassen} 
\forms{
\PF jagrantha
\PS grathyate
\GDV grathya, grathitavya, grathanīya
\INFIN grathitum
\ABS granthitvā, \mbox{-grathya}
\CS grathayati; granthayate 
}

\dhatu{grabh} \vw{grah}

\dhatu{gras} \pres{grasate; -ti \gan{I Ā}}, \pppp{grasta} \artha{verschlingen} 
\forms{
\PF jagrase
\AO agrasīt
\FT grasiṣyati; -te
\PS grasyate
\GDV grasya, grasanīya
\INFIN gras(i)tum
\ABS gras(i)tvā, \mbox{-grasya}, grāsam
\CS grāsayati 
}

\dhatu{grah} \pres{gṛhṇāti; gṛhṇīte \gan{IX PĀ}}, \pppp{gṛhīta} \artha{greifen} 
\forms{
\IPV gṛhāṇa,  gṛhṇātu
\PF jagrāha, jagṛhuḥ; jagṛhe
\AO agrahīt; agrahīṣṭa
\FT grahīṣyati; -te
\PFT grahītā
\PS gṛhyate
\GDV gṛhya, grahītavya/gṛhītavya, grahaṇīya
\INFIN grahītum
\ABS gṛhītvā \artha{mit}, \mbox{-gṛhya}
\CS grāhayati; -te \CS \AO ajigrahat \CS \GDV grāhya
\DS jighṛkṣati; -te
}

\dhatu{glai} \pres{glāyati; -te \gan{I P} | glāti}, \pppp{glāna} \artha{erschöpft, verdrossen, entmutigt sein} 
\forms{
\INJ mā glāsīḥ
\CS gl\shortlonga{}payati
}

\medskip

\dhatu{ghaṭ} \pres{ghaṭate; -ti \gan{I Ā}}, \pppp{ghaṭita} \artha{sich bemühen; sich ereignen} 
\forms{
\PF jaghaṭe
\FT ghaṭiṣyate
\INFIN ghaṭitum
\CS gh\shortlonga{}ṭayati; ghaṭayate \artha{verfertigen} 
\CS \GDV ghaṭayitavya, \mbox{-ghāṭanīya}
\IS jāghaṭīti
}
\Zusatz{na ghaṭate \artha{ist nicht passend}}

\dhatu{ghaṭṭ} \pres{ghaṭṭayati \gan{X PĀ}}, \pppp{ghaṭṭita} \artha{schütteln} 
\forms{
\PF jaghaṭṭire
\PS ghaṭṭyate
}

\dhatu{gharṣ} \vw{ghṛṣ}

\dhatu{ghas/jakṣ} \pres{--- \gan{I P} | jakṣiti \gan{II P} | jakṣati}, \pppp{jagdha} \artha{essen, verschlingen} 
\forms{
\IPV jagdhi
\AO aghasat
\INFIN jagdhum
\ABS jagdhvā
\DS jighatsati
}
\Zusatz{ergänzt in einigen Formen das unvollständige Paradigma der Wurzel \vw{ad}}

\dhatu{ghātay} \pres{ghātayati; -te} \gand{Denom.}, \pppp{ghātita} \artha{schlagen}
\forms{
\GDV ghātya, ghātanīya
}
\Zusatz{ghāta m.\ \artha{Schlag}}

\dhatu{ghuṣ} \pres{ghoṣati; -te \gan{I PĀ}}, \pppp{ghuṣṭa} \artha{tönen} 
\forms{
\PS ghuṣyate
\GDV ghuṣya
\ABS \mbox{-ghuṣya}
\CS ghoṣayati  \CS \PS ghoṣyate \CS \GDV ghoṣaṇīya
}

\dhatu{ghūrṇ} \pres{ghūrṇati \gan{VI P} | ghūrṇate \gan{I Ā}}, \pppp{ghūrṇita} \artha{schwanken} 
\forms{
\PF jughūrṇa; -ṇe
\CS ghūrṇayati
}
\dhatu{ghṛṣ} \pres{gharṣati; -te \gan{I P}}, \pppp{ghṛṣṭa} \artha{reiben, zerreiben} 
\forms{
\PF jagharṣa
\PS ghṛṣyate
\ABS ghṛṣṭvā, \mbox{-ghṛṣya}
\CS \PPP gharṣita
}

\dhatu{ghrā} \pres{jighrati; -te \gan{I P}}, \pppp{ghrāta} \artha{riechen} 
\forms{
\PF jaghrau
\PS ghrāyate
\GDV ghreya, ghrātavya
\INFIN ghrātum
\ABS jighritvā, \mbox{-ghrāya}/\mbox{-jighrya}
\CS ghrāpayati
}

\medskip

\dhatu{cakās} \pres{cakāsti \gan{II P}}, \pppp{cakāsita} \artha{strahlen} 

\dhatu{cakṣ} \pres{caṣṭe \gan{II Ā} | cakṣati; -te}, \pppp{---} \artha{erblicken; verkünden} 
\forms{
\IND cakṣe, cakṣe, caṣṭe, cakṣmahe, caḍḍhve, cakṣate | cakṣati; -te
\PF cacakṣe
\FT cakṣyate
\INFIN caṣṭum
\ABS \mbox{-cakṣya}
\CS cakṣayati \CS \PPP cakṣita
}

\dhatu{cam} \pres{camati/(ā-)cāmati \gan{I P}}, \pppp{cānta} \artha{schlürfen} 
\forms{
\PF cacāma, cemuḥ
\FT camiṣyati
\GDV cāmya
\ABS (ā-)camya
\CS cāmayati
}

\dhatu{car} \pres{carati; -te \gan{I P}}, \pppp{carita} \artha{sich bewegen, gehen, ausüben} 
\forms{
\PF cacāra, cacartha, ceruḥ; cere
\FT cariṣyati; -te
\PS caryate
\GDV carya/\mbox{-cārya}, car(i)tavya, caraṇīya
\INFIN car(i)tum
\ABS caritvā, \mbox{-carya}
\CS cārayati; -te 
}
\Zusatz{ā-cāra m.\ \artha{(gutes) Benehmen}}

\dhatu{carc} \pres{carcayati \gan{X PĀ}}, \pppp{carcita} \artha{lesen, erwähnen} 
\Zusatz{ursprünglich: \artha{(ein Wort bei der schulmäßigen Rezitation des Veda) wiederholen}, candana-carcita \artha{mit Sandel bedeckt o.\ bestrichen}.}

\dhatu{carv} \pres{carvayati; -te \gan{X PĀ}}, \pppp{carvita/cūrṇa} \artha{kauen, schmecken} 
\forms{
\PS carvyate
\GDV carvya
}

\dhatu{cal} \pres{calati; -te \gan{I | VI P}}, \pppp{calita} \artha{in Bewegung kommen, zittern} 
\forms{
\PF cacāla, celuḥ
\FT caliṣyati
\GDV calitavya
\INFIN calitum
\CS calayati/cālayati; -te  \CS \GDV cālya
\DS cicaliṣati
}

\dhatu{ci} \pres{cinoti; cinute \gan{V PĀ}}, \pppp{cita} \artha{sammeln; bemerken}
\forms{
\IND cinoti, cinvati; cinute
\PF cikāya; cikye, cikyire
\FT ceṣyati; -te
\PFT cetā
\PS cīyate
\GDV cayanīya
\INFIN cetum
\ABS citvā, \mbox{-citya}
\CS cāyayate
\DS cikīṣate/cicīṣati
}

\dhatu{cikits} \vw{cit}

\dhatu{cit} \pres{cetati \gan{I P}}, \pppp{citta} \artha{wahrnehmen} 
\forms{
\PF  ciketa/ciceta
\CS cetayati
\CS \GDV cetayitavya
\DS cikitsati; -te \artha{beabsichtigen; heilen} 
\DS \CS cikitsayati \artha{heilen}
\DS \CS \GDV cikitsya, cikitsanīya
}
\Zusatz{cit f.\ \artha{Bewusstsein}; citta n.\ \artha{der Geist} i.\,S.\,v.\ \artha{das Denken}; cikitsā f.\ \artha{Heilkunde}}

\dhatu{citr} \pres{citrayati \gan{X PĀ}}, \pppp{citrita} \artha{bunt machen, zeichnen} 

\dhatu{citrīy} \pres{citrīyate} \gand{Denom.} \pppp{---} \artha{erstaunen}
\Zusatz{citra \artha{bunt; wunderbar}, n.\ \artha{Schmuck, Bild; Wunder}}

\dhatu{cint} \pres{cintayati; -te \gan{X PĀ}}, \pppp{cintita} \artha{(be)denken} 
\forms{
\PPF cintayām āsa/cakāra
\FT cintayiṣyati
\PS cintyate
\GDV cintya, cintayitavya, cintanīya
\INFIN cintayitum
\ABS cintayitvā, \mbox{-cintya}
}
\Zusatz{cintā f.\ \artha{Sorge}}


\dhatu{ciray} \pres{cir\shortlonga{}yati} \gand{Denom.} \pppp{cir\shortlonga{}yita} \artha{zögern, säumen}
\Zusatz{v.\ cira \artha{lang (Zeit)}}

\dhatu{cihnay} \pres{cihnayati} \gand{Denom.} \pppp{cihnita} \artha{(kenn)zeichnen}
\Zusatz{cihna n.\ \artha{Zeichen, Merkmal}}

\dhatu{cud} \pres{codayati; -te \gan{X PĀ}}, \pppp{codita} \artha{antreiben} 
\forms{
\PPF codayām āsa
\PS codyate
\GDV codya, codayitavya, codanīya
\INFIN codayitum
\ABS codayitvā, \mbox{-codya}
}
\Zusatz{in der Rechtslit.\ auch: \artha{vorschreiben}, vidhi-codita \artha{durch eine Regel vorgeschrieben}}

\dhatu{cumb} \pres{cumbati \gan{I P}}, \pppp{cumbita} \artha{küssen} 
\forms{
\PF cucumba
\PS cumbyate
\INFIN cumbitum
\ABS \mbox{-cumbya}
\DS cucumbiṣati
}

\dhatu{cur} \pres{corayati \gan{X PĀ}}, \pppp{corita} \artha{stehlen} 
\forms{
\PPF corayāṃ cakāra
\AO acūcurat
\PS coryate
\GDV corayitavya
\ABS corayitvā
}

\dhatu{cūrṇ} \pres{cūrṇayati \gan{X PĀ}}, \pppp{cūrṇita} \artha{zerreiben} 
\forms{
\PPF cūrṇayām āsa
\ABS cūrṇayitvā
}

\dhatu{cūṣ} \pres{cūṣati \gan{I P}, cūṣayati}, \pppp{cūṣita} \artha{saugen} 
\forms{
\PS cūṣyate
\GDV cūṣya\altern coṣya
}


\dhatu{ceṣṭ} \pres{ceṣṭati; -te \gan{I Ā}}, \pppp{ceṣṭita} \artha{sich regen, sich abmühen} 
\forms{
\PF ciceṣṭa
\GDV ceṣṭitavya
\INFIN ceṣṭitum
\ABS ceṣṭitvā
\CS ceṣṭayati; -te
}

\dhatu{cyu} \pres{cyavate; -ti \gan{I Ā}}, \pppp{cyuta} \artha{fortgehen, herabfallen} 
\forms{
\INFIN cyavitum
\CS cyāvayati; -te 
}

\dhatu{cyut} \pres{cyotati \gan{I P}}, \pppp{cyotita} \artha{träufeln}
\vw{śc(y)ut}

\medskip

\dhatu{chad} \pres{chādayati; -te \gan{X PĀ}}, \pppp{channa/chādita} \artha{verhüllen, verbergen} 
\forms{
\PPF chādayām āsa/cakāra
\FT chādayiṣyati
\PS chādyate
\GDV chādya
\INFIN chādayitum
\ABS chādayitvā, \mbox{-chādya}
}


\dhatu{chid} \pres{chinatti \gan{VII PĀ}}, \pppp{chinna} \artha{(ab)schneiden} 
\forms{
\IND chinatti, chindanti
\PF ciccheda; cicchide
\AO acchidat/acchaitsīt
\FT chetsyati; -te
\PS chidyate
\GDV chedya, chettavya, chedanīya
\INFIN  chettum
\ABS chittvā, \mbox{-chidya}
\CS chedayati
}

\medskip

\dhatu{jakṣ/jagh} \vw{ghas/jakṣ}, \vw{ad}

\dhatu{jan} \pres{jāyate; -ti \gan{IV Ā}}, \pppp{jāta} \artha{geboren werden, entstehen} 
\forms{
\PF jajñe; jajāna, jajñuḥ
\AO ajaniṣṭa
\FT janiṣyate
\PFT janitā
\PS janyate; -ti
\PS \AO ajani
\GDV janya, janitavya, jananīya
\INFIN janitum
\ABS jātvā/janitvā, \mbox{-janya}
\CS janayati; -te
\CS \AO ajījanat
\CS \PPP janita
\CS \GDV janayitavya 
\DS jijaniṣate
}
\Zusatz{punar-janman m.\ \artha{Wiedergeburt}}

\dhatu{jap} \pres{japati; -te \gan{I P}}, \pppp{jap(i)ta} \artha{flüstern} 
\forms{
\PF jajāpa, jepuḥ
\FT japiṣyati
\PS japyate
\GDV japya/jāpya, japtavya, japanīya
\INFIN  jap(i)tum
\ABS jap(i)tvā
}

\dhatu{jalp} \pres{jalpati; -te \gan{I P}}, \pppp{jalpita} \artha{murmeln, klagen} 
\forms{
\PF jajalpa; jajalpire
\PS jalpyate
\INFIN jalpitum
}

\dhatu{jā} \vw{jan}

\dhatu{jāgṛ} \pres{jāgarti \gan{II P}}, \pppp{jāgarita} \artha{(er)wachen} 
\forms{
\IND jāgarti, jāgṛtaḥ, jāgrati
\IPV jāgarāṇi,  jāgṛhi,  jāgartu
\IMP ajāgaram, ajāgaḥ, ajāgaḥ, ajāgṛtām, ajāgaruḥ
\PF jajāgāra
\PPF jāgarām āsa/cakāra
\FT jāgariṣyati
\GDV jāgarya, jāgaritavya, jāgaraṇīya
\CS jāgarayati
}

\dhatu{ji} \pres{jayati; -te \gan{I P}}, \pppp{jita} \artha{(be)siegen} 
\forms{
\PF jigāya, jigyiva, jigyuḥ; jigye
\AO ajaiṣīt
\FT jeṣyati; -te
\PS jīyate
\GDV jeya, jetavya
\INFIN jetum
\ABS jitvā, \mbox{-jitya}
\CS jāpayati
\DS jigīṣati
}

\dhatu{jīr} \vw{jṝ}

\dhatu{jīv} \pres{jīvati; -te \gan{I P}}, \pppp{jīvita} \artha{leben} 
\forms{
\PF jijīva, jijīvuḥ
\AO ajīvīt
\FT jīviṣyati
\PS jīvyate
\GDV \mbox{-jīvya}, jīvitavya, jīvanīya
\INFIN jīvitum
\ABS jīvitvā, \mbox{-jīvya}
\CS jīvayati
\DS jijīviṣati; -te
}
\Zusatz{jīvana n.\ \artha{Leben}, jīva\altern jīvātman m.\ \artha{Seele}}

\dhatu{juṣ} \pres{juṣate; -ti \gan{VI Ā}}, \pppp{juṣṭa} \artha{sich erfreuen (an), bewohnen} 
\forms{
\PF jujuṣe; jujoṣa
\CS joṣayate; -ti
}
\Zusatz{ratna-juṣṭa \artha{mit Juwelen versehen}}

\dhatu{jṛmbh} \pres{jṛmbhate; -ti \gan{I Ā}}, \pppp{jṛmbhita} \artha{gähnen, sich öffnen} 
\forms{
\PF jajṛmbhe
\ABS jṛmbhitvā
\CS jṛmbhayati
\IS jarījṛmbhate
}

\dhatu{jṝ} \pres{jīryati; -te \gan{IV P}}, \pppp{jīrṇa} \artha{altern} 
\forms{
\PF jajāra
\PS jīryate
\ABS jar\shortlongi{}tvā
\CS jarayati; -te 
}

\dhatu{jñā} \pres{jānāti; jānīte \gan{IX PĀ}}, \pppp{jñāta} \artha{(er)kennen} 
\forms{
\PF jajñau; jajñe
\AO ajñāsīt
\FT jñāsyati; -te
\PFT jñātā
\PS jñāyate \PS \AO ajñāyi
\GDV jñeya, jñātavya
\INFIN jñātum
\ABS jñātvā, \mbox{-jñāya}
\CS (ā-)jñāpayati/jñapayati; jñapayate
\artha{mitteilen, befehlen}  \CS \AO ajijñapat \CS \PPP jñāpita/jñapta
\CS \GDV jñāpya, jñāpayitavya, jñāpanīya
\DS jijñāsate; -ti
}
\Zusatz{(vi-)jñāna n.\ \artha{Wissen, Erkenntnis}}

\dhatu{jvar} \pres{jvarayati \gan{X P}}, \pppp{jvarita} \artha{fiebern} 

\dhatu{jval} \pres{jvalati; -te \gan{I P}}, \pppp{jvalita} \artha{brennen, leuchten} 
\forms{
\PF jajvāla
\AO ajvalīt
\FT jvaliṣyati
\ABS jvalitvā, \mbox{-jvalya}
\CS jv\shortlonga{}layati
\IS jājvalīti
}

\medskip

\dhatu{ṭaṅk} \pres{ṭaṅkayati \gan{X P}}, \pppp{ṭaṅkita} \artha{binden, meißeln} 
\forms{
\PS \AO aṭaṅki
}
\Zusatz{uṭ-ṭaṅkita \artha{(Münze) geprägt, gestempelt, charakterisiert}}

\medskip

\dhatu{ḍamb} \pres{(vi-)ḍambayati \gan{X PĀ}}, \pppp{(vi-)ḍambita} \artha{nachahmen; verspotten}
\forms{
  \PS (vi-)ḍambyate
  \GDV \mbox{(vi-)ḍambya}
}
\Zusatz{vi-ḍamba m.\ \artha{Spott, Imitation}}


\dhatu{ḍī} \pres{ḍayate \gan{I Ā} | ḍīyate \gan{IV Ā}}, \pppp{ḍīna} \artha{fliegen} 
\forms{
\ABS (uḍ-)ḍīya
\CS (uḍ-)ḍāpayati
}


\medskip

\dhatu{ḍhauk} \pres{ḍhaukate \gan{I Ā}}, \pppp{ḍhaukita} \artha{sich nähern} 
\forms{
\PF ḍuḍhauke
\CS ḍhaukayati \artha{herbeischaffen, darreichen}
}

\medskip

\dhatu{takṣ} \pres{takṣati \gan{I P}}, \pppp{taṣṭa} \artha{(Holz) bearbeiten, verfertigen} 
\forms{
\PF tatakṣa
\PS takṣyate
\ABS takṣitvā/taṣṭvā, \mbox{-takṣya}
\CS takṣayati
}
\Zusatz{takṣan m.\ \artha{Zimmermann}}

\dhatu{taḍ} \pres{tāḍayati \gan{X PĀ}}, \pppp{tāḍita} \artha{schlagen, verwunden} 
\forms{
\PF tatāḍa
\FT tāḍayiṣyati
\PS tāḍyate
\GDV tāḍya, tāḍanīya
}

\dhatu{tan} \pres{tanoti; tanute \gan{VIII PĀ}}, \pppp{tata} \artha{(sich) dehnen, ausbreiten} 
\forms{
\PF tatāna, tenuḥ; tene
\AO atānīt
\FT taniṣyati; -te
\PFT tāyitā
\PS tanyate/tāyate
\ABS tatvā, \mbox{-tatya}/\mbox{-tāya}
}
\Zusatz{vi-tan \artha{ausbreiten, verfassen}}

\dhatu{tap} \pres{tapati; -te \gan{I P} | tapyati; -te \gan{IV Ā}}, \pppp{tap(i)ta} \artha{brennen; leiden} 
\forms{
\PF tatāpa; tepe
\FT tapsyati/tapiṣyati; tapsyate
\PFT taptā
\PS tapyate
\GDV tapya, taptavya, tapanīya
\INFIN taptum
\ABS taptvā, \mbox{-tapya}
\CS tāpayati  \CS \AO atītapat
}

\dhatu{tapasy} \pres{tapasyati} \gand{Denom.} \pppp{---} \artha{sich kasteien, Askese üben}
\Zusatz{tapas n.\ \artha{Askese}}

\dhatu{tam} \pres{tāmyati; -te \gan{IV P}}, \pppp{tānta} \artha{ermatten, vergehen} 
\forms{
\PF tatāma
\CS tamayati
}

\dhatu{tar} \vw{tṝ}

\dhatu{taraṅg} \pres{taraṅg(ay)ati; -te} \gand{Denom.} \pppp{taraṅgita} \artha{wogen}
\Zusatz{taraṅga m.\ \artha{Welle}}

\dhatu{taralay} \pres{taralayati} \gand{Denom.} \pppp{taralita} \artha{(sich) hin und her bewegen}
\Zusatz{tarala adj.\ \artha{unbeständig}}

\dhatu{tark} \pres{tarkayati; -te \gan{X PĀ}}, \pppp{tarkita} \artha{vermuten} 
\forms{
\PPF tarkayām āsa/cakāra
\PS tarkyate
\GDV \mbox{-tarkya}, tarkaṇīya
\INFIN tarkayitum
\ABS tarkayitvā, \mbox{-tarkya}
}
\Zusatz{tarka m.\ \artha{kritische Prüfung}}

\dhatu{tarj} \pres{tarjati; -te \gan{I P}}, \pppp{tarjita} \artha{(be)drohen} 
\forms{
\ABS \mbox{-tarjya}
\CS tarjayati 
}

\dhatu{tā} \vw{tan}

\dhatu{tikṣ} \vw{tyaj}

\dhatu{tij} \pres{tejate \gan{I Ā}}, \pppp{tikta} \artha{scharf sein; schärfen} 
\forms{
\CS tejayati  \CS \PPP tejita
}

\dhatu{timiray} \pres{timirayati} \gand{Denom.} \pppp{---} \artha{verfinstern}
\Zusatz{v. timira n.\ \artha{Finsternis}}

\dhatu{tiray} \pres{tirayati} \gand{Denom.} \pppp{---} \artha{verhüllen}
\Zusatz{tirohita \artha{verborgen}}

\dhatu{tilakay} \pres{tilakayati} \gand{Denom.} \pppp{tilakita} \artha{kennzeichnen, schmücken}
\Zusatz{tilaka n.\ \artha{Körpermal; Sektenzeichen}}

\dhatu{tīray} \pres{tīrayati} \gand{Denom.} \pppp{tīrita} \artha{vollenden}
\Zusatz{tīra n.\ \artha{Ufer}}

\dhatu{tud} \pres{tudati; -te \gan{VI PĀ}}, \pppp{tunna} \artha{stoßen, stechen} 
\forms{
\PF tutoda
\PS tudyate
\GDV todya
\CS todayati
}

\dhatu{tul} \pres{tolayati/tulayati; tulayate \gan{X PĀ}}, \pppp{tolita/tulita} \artha{prüfen, vergleichen; gleich\-kommen} 
\forms{
\PPF tolayām āsa
\PS tolyate/tulyate
\GDV tolya/tulya
\ABS tolayitvā/tulayitvā
}
\Zusatz{tulā f.\ \artha{Waage}}

\dhatu{tuṣ} \pres{tuṣyati; -te \gan{IV P}}, \pppp{tuṣṭa} \artha{zufrieden sein} 
\forms{
\PF tutoṣa
\PS tuṣyate
\GDV \mbox{-toṣya}, \mbox{-toṣṭavya}, \mbox{-toṣaṇīya}
\INFIN toṣṭum
\ABS \mbox{-tuṣya}
\CS toṣayati \CS \PPF toṣayām āsa \CS \PS toṣyate \CS \GDV toṣayitavya
}
\Zusatz{saṃtoṣa m.\ \artha{Zufriedenheit}}

\dhatu{tūr} \vw{tvar}

\dhatu{tṛp} \pres{tṛpyati; -te \gan{IV P}}, \pppp{tṛpta} \artha{satt o.\ befriedigt werden, genießen} 
\forms{
\PF tatarpa, tatṛpiva
\PS tṛpyate
\CS tarpayati; -te  \CS \AO atītṛpat
}
\Zusatz{tarpaṇa n.\ \artha{Wasserspende}}

\dhatu{tṛṣ} \pres{tṛṣyati \gan{IV P}}, \pppp{tṛṣita} \artha{dürsten} 
\forms{
\CS tarṣayati \CS \PPP tarṣita
}
\Zusatz{tṛṣnā f.\ \artha{Durst}}

\dhatu{tṝ} \pres{tarati; -te \gan{I P}}, \pppp{tīrṇa} \artha{über etwas (eig.\ ein Gewässer) setzen, entkommen, meistern} 
\forms{
\PF tatāra, teruḥ; tatare
\AO atārṣīt/atārīt
\FT tariṣyati; -te
\PS tīryate
\GDV tārya, tar(i)tavya, taraṇīya
\INFIN tar(\shortlongi{})tum
\ABS tīrtvā, \mbox{-tīrya}
\CS tārayati; -te \CS \GDV tāraṇīya
\DS titīrṣati
}
\Zusatz{tārā f.\ \artha{Stern}}

\dhatu{tyaj} \pres{tyajati; -te \gan{I P}}, \pppp{tyakta/tyajita} \artha{verlassen, entsenden, (Pfeil) abschießen} 
\forms{
\PF tatyāja; tatyaje
\AO atyākṣīt
\FT tyakṣyati/tyajiṣyati; -te
\PS tyajyate
\GDV tyājya, tyaktavya, tyajanīya
\INFIN tyaktum
\ABS tyaktvā, \mbox{-tyajya} \Zusatz{oft nur: \artha{ohne, außer}}
\CS tyājayati
\DS titikṣati; -te \artha{erdulden}
\DS \IPV titikṣasva
}
\Zusatz{tyāga m.\ \artha{Aufgeben, Verlassen}}


\dhatu{trap} \pres{trapate; -ti \gan{I Ā}}, \pppp{trapta} \artha{sich schämen} 
\forms{
\GDV \mbox{-trapaṇīya}
\CS trapayati
}
\Zusatz{trapā f.\ \artha{Scham}}

\dhatu{tras} \pres{trasati; -te | trasyati; -te \gan{IV P}}, \pppp{trasta} \artha{erzittern, erschrecken} 
\forms{
\PF tatrāsa, tatrasuḥ/tresuḥ; tatrasire
\FT trasiṣyati
\GDV trāsanīya
\CS trāsayati \CS \PS trāsyate
}
\Zusatz{trāsa m.\ \artha{Furcht}}

\dhatu{trā/trai} \pres{trāti | trāyate \gan{I Ā}}, \pppp{trāta} \artha{beschützen} 
\forms{
\IPV trāhi, trātu
\FT trāsyate; -ti
\PS trāyate
\GDV trātavya
\INFIN trātum
\ABS trātvā
}
\Zusatz{trātṛ m.\ \artha{Beschützer}, trāṇa n.\ \artha{Schützen}}

\dhatu{truṭ} \pres{truṭati \gan{VI P} | truṭyati}, \pppp{truṭita} \artha{zerbrechen} 
\forms{
\PF tutroṭa
\CS troṭayati 
}

\dhatu{trai} \vw{trā/trai}

\dhatu{tvar} \pres{tvarate; -ti \gan{I Ā}}, \pppp{tvarita/tūrṇa} \artha{eilen} 
\forms{
\PF tatvare
\GDV tvaraṇīya
\CS tvarayati; -te \CS \AO atitvarat \CS \PS tvaryate
\ABS tvaritvā
}
\Zusatz{tvarā f.\ \artha{Eile}, tvarayā adv.\ \artha{eilig}}

\medskip

\dhatu{daṃś} \pres{daśati; -te \gan{I P}}, \pppp{daṣṭa} \artha{beißen} 
\forms{
\PF dadaṃśa, dadaṃśuḥ
\FT daśiṣyati/daṅkṣyati
\PS daśyate
\ABS daṃṣṭvā, \mbox{-daśya}
\CS daṃśayati  \CS \PPP daṃśita
}
\Zusatz{daṃṣṭra m.\ \artha{Reißzahn}}

\dhatu{daṇḍ} \pres{daṇḍayati \gan{X PĀ}}, \pppp{daṇḍita} \artha{strafen} 
\forms{
\PS daṇḍyate
\GDV daṇḍya, daṇḍanīya
\INFIN daṇḍayitum
\ABS daṇḍayitvā
}
\Zusatz{daṇḍa m.\ \artha{Stock; Strafe}}

\dhatu{danturay} \pres{danturayati} \gand{Denom.} \pppp{danturita} \artha{dicht besetzt, voll von}
\Zusatz{danta m.\ \artha{Zahn} -- dantura \artha{voller Zähne\altern Spitzen, uneben}}


\dhatu{dam} \pres{dāmyati \gan{IV P}}, \pppp{dānta} \artha{zähmen, bezwingen} 
\forms{
\ABS damitvā, \mbox{-damya}
\CS damayati \CS \PPP damita
}
\Zusatz{dama m.\ \artha{Selbstbeherrschung}}

\dhatu{day} \pres{dayate \gan{I Ā}}, \pppp{dayita} \artha{teilen; teilnehmen} 
\forms{
\PPF dayām āsa
\CS \OP dayayet
}
\Zusatz{dayita \artha{geliebt}, m.\ \artha{Geliebter}}

\dhatu{daridrā} \vw{drā [1]}

\dhatu{dal} \pres{dalati \gan{I P}}, \pppp{dalita} \artha{bersten} 
\forms{
\AO adalīt
\FT daliṣyati; -te
\CS d\shortlonga{}layati lyate
}
\Zusatz{dala m.\ \artha{Blatt; Stück}}

\dhatu{dah} \pres{dahati; -te \gan{I P} | dahyati}, \pppp{dagdha} \artha{(ver)brennen} 
\forms{
\PF dad\shortlonga{}ha, dehitha/dadagdha, dadāha; dehe
\AO adhākṣīt
\FT dhakṣyati/dahiṣyati; dhakṣyate
\PS dahyate
\GDV dāhya, dagdhavya
\INFIN dagdhum
\ABS dagdhvā, \mbox{-dahya}
\CS dāhayati \CS \AO adīdahat
\DS didhakṣati; -te
\IS dandahīti; dandahyate
}
\Zusatz{dāha m.\ \artha{Brennen, Hitze}}

\dhatu{dā} \pres{dadāti; datte \gan{III PĀ}}, \pppp{datta} \artha{geben} 
\forms{
\IND dadāmi, dadāsi, dadāti, dadvaḥ, datthaḥ, dattaḥ, dadmaḥ, dattha, dadati; dade, datse, datte, dadvahe, dadāthe, dadāte, dadmahe, daddhve, dadate
\PF dadau; dade
\AO adāt; adita, adiṣata
\FT dāsyati; -te
\PFT dātā
\PS dīyate
\GDV deya, dātavya
\INFIN dātum
\ABS dattvā, \mbox{-dāya}
\CS dāpayati \CS \PPF dāpayām āsa\CS \GDV dāpya, dāpayitavya, dāpanīya
\DS ditsati; -te
}
\Zusatz{dāna n.\ \artha{Gabe}; präfigiert mit ā-: \PPP ā-tta \ABS  ā-dāya \artha{mit}}

\dhatu{div [1]} \pres{dīvyati; -te \gan{IV  P}}, \pppp{dyūta} \artha{spielen, strahlen, wetten}

\forms{
\AO adevīt
\FT deviṣyati
\GDV devitavya
\INFIN devitum
\ABS dyūtvā, \mbox{-dīvya}
\CS devayati
}
\Zusatz{divya \artha{himmlisch}; deva m.\ \artha{Gott}}

\dhatu{div [2]/dev} \pres{(pari-)devate; -ti \gan{I Ā}}, \pppp{(pari-)dyūna} \artha{wehklagen} 
\forms{
\INFIN (pari-)devitum
\CS (pari-)devayate; -ti 
}

\dhatu{diś} \pres{diśati; -te \gan{VI PĀ}}, \pppp{diṣṭa} \artha{zeigen; anordnen} 
\forms{
\PF dideśa; didiśe
\AO adikṣat
\FT dekṣyati; -te
\PS diśyate
\GDV \mbox{-deśya}, \mbox{-deṣṭavya}
\INFIN deṣṭum
\ABS diṣṭvā, \mbox{-diśya}
\CS deśayati; -te
\DS didikṣati
}
\Zusatz{ud-deśa m.\ \artha{Nennung}; ud-diśya \artha{gegen, auf, nach, zu, in Bezug auf, in Betreff von, über}}


\dhatu{dih} \pres{degdhi; digdhe \gan{II PĀ}}, \pppp{digdha} \artha{bestreichen} 
\forms{
\IND dehmi, dhekṣi, degdhi, dihvaḥ, digdhaḥ, digdhaḥ, dihmaḥ, digdha, dihanti;
dihe, dhikṣe, digdhe, dihvahe, dihāthe, dihāte, dihmahe, dhigdhve, dihate
\OP dihyāt; dihīta
\IPV dehāni, digdhi, degdhu, dehāva, digdham, digdhām, dehāma, digdha, dihantu;
dehai, dhikṣva, digdhām, dehāvahai, dihāthām, dihātām, dehāmahai, dhigdhvam, dihatām
\IMP adeham, adhek, adhek, adihva, adigdham, adigdhām, adihma, adigdha, adihan;
adihi, adigdhāḥ, adigdha, adihvahi, adihāthām, adihātām, adihmahi, adhigdhvam, adihata
\PF dideha; didihe
\PS dihyate
\GDV \mbox{-dehya}
\ABS digdhvā, \mbox{-dihya}
\CS dehayati; -te
}
\Zusatz{saṃ-deha m.\ \artha{Zweifel}; saṃ-digdha \artha{zweifelhaft}}

\dhatu{dīkṣ} \pres{dīkṣate \gan{I Ā}}, \pppp{---} \artha{sich weihen} 
\forms{
\PF didīkṣuḥ
\FT dīkṣiṣyate
\PS dīkṣyate
\ABS dīkṣitvā
\CS dīkṣayati; -te \CS \PPP dīkṣita
}
\Zusatz{dīkṣā f.\ \artha{Einweihung}}

\dhatu{dīp} \pres{dīpyate; -ti \gan{IV Ā}}, \pppp{dīpta} \artha{strahlen, brennen} 
\forms{
\PF didīpe
\INFIN dīpitum
\CS dīpayati; -te \CS \AO adidīpat/adīdipat
\IS dedīpyate
}

\dhatu{dīv} \vw{div [1]}

\dhatu{du} \pres{dunoti \gan{V P}}, \pppp{dūna/duta} \artha{brennen; quälen} 
\forms{
\PS dūyate
\CS dāvayati
}

\dhatu{duḥkh} \pres{duḥkhayati \gan{X P}}, \pppp{duḥkhita} \artha{schmerzen} 
\Zusatz{duḥkha \artha{unangenehm}, n.\ \artha{Leid}}

\dhatu{dul} \pres{dolayati \gan{X PĀ}}, \pppp{dolita} \artha{schwingen} 
\Zusatz{dolā f.\ \artha{Schaukel}}

\dhatu{duṣ} \pres{duṣyati; -te \gan{IV P}}, \pppp{duṣṭa} \artha{schlecht werden/sein, sich vergehen} 
\forms{
\PS duṣyate
\GDV dūṣya
\CS dūṣayati/doṣayati; dūṣayate \artha{beschuldigen, für falsch erklären} \CS \PPP dūṣita \CS \PS dūṣyate
}
\Zusatz{duṣṭa \artha{schlecht}, m.\ \artha{schlechter Mensch}; doṣa m.\ \artha{Fehler}}

\dhatu{duh} \pres{dogdhi; dugdhe \gan{II PĀ}}, \pppp{dugdha} \artha{melken} 
\forms{
\IND dohmi, dhokṣi, dogdhi, duhvaḥ, dugdhaḥ, dugdhaḥ, duhmaḥ, dugdha, duhanti;
duhe, dhukṣe, dugdhe, duhvahe, duhāthe, duhāte, duhmahe, dhugdhve, duhate
\OP duhyāt; duhīta
\IPV dohāni, dugdhi, dogdhu, dohāva, dugdham, dugdhām, dohāma, dugdha, duhantu;
dohai, dhukṣva, dugdhām, dohāvahai, duhāthām, duhātām, dohāmahai, dhugdhvam, duhatām
\IMP adoham, adhok, adhok, aduhva, adugdham, adugdhām, aduhma, adugdha, aduhan;
aduhi, adugdhāḥ, adugdha, aduhvahi, aduhāthām, aduhātām, aduhmahi, adhugdhvam, aduhata
\PF dudoha; duduhe
\AO adhukṣat; adhukṣata
\FT dhokṣyate
\PS duhyate
\GDV duhya/dohya, dogdhavya
\INFIN dogdhum
\ABS dugdhvā
\CS dohayati; -te \CS \AO adūduhat
\DS dudhukṣati
}
\Zusatz{saṃ-doha m.\ \artha{Fülle}}

\dhatu{dūray} \pres{dūrayati} \gand{Denom.} \pppp{dūrita} \artha{fern sein; entfernen}
\Zusatz{dūra n.\ \artha{Ferne}; dūraṃ tyaj \artha{weithin meiden}}

\dhatu{dūṣay} \vw{duṣ}

\dhatu{dṛ [1]} \pres{(ā-)driyate \gan{VI Ā}}, \pppp{(ā-)dṛta} \artha{berücksichtigen, geachtet werden} 
\forms{
  \PS driyate
  \GDV (ā-)dṛtya, (ā-)dartavya, (ā-)daraṇīya
\ABS (ā-)dṛtya
}
\Zusatz{ā-dara m.\ \artha{Rücksicht}}

\dhatu{dṛ [2]/dṝ} \pres{dṛṇāti \gan{IX P}}, \pppp{dīrṇa} \artha{bersten} 
\forms{
\PF dadāra
\PS dīryate
\GDV \mbox{-daraṇīya}
\CS d\shortlonga{}rayati; dārayate \CS \AO adīdarat \artha{spalten}
}
\Zusatz{d\shortlonga{}ra m.\ \artha{Spalte, Höhle}} 

\dhatu{dṛp} \pres{dṛpyati \gan{IV P}}, \pppp{dṛpta} \artha{ausgelassen/stolz sein} 
\forms{
\CS darpayati \CS \PPP darpita
}
\Zusatz{darpa m.\ \artha{Stolz}}

\dhatu{dṛbh} \pres{dṛbhati \gan{VI P}}, \pppp{dṛbdha} \artha{verflechten} 
\forms{
\CS \PPP darbhita
}
\Zusatz{darbha m.\ \artha{(rituell verwendetes) Darbhagras}}

\dhatu{dṛś} \pres{[paśyati] \gan{I P}}, \pppp{dṛṣṭa} \artha{sehen} 
\forms{
\IMP [apaśyat]
\PF dadarśa, dadṛśuḥ; dadṛśe
\AO adrākṣīt/adarśat
\FT drakṣyati; -te
\PFT draṣṭā
\PS dṛśyate
\GDV dṛśya, draṣtavya, darśanīya
\INFIN draṣṭum
\ABS dṛṣṭvā, \mbox{-dṛśya}
\CS darśayati; -te \CS \AO adīdṛśat
\DS didṛkṣate
\IS darīdṛśyate
}
\Zusatz{unvollständiges Paradigma, präsentische Formen werden von \vw{paś} ergänzt; darśana n.\ \artha{Audienz; philosophisches System}}

\dhatu{dev} \vw{div [2]/dev}

\dhatu{dolāy} \pres{dolāyate} \gand{Denom.} \pppp{dolāyita} \artha{schaukeln}
\Zusatz{v.\ dolā f.\ \artha{Schaukel}}

\dhatu{dyut} \pres{dyotate \gan{I Ā}}, \pppp{dyutita} \artha{leuchten} 
\forms{
\PF didyute
\AO adyutat
\GDV dyotya
\ABS \mbox{-dyutya}
\CS dyotayati \CS \PS dyotyate \CS \PPP dyotita
}
\Zusatz{dyuti f.\ \artha{Glanz}}

\dhatu{draḍhay} \pres{draḍhayati} \gand{Denom.} \pppp{---} \artha{fest machen}
\Zusatz{v.\ dṛḍha \artha{fest}}

\dhatu{drā/dra [1]} \pres{drāti \gan{II P}}, \pppp{drāṇa} \artha{laufen, eilen} 
\forms{
\PF (vi-)dadruḥ
\CS drāpayati
\IS daridrāti \artha{arm sein} 
}

\dhatu{drā/dra [2]/drai} \pres{ni-drāti | ni-drāyate \gan{I P}}, \pppp{ni-drāṇa/ni-drita} \artha{schlafen} 
\forms{
\PF ni-dadrau
\FT ni-drāsyati
\INFIN ni-drātum
\DS ni-didrāsati
}
\Zusatz{yoga-ni-drā f.\ \artha{Yoga-Schlaf}}

\dhatu{dru} \pres{dravati; -te \gan{I P}}, \pppp{druta} \artha{laufen, eilen; zerfließen} 
\forms{
\PF dudrāva, dudrotha, dudruva, dudruvatuḥ, dudruma, dudruvuḥ; dudruve
\AO adudruvat
\PS drūyate
\INFIN drotum
\ABS drutvā, \mbox{-drutya}
\CS drāvayati; -te \CS \AO adudruvat \CS \PS drāvyate \CS \GDV drāvya
}
\Zusatz{dravya n.\ \artha{Substanz}}

\dhatu{druh} \pres{druhyati; -te \gan{IV P}}, \pppp{drugdha} \artha{schädigen} 
\forms{
\PF dudroha, dudrohitha, dudruhiva; dudruhe
\AO adruhat
\GDV druhya, drogdhavya
}

\dhatu{drai} \vw{drā/dra [2]/drai}

\dhatu{dviguṇay} \pres{dviguṇayati} \gand{Denom.} \pppp{dviguṇita} \artha{verdoppeln}
\Zusatz{v.\ dvi-guṇa \artha{zweifach}}

\dhatu{dviṣ} \pres{dveṣṭi \gan{II PĀ}}, \pppp{dviṣṭa} \artha{hassen} 
\forms{
\IND dveṣmi, dvekṣi, dveṣṭi, dviṣvaḥ, dviṣṭhaḥ, dviṣṭaḥ, dviṣmaḥ, dviṣṭha, dviṣanti
\IMP adveṣam, adveṭ, adveṭ, adviṣva, adviṣṭam, adviṣṭām, adviṣma, adviṣṭa, adviṣan
\PF didveṣa
\PS dviṣyate
\GDV dveṣya, dveṣaṇīya
\INFIN dveṣṭum
\CS dveṣayati
}

\medskip

\dhatu{dham/dhmā} \pres{dhamati; -te \gan{I P}}, \pppp{dhmāta} \artha{blasen} 
\forms{
\PF dadhmau; dadhmire
\AO adhmāsīt
\FT dhamiṣyati
\PS dhamyate/dhmāyate
\GDV dhmātavya
\ABS dhmātvā, \mbox{-dhmāya}
\CS dhmāpayati
\IS dādhmāyate
}

\dhatu{dhay} \vw{dhā [2]}

\dhatu{dhā [1]} \pres{dadhāti; dhatte \gan{III PĀ}}, \pppp{hita} \artha{setzen, stellen, legen} 
\forms{
\IND dadhāmi, dadhāsi, dadhāti, dadhvaḥ, dhatthaḥ, dhattaḥ, dadhmaḥ, dhattha, dadhati; dadhe, dhatse, dhatte, dadhvahe, dadhāthe, dadhāte, dadhmahe, dhaddhve, dadhate
\OP dadhyāt; dadhīta
\IPV dadhāni, dhehi, dadhātu, dhattām, dadhatu; dadhai,  dhatsva,  dhattām,  dadhātām,  dadhatām
\IMP adadhāt,  adhattām, adadhuḥ; adhatta,  adadhātām,  adadhata
\PF dadhau; dadhe
\AO adhāt; adhita
\INJ dhīmahi \artha{wir mögen gedenken}
\FT dhāsyati; -te
\PS dhīyate \PS \AO adhāyi
\GDV dheya, dhātavya, \mbox{-dhānīya}
\INFIN dhātum
\ABS hitvā, \mbox{-dhāya}
\CS dhāpayati \CS \PS dhāpyate \CS \GDV dhāpayitavya
\DS dhitsati; -te
}
\Zusatz{hita n.\ \artha{das Gute}}

\dhatu{dhā [2]/dhe} \pres{dhayati \gan{I P}}, \pppp{dhīta} \artha{saugen} 

\dhatu{dhāv} \pres{dhāvati; -te \gan{I PĀ}}, \pppp{dhāvita/dhauta} \artha{laufen, strömen; waschen} 
\forms{
\PF dadhāva, dadhāvire
\FT dhāviṣyati
\PS dhāvyate
\PPP dhāvita \artha{gelaufen} \PPP dhauta \artha{gewaschen}
\ABS dhāvitvā, dhāvya
\CS dhāvayati
}

\dhatu{dhī} \vw{dhā} \vw{dhyā}

\dhatu{dhu/dhū} \pres{dhunoti; dhunute \gan{V PĀ} | dhunāti; dhunīte \gan{IX PĀ}}, \pppp{dh\shortlongu{}ta} \artha{schütteln; entfernen} 
\forms{
\PF dudhāva
\FT dhaviṣyati
\PS dhūyate
\INFIN dhavitum
\ABS \mbox{-dhūya}
\CS dhūnayati 
\IS dodhavīti; dodhūyate
}

\dhatu{dhukṣ} \pres{dhukṣate \gan{I Ā}}, \pppp{dhukṣita} \artha{anzünden} 
\forms{
\CS dhukṣayati
}

\dhatu{dhūp} \pres{dhūp\shortlonga{}yati \gan{X P} | --- \gan{I P}}, \pppp{dhūpita} \artha{räuchern}
\forms{
}
\Zusatz{dhūpa m.\ \artha{Räucherwerk}}

\dhatu{dhṛ} \pres{[Präsens nicht gebräuchlich] \gan{I Ā}}, \pppp{dhṛta} \artha{halten, tragen}

\forms{
\PF dadhāra;  dadhre
\FT dhariṣyati; -te
\PFT dhartā
\PS dhriyate \PS \AO adhāri
\GDV dhārya, dhartavya, \mbox{-dharaṇīya}
\INFIN dhartum
\ABS dhṛtvā, \mbox{-dhṛtya}
\CS dhārayati; -te \CS \PPF dhārayām āsa \CS \AO adīdharat \CS \GDV dhārayitavya, dhāraṇīya
}

\dhatu{dhṛṣ} \pres{dhṛṣṇoti \gan{V P}}, \pppp{dhṛṣṭa} \artha{wagen, dreist sein} 
\forms{
\GDV dharṣaṇīya
\INFIN dharṣitum
\ABS \mbox{-dhṛṣya}
\CS dharṣayati
}

\dhatu{dhe} \vw{dhā [2]/dhe}

\dhatu{dhmā} \vw{dham/dhmā}

\dhatu{dhyā/dhyai} \pres{dhyāyati; -te \gan{I P} | dhyāti}, \pppp{dhyāta/dhīta}
\artha{überlegen, vorstellen, meditieren} 
\forms{
\PF dadhyau
\FT dhyāsyati
\PS dhyāyate
\GDV dhyeya, dhyātavya
\INFIN dhyātum
\ABS dhyātvā, \mbox{-dhyāya}
\DS didhyāsate
}
\Zusatz{dhīmahi \vw{dhā}; dhyāna n.\ \artha{Meditation}}

\dhatu{dhvaṃs/dhvas} \pres{dhvaṃsate; -ti \gan{I Ā}}, \pppp{dhvasta} \artha{herabfallen, zerfallen} 
\forms{
\PF dadhvaṃsire
\FT dhvaṃsiṣyate
\PS dhvasyate
\INFIN  dhvaṃsitum
\ABS \mbox{-dhvasya}
\CS dhvaṃsayati
}
\Zusatz{dhvasta \artha{zerfallen}, aber: rajasā dhvasta \artha{von Staub verhüllt}}

\dhatu{dhvan} \pres{dhvanati \gan{I P}}, \pppp{dhvanita/dhvānta} \artha{tönen} 
\forms{
\PF dadhvāna, dadhvanuḥ
\PS dhvanyate
\ABS dhvanitvā
\CS dhvanayati 
}
\Zusatz{dhvani m.\ \artha{Ton; Andeutung}}

\medskip

\dhatu{naṃs} \vw{nam}

\dhatu{naṭ} \pres{naṭati \gan{I P}}, \pppp{naṭita} \artha{tanzen, (Theater, eine Rolle) spielen} 
\forms{
\GDV naṭanīya
\CS nāṭayati  \CS \GDV nāṭayitavya
}
\Zusatz{nāṭya n.\ \artha{Schauspiel}}

\dhatu{nad} \pres{nadati; -te \gan{I P}}, \pppp{nadita} \artha{tönen, brüllen} 
\forms{
\PF nanāda,  neditha,  neduḥ; nedire
\PS nadyate
\ABS naditvā, \mbox{-nadya}
\CS n\shortlonga{}dayati; -te 
\IS nānadyate
}

\dhatu{nand} \pres{nandati; -te \gan{I P}}, \pppp{nandita} \artha{sich freuen} 
\forms{
\PF nananda, nanandatuḥ, nananduḥ
\FT nandiṣyati; -te
\GDV \mbox{-nandya}, nanditavya, nandanīya
\INFIN nanditum
\ABS nanditvā, \mbox{-nandya}
\CS nandayati; -te \CS \PPF nandayām āsa\CS \GDV \mbox{-nandayitavya}
}
\Zusatz{nandana \artha{erfreuend}, m.\ \artha{Sohn}, f.\ \artha{Tochter}, n.\ \artha{Freude}}

\dhatu{nam} \pres{namati; -te \gan{I P}}, \pppp{nata} \artha{(ver)beugen} 
\forms{
\PF nanāma, nemuḥ
\AO anaṃsīt
\FT namiṣyati
\PS namyate
\GDV nāmya, namanīya
\INFIN namitum/nantum,
\ABS natvā, \mbox{-namya}
\CS n\shortlonga{}mayati \CS \AO anīnamat
\DS ninaṃsati
}

\dhatu{namasy} \pres{namasyati; -te} \gand{Denom.} \pppp{---} \artha{verehren, anbeten}
\Zusatz{v.\ namas n.\ \artha{Verbeugung, Verehrung}}

\dhatu{nard} \pres{nardati; -te \gan{I P}}, \pppp{nardita} \artha{brüllen} 
\forms{
\PF nanarda
\IS \PPA nānardyamāna
}

\dhatu{naś} \pres{naśyati; -te \gan{IV P}}, \pppp{naṣṭa} \artha{vergehen, verschwinden} 
\forms{
\PF nanāśa, neśuḥ
\AO anaśat
\FT naśiṣyati/naṅkṣyati; -te
\GDV nāśya
\INFIN naṣṭum/naśitum
\CS nāśayati; -te \CS \AO anīnaśat
}
\Zusatz{nāśa m.\ \artha{Untergang}}

\dhatu{nah} \pres{nahyati; -te \gan{IV PĀ}}, \pppp{naddha} \artha{binden, (Rüstung) anlegen} 
\forms{
\PF nanāha; nehe
\PS nahyate
\GDV \mbox{-naddhavya}
\ABS \mbox{-nahya}
\CS nāhayati
}

\dhatu{nāth} \pres{nāthate; -ti \gan{I Ā}}, \pppp{nāthita} \artha{bitten, flehen} 
\forms{
\PS nāthyate
\INFIN nāthitum
\ABS \mbox{-nāthya}
}
\Zusatz{nātha m.\ \artha{Gebieter}}

\dhatu{nigaḍay} \pres{nigaḍayati} \gand{Denom.} \pppp{nigaḍita} \artha{fesseln}
\Zusatz{v.\ nigaḍa m.\ \artha{eiserne Fußkette für Elefanten}}

\dhatu{nij} \pres{nenekti; nenikte \gan{III PĀ}}, \pppp{nikta} \artha{sich waschen} 
\forms{
\PS nijyate
\ABS niktvā, \mbox{-nijya}
}

\dhatu{nidrā(y)} \vw{drā [2]}

\dhatu{nind} \pres{nindati; -te \gan{I P}}, \pppp{nindita} \artha{tadeln, schmähen} 
\forms{
\PF nininda
\FT nindiṣyati
\PFT ninditā
\PS nindyate
\GDV nindya, nindanīya
\INFIN ninditum
\ABS ninditvā, \mbox{-nindya}
\CS nindayati
}
\Zusatz{nindā f.\ \artha{Tadel; Schmähung}}

\dhatu{nibiḍay} \pres{nibiḍayati} \gand{Denom.} \pppp{nibiḍita} \artha{fest umschlingen, dicht werden}
\Zusatz{nibiḍa \ \artha{dicht, fest}}

\dhatu{niśam/niśām} \vw{śam [2]}

\dhatu{nī} \pres{nayati; -te \gan{I PĀ}}, \pppp{nīta} \artha{führen} 
\forms{
\PF nināya, ninyuḥ; ninye
\AO anaiṣīt \AO \PS anāyi, anāyiṣata
\FT neṣyati/nayiṣyati; -te
\PFT netā
\PS nīyate
\GDV neya, netavya/nayitavya
\INFIN netum/nayitum
\ABS nītvā/nayitvā \Zusatz{oft nur: \artha{mit}}, \mbox{-nīya}
\CS nāyayati; -te
\DS ninīṣati; -te
\IS nenīyate
}
\Zusatz{nayana n.\ \artha{Auge}}

\dhatu{nu/nū} \pres{nauti \gan{II P}}, \pppp{nuta} \artha{preisen, jubeln, schreien} 
\forms{
\IND nauti, nuvanti
\PF nunāva
\PS nūyate
\ABS nutvā
}

\dhatu{nud} \pres{nudati; -te \gan{VI PĀ}}, \pppp{nutta/nunna} \artha{stoßen; entfernen} 
\forms{
\PF nunoda; nunude
\FT notsyati; -te
\GDV nodya
\ABS \mbox{-nudya}
\CS nodayati \CS \PS nodyate \CS \GDV \mbox{-nodayitavya}
}
\Zusatz{tamo-nud m.\ \artha{Sonne, Mond}}

\dhatu{nṛt} \pres{nṛtyati; -te \gan{IV P}}, \pppp{nṛtta} \artha{tanzen, (Theater) spielen}

\forms{
\PF nanarta,  nanṛtuḥ
\FT nartiṣyati
\PS nṛtyate
\GDV nṛtya, nartitavya
\INFIN nart(i)tum
\ABS nartitvā
\CS nartayati
\DS ninartiṣati
\IS nar\shortlongi{}nartti; nar\shortlongi{}nṛtyate
}
\Zusatz{nṛtya n.\ \artha{Tanz}; nartaka m.\ \artha{Tänzer}}

\medskip

\dhatu{pac} \pres{pacati; -te \gan{I PĀ}}, \pppp{pakva} \artha{kochen; verdauen; reifen (lassen)} 
\forms{
\PF pece, pecire
\FT pakṣyati; -te
\PFT paktā
\PS pacyate
\GDV pācya/pākya, paktavya
\ABS paktvā
\CS pācayati; -te \CS \PS pācyate
\IS pāpacyate
}
\Zusatz{pakva meist nur adj.\ \artha{reif}}

\dhatu{paṭ} \pres{pāṭayati; -te \gan{X PĀ} | --- \gan{I P}}, \pppp{pāṭita} \artha{sich spalten, zerreißen} 
\forms{
\GDV pāṭya, pāṭanīya
}

\dhatu{paṭh} \pres{paṭhati; -te \gan{I P}}, \pppp{paṭhita} \artha{lesen, studieren} 
\forms{
\PF papāṭha
\FT paṭhiṣyati
\PS paṭhyate
\GDV pāṭhya, paṭhitavya, paṭhanīya
\INFIN paṭhitum
\ABS paṭhitvā
\CS pāṭhayati \CS \PS pāṭhyate
\DS pipaṭhiṣati
\IS pāpaṭhīti; pāpaṭhyate
}
\Zusatz{pāṭha m.\ \artha{Vortrag; Lesart}}

\dhatu{paṇ} \pres{paṇate; -ti \gan{I Ā}}, \pppp{paṇita} \artha{handeln; wetten, einsetzen} 
\forms{
\PS paṇyate
\GDV paṇya
\CS p\shortlonga{}ṇayati
}
\Zusatz{paṇa m.\ \artha{Wetteinsatz}}

\dhatu{pat} \pres{patati; -te \gan{I P}}, \pppp{patita} \artha{fliegen; fallen; eintreffen} 
\forms{
\PF papāta, petuḥ
\AO apaptat
\FT patiṣyati; -te
\GDV pātya, patitavya, pātanīya
\INFIN patitum
\ABS patitvā, \mbox{-patya}
\CS pātayati; -te  \CS \PPF pātayām āsa 
\DS pitsati
}

\dhatu{pad} \pres{padyate; -ti \gan{IV Ā}}, \pppp{panna} \artha{fallen; hingehen} 
\forms{
\PF pede; papāda, peduḥ
\FT patsyati; -te
\PS \AO apādi
\GDV \mbox{-pādya}, \mbox{-pattavya}, \mbox{-padanīya}
\INFIN pattum
\ABS \mbox{-padya}
\CS pādayati; -te , pādayitavya
\DS pitsate
\IS panīpadyate
}\Zusatz{häufig präfigiert, z.\,B.\ ā-panna \artha{hineingeraten}; vyā-panna \artha{vernichtet};
ut-panna \artha{entstanden}; upa-panna \artha{versehen mit}}

\dhatu{palāy} \pres{palāyati; -te} \gand{erklärt als palā-i} \pppp{palāyita} \artha{fliehen}
\forms{
\AO apalāyiṣṭa
\FT palāyiṣyate
\PS palāyyate
\GDV palāyitavya
\INFIN palāyitum
}

\dhatu{pavitray} \pres{pavitrayati} \gand{Denom.} \pppp{pavitrita} \artha{läutern, reinigen}
\forms{
}
\Zusatz{v.\ pavitra n.\ \artha{Läuterungsmittel}}

\dhatu{paś} \pres{paśyati}, \pppp{[dṛṣṭa]} \artha{sehen}
\forms{
\IMP apaśyat
\PF [dadarśa, dadṛśuḥ; dadṛśe]
\AO [adrākṣīt/adarśat]
\FT [drakṣyati; -te]
\PFT [draṣṭā]
\PS [dṛśyate]
\GDV [dṛśya, draṣtavya, darśanīya]
\INFIN [draṣṭum]
\ABS [dṛṣṭvā, \mbox{-dṛśya}]
\CS [darśayati; -te]
\CS \AO [adīdṛśat]
\DS [didṛkṣate]
\IS [darīdṛśyate]
}\Zusatz{unvollständiges Paradigma, nicht-präsentische Formen werden von \vw{dṛś} ergänzt; vi-paśyanā f.\ \artha{buddh.\ Meditationsform}}

\dhatu{pā [1]} \pres{pibati; -te \gan{I P}}, \pppp{pīta} \artha{trinken; (bildl.\ mit Augen/Ohren) einsaugen} 
\forms{
\PF papau, papitha/papātha, papau, papuḥ
\AO apāt
\FT pāsyati; -te
\PS pīyate \PS \AO apāyi
\ABS pītvā, \mbox{-pīya}
\GDV peya, pātavya, pānīya
\INFIN pātum
\CS pāyayati; -te
\DS pipāsati
\IS pepīyate
}
\Zusatz{pāna n.\ \artha{Trank; Trinken}; pātra n.\ \artha{Trinkgefäß; Empfänger (von Gaben)}}

\dhatu{pā [2]} \pres{pāti \gan{II P}}, \pppp{---} \artha{schützen, bewachen, befolgen} 
\forms{
\IPV pātu
\AO apāsīt
\GDV pātavya
\INFIN pātum
}
\Zusatz{pati m.\ \artha{Gemahl}; patnī f.\ \artha{Gattin}}

\dhatu{pāl} \pres{pālayati; -te \gan{X PĀ}}, \pppp{pālita} \artha{hüten} 
\forms{
\AO apīpalat
\PS pālyate
\GDV pālya, \mbox{-pālitavya}, pālanīya
\ABS pālayitvā
}
\Zusatz{go-pāla m.\ \artha{Kuhhirte}}

\dhatu{piṇḍ} \pres{piṇḍayati \gan{X P}}, \pppp{piṇḍita} \artha{zusammenballen, vereinigen} 
\Zusatz{piṇḍa m.\ \artha{Kloß (als Gabe im Ahnenopfer)}}

\dhatu{pib} \vw{pā [1]}

\dhatu{piṣ} \pres{pinaṣṭi \gan{VII P}}, \pppp{piṣṭa} \artha{zermalmen} 
\forms{
\IND pinaṣmi, pinakṣi, pinaṣṭi, piṃṣmaḥ, piṃṣṭha, piṃṣanti
\IMP apinaṣam, apinaṭ, apinaṭ, apiṃṣva
\PF pipeṣa; pipiṣe
\PS piṣyate
\INFIN peṣṭum
\ABS piṣṭvā
\CS peṣayati \CS \AO apīpiṣat
}

\dhatu{pī} \vw{pyā(y)}

\dhatu{pīḍ} \pres{pīḍayati; -te \gan{X PĀ}}, \pppp{pīḍita} \artha{pressen, plagen}
\forms{
\PPF pīḍayām āsa/cakāra
\FT pīḍayiṣyati
\PS pīḍyate
\GDV pīḍayitavya, pīḍanīya
\INFIN pīḍayitum
\ABS pīḍayitvā, \mbox{-pīḍya}
}
\Zusatz{pīḍā f.\ \artha{Qual}}

\dhatu{puṭ} \pres{puṭati \gan{VI P}}, \pppp{puṭita} \artha{umhüllen} 
\forms{
\PS puṭyate
}

\dhatu{putriy/putrīy} \pres{putr\shortlongi{}yati} \gand{Denom.} \pppp{---} \artha{sich einen Sohn wünschen}
\Zusatz{v.\ putra m.\ \artha{Sohn}}

\dhatu{puth} \pres{pothayati; -te \gan{X PĀ} | --- \gan{IV P}}, \pppp{pothita} \artha{zerdrücken} 
\forms{
\PPF pothayām āsa/cakāra
\FT pothayiṣyati
\ABS pothayitvā
}

\dhatu{puṣ} \pres{puṣyati; -te \gan{IV P} | puṣṇāti \gan{IX P}}, \pppp{puṣṭa} \artha{gedeihen, aufziehen} 
\forms{
\PF pupoṣa
\PS puṣyate
\GDV poṣya, poṣaṇīya
\CS poṣayati 
}\Zusatz{°puṣṭa \artha{reich an \dots}, aber anya-puṣṭa \artha{von anderen aufgezogen} = \artha{Kuckuck}}

\dhatu{pū} \pres{punāti; punīte \gan{IX PĀ}}, \pppp{pūta} \artha{reinigen} 
\forms{
\PF pupuve
\PS pūyate
\CS pāvayati; -te \CS \PS pāvyate
}
\Zusatz{pavana m.\ \artha{Wind}; pāvana \artha{reinigend}, m.\ \artha{Feuer}}

\dhatu{pūj} \pres{pūjayati; -te \gan{X PĀ}}, \pppp{pūjita} \artha{(ver)ehren} 
\forms{
\PPF pūjayām āsa/cakāra
\AO apūpujat
\FT pūjayiṣyati
\PS pūjyate
\GDV pūjya, pūjayitavya, pūjanīya
\INFIN pūjayitum
\ABS pūjayitvā
}
\Zusatz{pūjā f.\ \artha{Verehrung, Gottesdienst}}

\dhatu{pūr} \vw{pṝ}

\dhatu{pṛ} \pres{(vyā-)pṛṇoti \gan{V P} | --- \gan{VI Ā}}, \pppp{(vyā-)pṛta} \artha{sich beschäftigen} 
\forms{
\PS \mbox{(vyā-)priyate}
\INFIN (vyā-)partum
\CS (vyā-)pārayati
}
\Zusatz{vyā-pāra m.\ \artha{Beschäftigung}}

\dhatu{pṛc} \pres{pṛṇakti \gan{VII P}}, \pppp{pṛkta} \artha{mischen}

\dhatu{pṛcch} \vw{prach}

\dhatu{pṝ} \pres{piparti \gan{III P}}, \pppp{pūrṇa/pūrta} \artha{(er)füllen, sättigen} 
\forms{
\IND piparti, piprati
\PF papāra; pupūre, pupūrire
\PS pūryate  \PS \AO apūri
\GDV pūrya, pūraṇīya
\INFIN pūritum
\ABS \mbox{-pūrya}
\CS pūrayati; -te  \CS \GDV pūrayitavya
}
\Zusatz{pūrṇa meist adj.\ \artha{voll}}

\dhatu{pyā(y)} \pres{-pyāyate \gan{I Ā}}, \pppp{-pyāna/pīna} \artha{schwellen, überfließen} 
\forms{
\GDV \mbox{-pyāyya}
\ABS \mbox{-pyāyya}
\CS \mbox{-pyāyayati}; -te
}
\Zusatz{pīna meist adj.\ \artha{fett}}

\dhatu{prakaṭay} \pres{prakaṭayati} \gand{Denom.} \pppp{prakaṭita} \artha{offenbaren, zeigen}
\Zusatz{v.\ prakaṭa \artha{deutlich}}

\dhatu{prach} \pres{pṛcchati; -te \gan{VI P}}, \pppp{pṛṣṭa} \artha{(er)fragen, begehren} 
\forms{
\PF papraccha, papracchuḥ
\AO aprākṣīt; apraṣṭa
\FT prakṣyati
\PS pṛcchyate
\GDV praṣṭavya
\INFIN praṣṭum
\ABS pṛṣṭvā, \mbox{-pṛcchya}
\DS pipṛcchiṣati
}
\Zusatz{praśna m.\ \artha{Frage}}

\dhatu{prath} \pres{prathate; -ti \gan{I Ā}}, \pppp{prathita} \artha{ausbreiten; berühmt werden} 
\forms{
\PF paprathe; paprathatuḥ
\INFIN prathitum
\CS prathayati; -te 
}
\Zusatz{prathā f.\ \artha{Entfaltung; Berühmtheit}}

\dhatu{praśnay} \pres{praśnayati} \gand{Denom.} \pppp{praśnita} \artha{fragen nach}
\Zusatz{v.\ praśna m.\ \artha{Frage}}

\dhatu{prī} \pres{prīṇāti; prīṇīte \gan{IX PĀ}}, \pppp{prīta} \artha{(sich) erfreuen (an)} 
\forms{
\AO apraiṣīt
\PS prīyate
\CS prīṇayati 
}
\Zusatz{prīti f.\ \artha{Freude}}

\dhatu{preṣ} \vw{iṣ [1]}

\dhatu{plu} \pres{plavate; -ti \gan{I Ā}}, \pppp{pluta} \artha{schwimmen; schweben; verschwinden; springen} 
\forms{
\PF pupluve; pupluvuḥ
\FT ploṣyati; -te
\PS plūyate
\INFIN plotum
\ABS plutvā, \mbox{-plutya}
\CS plāvayati; -te \CS \PPP plāvita \CS \GDV plāvya
\IS poplūyate
}\Zusatz{°pluta \artha{erfüllt von \dots}}

\dhatu{pluṣ} \pres{ploṣati \gan{I P}}, \pppp{pluṣṭa} \artha{brennen} 
\forms{
\PS pluṣyate
}

\medskip

\dhatu{phal [1]} \pres{phalati \gan{I P}}, \pppp{phalita/phulla} \artha{bersten} 
\forms{
\PF paphāla, pheluḥ
\FT phaliṣyati
\CS phālayati
}

\dhatu{phal [2]} \pres{phalati; -te \gan{I P}}, \pppp{phalita} \artha{Früchte bringen, reifen} 
\forms{
\PF phelire
\FT phaliṣyati
\GDV phalitavya
}
\Zusatz{phala n.\ \artha{Frucht; (karmisches) Resultat}}

\medskip

\dhatu{bandh} \pres{badhnāti \gan{IX P}}, \pppp{baddha} \artha{binden, versperren, hervorrufen} 
\forms{
\PF babandhitha/babanddha, babandha, babandhuḥ
\FT bhantsyati/bandhiṣyati; -te
\PS badhyate
\GDV bandhya, baddhavya, bandhanīya
\INFIN ba(n)ddhum/bandhitum
\ABS baddhvā, \mbox{-badhya}
\CS bandhayati
}
\Zusatz{bandhu m.\ \artha{Verwandter, Freund}}

\dhatu{bādh} \pres{bādhate; -ti \gan{I Ā}}, \pppp{bādhita} \artha{bedrängen, beseitigen, plagen} 
\forms{
\PF babādhe
\FT bādhiṣyati; -te
\PS bādhyate
\GDV bādhya, bādhitavya, bādhanīya
\INFIN bādhitum
\ABS bādhitvā
\CS bādhayati
\DS bībhatsate
}

\dhatu{budh} \pres{bodhati; -te \gan{I PĀ} | budhyate; -ti \gan{IV Ā}}, \pppp{buddha} \artha{(er)wachen; wahrnehmen, erkennen} 
\forms{
\PF bubodha; bubudhe
\AO abuddhāḥ, abuddha / abodhiṣam, abodhiṣīḥ, abodhīt; abodhiṣata
\FT bhotsyati; \mbox{-te}
\PS budhyate
\PS \AO abodhi
\ABS buddhvā, \mbox{-budhya}
\GDV bodhya, boddhavya, bodhanīya
\INFIN boddhum
\CS bodhayati; -te \CS \GDV bodhayitavya
\DS bubhutsati; -te
\IS bobudhīti
}
\Zusatz{saṃ-bodhi f.\ \artha{Erleuchtung (buddh.)}}

\dhatu{bṛṃh} \pres{bṛṃhati \gan{I P}}, \pppp{bṛṃhita} \artha{wachsen; schreien} 
\forms{
\PF babṛṃhe, babṛṃhire
\GDV bṛṃhaṇīya
\CS bṛṃhayati; -te  \CS \GDV bṛṃhayitavya
}\Zusatz{brahman n.\ \artha{das Absolute}, m.\ \artha{Schöpfergott}; sva-śakti-bṛṃhita \artha{durch seine eigene Kraft gestärkt}}

\dhatu{brū} \pres{bravīti; brūte \gan{II PĀ}}, \pppp{---} \artha{sagen, sprechen (zu)} 
\forms{
\IND bravīmi/brūmi,  bravīṣi,  bravīti,  brūvaḥ, brūthaḥ, brūtaḥ, brūmaḥ, brūtha, bruvanti; bruve, brūṣe, brūte, bruvate
\OP brūyāt
\IPV bravāṇi, brūhi/bravīhi, bravītu,  bravāva,  brūtam,  brūtām,  bravāma,  brūta,  bruvantu
\IMP abravam/abruvam,  abravīḥ,  abravīt,  abrūva, abrūtam, abrūtām, abrūma, abrūta, abruvan
\PPA bruvant; bruvāṇa/bruvamāṇa
}

\medskip

\dhatu{bhakṣ} \pres{bhakṣati; -te | bhakṣayati; -te \gan{X PĀ}}, \pppp{bhakṣita} \artha{essen} 
\forms{
\PF babhakṣa
\PPF bhakṣayām āsa/cakāra
\FT bhakṣayiṣyati
\PS bhakṣyate \PS \AO abhakṣi
\GDV bhakṣya, bhakṣ(ay)itavya, bhakṣaṇīya
\INFIN bhakṣitum
\ABS bhakṣayitvā
\DS bibhakṣayiṣati
}

\dhatu{bhaj} \pres{bhajati; -te \gan{I PĀ}}, \pppp{bhakta} \artha{(ver)teilen; empfangen; sich hingeben, verehren} 
\forms{
\PF babh\shortlonga{}ja, babhaktha, babhāja, bhejuḥ; bheje, bhejire
\AO abhākṣīt; abhakta
\FT bhajiṣyati; -te
\PS bhajyate
\GDV bhajya/bhājya, bhajitavya/bhaktavya, bhajanīya
\INFIN bhajitum/bhaktum
\ABS bhaktvā, \mbox{-bhajya}
\CS bhājayati; -te 
\DS bhikṣati; -te
}
\Zusatz{bhakti f.\ \artha{Gottesliebe}; vi-bhakti f.\ \artha{Fall (gramm.)}}

\dhatu{bhañj} \pres{bhanakti \gan{VII  P}}, \pppp{bhagna} \artha{brechen, biegen} 
\forms{
\OP bhañjyāt
\IPV bhanaktu
\IMP abhanak
\PF babhañja; babhañjire
\AO abhāṅkṣīt
\FT bhaṅkṣyati
\PFT bhaṅktā
\PS bhajyate \PS \AO abhāji
\ABS bha(ṅ)ktvā, \mbox{-bhajya}
}

\dhatu{bhaṇ} \pres{bhaṇati \gan{I P}}, \pppp{bhaṇita} \artha{reden} 
\forms{
\PF babhāṇa
\PS bhaṇyate \PS \AO abhāṇi
\GDV bhaṇanīya
\ABS bhaṇitvā
\DS bibhaṇiṣati
}
\Zusatz{bhāṇa m.\ \artha{Monologstück}}

\dhatu{bharts} \pres{bhartsayati; -te \gan{X Ā}}, \pppp{bhartsita} \artha{drohen, schelten} 
\forms{
\PPF bhartsayām āsa/cakāra
\PS bhartsyate
\PPA bhartsayant
\ABS bhartsayitvā
}

\dhatu{bhal} \pres{(ni-)bhālayati; -te \gan{X Ā}}, \pppp{(ni-)bhālita} \artha{sehen} 
\forms{
\INFIN \mbox{(ni-)bhālayitum}
}

\dhatu{bhaṣ} \pres{bhaṣati; -te \gan{I P}}, \pppp{bhaṣita} \artha{bellen} 
\forms{
\INFIN bhaṣitum
}

\dhatu{bhas} \pres{babhasti \gan{III P}}, \pppp{bhasita} \artha{verzehren} 

\dhatu{bhā} \pres{bhāti \gan{II P}}, \pppp{bhāta} \artha{(er)scheinen, glänzen} 
\forms{
\IND bhāti, bhānti
\IPV bhātu
\IMP abhāt,  abhān/abhuḥ
\PF babhau
\FT bhāsyati
\INFIN bhātum
\CS bhāpayate
}
\Zusatz{bhāna n.\ \artha{Schein}}

\dhatu{bhāṣ} \pres{bhāṣate; -ti \gan{I Ā}}, \pppp{bhāṣita} \artha{reden} 
\forms{
\PF babhāṣe
\AO abhāṣiṣi, abhāṣiṣṭa
\FT bhāṣiṣyate
\PS bhāṣyate
\GDV bhāṣya, bhāṣitavya, \mbox{-bhāṣaṇīya}
\INFIN bhāṣ(i)tum
\ABS bhāṣitvā, \mbox{-bhāṣya}
\CS bhāṣayati; -te
}
\Zusatz{bhāṣā f.\ \artha{Sprache}, besonders \artha{Umgangssprache}}

\dhatu{bhās} \pres{bhāsate; -ti \gan{I Ā}}, \pppp{bhāsita} \artha{glänzen, (ein)leuchten} 
\forms{
\PF babhāse
\GDV bhāsya
\ABS bhāsitvā
\CS bhāsayati; -te
}
\Zusatz{bhās f., (ā-)bhāsa m.\ \artha{Licht}}

\dhatu{bhikṣ} \pres{bhikṣate; -ti \gan{I Ā}}, \pppp{bhikṣita} \artha{erstreben, (er)betteln} 
\forms{
\PF bibhikṣe
\FT bhikṣiṣyate
\INFIN bhikṣitum
\ABS bhikṣitvā
\CS bhikṣayati
}
\Zusatz{bhikṣu m.\ \artha{Mönch}}\\
\vw{bhaj}

\dhatu{bhid} \pres{bhinatti; bhintte \gan{VII PĀ}}, \pppp{bhinna} \artha{spalten, brechen} 
\forms{
\IND bhinatti, bhindanti; bhintte
\PF bibheda, bibhiduḥ; bibhide
\FT bhetsyati; -te
\PS bhidyate
\GDV bhedya, bhettavya, bhedanīya
\INFIN bhettum
\ABS bhittvā, \mbox{-bhidya}
\CS bhedayati; -te
}
\Zusatz{a-bheda m.\ \artha{Nicht-Dualität}}

\dhatu{bhī} \pres{bibheti \gan{III P} | bibhyati}, \pppp{bhīta} \artha{sich fürchten} 
\forms{
\IND bibheti, bibhyati; bibhyati, bibhyanti
\IMP abibhet,  abibhayuḥ
\PF bibhāya, bibhyuḥ; bibhye
\AO abhaiṣīt
\PS bhīyate
\GDV \mbox{-bheya}, bhetavya
\INFIN bhetum
\ABS \mbox{-bhīya}
\CS bhāyayati/bhīṣayati; -te
\IS bebhīyate
}
\Zusatz{bhaya n.\ \artha{Furcht; Gefahr}}

\dhatu{bhuj [1]} \pres{bhujati \gan{VI P}}, \pppp{bhugna} \artha{biegen} 
\forms{
\PF bubhoja
\ABS (ā-)bhujya
}\Zusatz{bhoga m.\ \artha{Haube (einer Kobra)}; vāyu-bhugna \artha{vom Wind gebeugt}}

\dhatu{bhuj [2]} \pres{bhunakti; bhuṅkte \gan{VII P}}, \pppp{bhukta} \artha{genießen; erfahren}

\forms{
\IND bhunakti, bhuñjanti; bhuṅkte, bhuñjate
\PF bubhujuḥ; bubhuje
\FT bhokṣyati; -te
\PS bhujyate
\GDV bhojya/bhogya, bhoktavya, bhojanīya
\INFIN bhoktum
\ABS bhu(ṅ)ktvā
\CS bhojayati/bhuñjāpayati; bhojayate \CS \GDV bhojayitavya
\DS bubhukṣati; -te
\IS bobhujīti; bobhujyate
}
\Zusatz{bhoga m.\ \artha{Speise; Genuss}}

\dhatu{bhū} \pres{bhavati; -te \gan{I P}}, \pppp{bhūta} \artha{werden, sein} 
\forms{
\OP bhavet
\AO abhūt
\PF babhūva, babhūvuḥ
\FT bhaviṣyati
\PFT bhavitā
\PS bhūyate \PS \AO abhāvi
\GDV bhavya/bhāvya, bhavitavya, bhavanīya
\INFIN bhavitum
\ABS bhūtvā, \mbox{-bhūya}
\CS bhāvayati; -te 
\CS \GDV bhāvya, bhāvayitavya, bhāvanīya
\DS bubhūṣati; -te
\IS bobhavīti
}
\Zusatz{einige Ableitungen auf \vw{S.\,\pageref{mindmap-bhu}}}

\dhatu{bhūṣ} \pres{bhūṣayati; -te \gan{X PĀ}}, \pppp{bhūṣita} \artha{schmücken} 
\forms{
\PFT bhūṣayitā
\PS \AO abhūṣi
\GDV bhūṣya, bhūṣayitavya, bhūṣaṇīya
}
\Zusatz{bhūṣaṇa n.\ \artha{Schmuck}}

\dhatu{bhṛ} \pres{bibharti \gan{III PĀ} | bharati; -te \gan{I PĀ}}, \pppp{bhṛta} \artha{(er)halten; davon-, herbeitragen; verschaffen} 
\forms{
\IND bibharti, bibhrati | bharati; -te
\IPV bibharāṇi,  bibhṛhi,  bibhartu
\PF babhāra,  babhartha,  babhṛva
\PPF bibharāṃ babhūva
\AO abhṛta
\FT bhariṣyati
\PS bhriyate
\GDV bhṛtya/\mbox{(saṃ-)bhārya}, bhartavya, bharaṇīya
\INFIN bhartum
\ABS bhṛtvā, \mbox{-bhṛtya}
\CS bhārayati
\DS bubhūrṣati
\IS barībharti
}
\Zusatz{bhāra m.\ \artha{Last}}

\dhatu{bhṛjj} \vw{bhrajj}

\dhatu{bhraṃś} \pres{bhraśyati; -te \gan{IV P} | bhraṃśate \gan{I Ā}}, \pppp{bhraṣṭa} \artha{stürzen, verloren gehen} 
\forms{
\CS bhraṃśayati
}
\Zusatz{bhraṣṭa-adhikāra m.\ \artha{einer, der sein Amt verloren hat}}

\dhatu{bhrajj} \pres{bhṛjjati \gan{VI PĀ}}, \pppp{bhṛṣṭa} \artha{rösten} 
\forms{
\PS bhṛjjyate
\ABS bhṛṣṭvā
\CS bharjjayati
}

\dhatu{bhram} \pres{bhrāmyati \gan{IV P} | bhramati; -te \gan{I P}}, \pppp{bhrānta} \artha{schweifen; sich irren} 
\forms{
\PF babhrāma,  babhramuḥ/bhremuḥ
\FT bhramiṣyati
\PS bhramyate \PS \AO abhrāmi
\GDV bhramaṇīya
\INFIN bhrāntum/bhramitum
\ABS bhrāntvā/bhramitvā, \mbox{-bhramya}/\mbox{-bhrāmya}
\CS bhr\shortlonga{}mayati \CS \AO abibhramat
\IS bambhramīti; bambhramyate
}
\Zusatz{(saṃ-)bhrama m.\ \artha{Irrtum, Verwirrung}}

\dhatu{bhraś} \vw{bhraṃś}

\dhatu{bhrāj} \pres{bhrājate; -ti \gan{I Ā}}, \pppp{bhrājita} \artha{glänzen, schimmern} 
\forms{
\PF babhrāje/bhreje; babhrāja
\FT bhrājiṣyate
\CS bhrājayati
}

\medskip

\dhatu{majj} \pres{majjati; -te \gan{VI P}}, \pppp{magna} \artha{sinken, untertauchen} 
\forms{
\PF mamajja
\AO amāṅkṣīt
\FT maṅkṣyati/majjiṣyati; maṅkṣyate
\PS \AO amajji
\GDV maṅktavya
\INFIN majjitum
\ABS \mbox{-majjya}
\CS majjayati; -te
\DS mimaṅkṣati
}
\Zusatz{jala-ni-magna \artha{im Wasser versunken}}

\dhatu{maṇḍ} \pres{maṇḍayati; -te \gan{X PĀ}}, \pppp{maṇḍita} \artha{schmücken} 
\forms{
\PPF maṇḍayām āsa/cakāra
}

\dhatu{math/manth} \pres{ma(n)thati \gan{I P} |  mathnāti \gan{IX P}}, \pppp{mathita} \artha{(um)rühren; quirlen; beschädigen} 
\forms{
\PF mama(n)tha, mama(n)thitha, mama(n)thuḥ
\FT ma(n)thiṣyati; -te
\PS mathyate
\GDV mathya,
\INFIN ma(n)thitum
\ABS ma(n)thitvā, \mbox{-mathya}
\CS ma(n)thayati/māthayati
}
\Zusatz{samudra-manthana n.\ \artha{Quirlung des Ozeans}}

\dhatu{mad} \pres{mādyati; -te \gan{IV P}}, \pppp{matta/madita} \artha{sich freuen (an); berauschen} 
\forms{
\AO amādīt, amādiṣuḥ
\CS m\shortlonga{}dayati; -te 
}
\Zusatz{pra-māda m.\ \artha{Nachlässigkeit}}

\dhatu{man} \pres{manyate \gan{IV Ā} | manute \gan{VIII Ā}}, \pppp{mata} \artha{denken, meinen} 
\forms{
\PF mene
\AO amaṃsta
\FT maṃsyate; -ti
\PS manyate
\GDV mantavya
\INFIN mantum
\ABS matvā, \mbox{-manya}/\mbox{-matya}
\CS mānayate; -ti \artha{ehren, beachten} 
\CS \GDV mānya, mānayitavya, mānanīya
\DS mīmāṃsate
}
\Zusatz{manas n.\ \artha{Denken; Geist}; māna m.\ \artha{Stolz}}\\
\vw{mnā}

\dhatu{mantr} \pres{mantrayati; -te \gan{X Ā}}, \pppp{mantrita} \artha{reden, (be)raten} 
\forms{
\PPF mantrayām āsa/cakāra
\FT mantrayiṣyati; -te
\GDV mantrya, mantritavya, mantraṇīya
\INFIN mantrayitum
\ABS mantrayitvā, -mantrya
}
\Zusatz{mantra m.\ \artha{(Zauber-)Spruch; Rat, Beratung}, mantrin m.\ \artha{Minister}}

\dhatu{manth} \vw{math/manth}

\dhatu{marj} \vw{mṛj}

\dhatu{malinay} \pres{malinayati} \gand{Denom.} \pppp{malinita} \artha{beschmutzen}
\Zusatz{v.\ malina \artha{schmutzig}}

\dhatu{mah} \pres{mahayati \gan{X PĀ}}, \pppp{mahita} \artha{verehren; sich ergötzen} 
\forms{
\PS mahyate
\GDV mahanīya
\ABS mahitvā
}

\dhatu{mahīy} \pres{mahīyate} \gand{Denom.} \pppp{---} \artha{übermütig sein, in Ehren stehen bei}
\Zusatz{v.\ mahant \artha{groß}}

\dhatu{mā} \pres{māti \gan{II P} | mimīte \gan{III Ā}}, \pppp{mita} \artha{messen, passen (in)} 
\forms{
\IND māti | mime, mimīṣe, mimīte, mimate  
\PF mamau; mame, mamire
\PS mīyate \PS \AO amāyi
\GDV meya, \mbox{-mātavya}
\INFIN mātum/mitum
\ABS mitvā, \mbox{-māya}
\CS māpayati \CS \GDV māpya
\DS mitsati
}\Zusatz{mita \artha{bemessen, klein}; nirmāṇa n.\ \artha{Werk}}

\dhatu{mārg} \pres{mārgati; -te \gan{I PĀ}}, \pppp{mārgita} \artha{suchen} 
\forms{
\FT mārgiṣyati
\GDV mārgitavya
\INFIN mārgitum
\ABS mārgitvā
\CS mārgayati
}
\Zusatz{mārga m.\ \artha{Pfad}}

\dhatu{mārṣṭ} \vw{mṛj}

\dhatu{mil} \pres{milati \gan{VI PĀ}}, \pppp{milita} \artha{zusammenkommen; sich einstellen} 
\forms{
\PF mimiluḥ
\FT miliṣyate
\GDV militavya, melanīya
\INFIN militum
\ABS militvā, \mbox{-milya}
\CS melayati
}
\Zusatz{mela m., melana n.\ \artha{Zusammenkunft}}

\dhatu{miśr} \pres{miśrayati \gan{X PĀ}}, \pppp{miśrita} \artha{mischen} 
\forms{
\PPF miśrayām āsa
\ABS miśrayitvā
}
\Zusatz{miśra \artha{vermischt; mannigfach}, °miśra \artha{mit \dots}}

\dhatu{miṣ} \pres{miṣati; -te \gan{VI P}}, \pppp{miṣita} \artha{(Augen, Knospen) öffnen} 
\forms{
\PF mimeṣa
\ABS \mbox{-miṣya}
}\Zusatz{un-meṣa m.\ \artha{Öffnen der Augen}, ni-meṣa m.\ \artha{Schließen der Augen}}

\dhatu{mih} \pres{mehati; -te \gan{I P}}, \pppp{mīḍha} \artha{urinieren} 
\forms{
\CS mehayati
}

\dhatu{mī} \pres{-minoti \gan{IX PĀ}}, \pppp{-mīta} \artha{mindern} 
\forms{
\PS \mbox{-mīyate} 
\GDV \mbox{-mātavya}
\CS \mbox{-māpayati}
}

\dhatu{mīl} \pres{mīlati; -te \gan{I P}}, \pppp{mīlita} \artha{die Augen schließen} 
\forms{
\PF mimīla
\PS mīlyate
\ABS \mbox{-mīlya}
\CS mīlayati; -te \CS \AO amimīlat
}

\dhatu{mukulay/mukulāy} \pres{mukulayati; mukulāyate} \gand{Denom.} \pppp{mukulita} \artha{knospen; sich schließen}
\Zusatz{v.\ mukula n.\ \artha{Knospe}}

\dhatu{muc} \pres{muñcati; -te \gan{VI PĀ}}, \pppp{mukta} \artha{lösen, befreien, verlassen,
(Pfeil) abschießen} 
\forms{
\PF mumoca; mumuce
\AO amucat; amukta
\FT mokṣyati; \mbox{-te}
\PS mucyate \PS \AO amoci
\GDV mocya, moktavya, mocanīya
\INFIN moktum
\ABS muktvā, \mbox{-mucya} \Zusatz{oft nur: \artha{ohne, außer}}
\CS mocayati; -te  \CS \AO amūmucat \CS \PS mocyate
\DS mumukṣati/mokṣate; mumukṣate  \DS \GDV mokṣayitavya
\CS \DS mumocayiṣati/mumokṣayiṣati
}
\Zusatz{mokṣa m.\ \artha{Befreiung}}\\
\vw{mokṣ(ay)}

\dhatu{muṇḍ(ay)} \pres{muṇḍayati \gan{I P}}, \pppp{muṇḍita} \artha{kahlscheren} 

\dhatu{mud} \pres{modate; -ti \gan{I Ā}}, \pppp{mudita} \artha{fröhlich sein} 
\forms{
\PF mumoda; mumude
\FT modiṣyate
\PS mudyate
\GDV modanīya
\INFIN moditum
\CS modayati; -te 
}
\Zusatz{ā-moda m.\ \artha{Freude; Wohlgeruch}}

\dhatu{mudray} \pres{mudrayati} \gand{Denom.} \pppp{mudrita} \artha{(ver)siegeln, stempeln}
\forms{
\GDV mudritavya
}
\Zusatz{v.\ mudrā f.\ \artha{Siegel}}

\dhatu{muṣ} \pres{muṣṇāti \gan{IX P} | muṣati}, \pppp{muṣita/muṣṭa} \artha{(be)stehlen, plündern} 
\forms{
\PF mumoṣa
\PS muṣyate \PS \AO amoṣi
\GDV moṣya, muṣitavya
\ABS muṣitvā, \mbox{-muṣya}
}
\Zusatz{saṃ-pra-moṣa m.\ \artha{Schwund}}

\dhatu{muh} \pres{muhyati; -te \gan{IV P}}, \pppp{mugdha/mūḍha} \artha{sich verwirren; ohnmächtig werden; fehlschlagen} 
\forms{
\PF mumoha; mumuhe
\FT mohiṣyati
\PS muhyate
\CS mohayati; -te \CS \PS mohyate \CS \PPP mohita
\IS momuhyate
}\Zusatz{mūḍha \artha{verwirrt}; mugdha \artha{unschuldig, naiv}}

\dhatu{mūtray} \pres{mūtrayati; -te} \gand{Denom.} \pppp{mūtrita} \artha{harnen}
\Zusatz{v.\ mūtra n.\ \artha{Harn}}

\dhatu{mūrch/murch} \pres{mūrcchati \gan{I P}}, \pppp{mūrcchita/mūrta} \artha{gerinnen, erstarren; ohnmächtig werden} 
\forms{
\PF mumūrccha
\CS mūrcchayati; -te
}\Zusatz{mūrta \artha{fest geworden, körperhaft}, mūrti f.\ \artha{Götterstatue}; mūrcchita \artha{ohnmächtig}}

\dhatu{mṛ} \pres{mriyate; -ti \gan{VI Ā}}, \pppp{mṛta} \artha{sterben}
\forms{
\PF mamāra,  mamartha,  mamriva, mamruḥ
\FT mariṣyati; -te
\GDV martya, martavya
\INFIN martum
\ABS mṛtvā
\CS mārayati \CS \AO amīmarat
\DS mumūrṣati
\IS marīmarti
}
\Zusatz{mṛtyu m.\ \artha{Tod, Todesgott}}

\dhatu{mṛg} \pres{mṛgayate \gan{X Ā}}, \pppp{---} \artha{jagen; (er)suchen} 
\forms{
\PPF mṛgayām āsa/cakāra
\FT mṛgayiṣyati; -te
\PS mṛgyate
\INFIN mṛgayitum
\ABS mṛgayitvā
}
\Zusatz{mṛga m.\ \artha{Gazelle}}

\dhatu{mṛj} \pres{mārṣṭi; mṛṣṭe \gan{II P} | mārjati; -te | mṛjati; -te}, \pppp{mṛṣṭa/mṛjita} \artha{(ab)wischen; entfernen} 
\forms{
\IND mārṣṭi,  mṛṣṭaḥ,  mṛjanti; mṛṣṭe, mṛjate | mārjati; -te | mṛjati; -te
\OP mṛjyāt
\IPV mārjāni,  mṛḍḍhi,  mārṣṭu,  mṛṣṭām,  mṛjantu
\IMP amārṭ,  amṛṣṭām,  amṛjan
\PF mamārja,  mamṛjuḥ; mamṛje
\AO amārjīt/amārkṣīt/amṛkṣat
\PS mṛjyate
\GDV mṛjya/mārgya, mārṣṭavya, mārjanīya
\INFIN m\shortlonga{}rṣṭum/mārjitum
\ABS mṛṣṭvā/mārjitvā, \mbox{-mṛjya}
\CS mārjayati; -te 
\CS \PPP mārjita
\CS \GDV mārjya
\CS \ABS \mbox{-mārjya}
\IS marmṛjyate
}

\dhatu{mṛd} \pres{mṛdnāti \gan{IX P} | mardati; -te}, \pppp{mṛdita} \artha{(ver- , zer-, ab-)reiben} 
\forms{
\PF mamarda, mamarduḥ
\PS mṛdyate
\GDV marditavya, mardanīya
\INFIN marditum
\ABS mṛditvā, \mbox{-mṛdya}
\CS mardayati; -te \CS \AO amīmṛdat \CS \PS mardyate \CS \PPP mardita
\DS mimardiṣati
}
\Zusatz{mṛd f.\ \artha{Erde}}

\dhatu{mṛś} \pres{mṛśati; -te \gan{VI P}}, \pppp{mṛṣṭa/mṛśita} \artha{berühren; überlegen} 
\forms{
\PF mamarśa, mamṛśuḥ; mamṛśe
\PS mṛśyate
\GDV \mbox{-mṛśya}, \mbox{-mṛṣṭavya}/\mbox{-marṣṭavya}, \mbox{-marśanīya}
\INFIN marṣṭum
\ABS \mbox{-mṛśya}
\CS marśayati
}
\Zusatz{vi-marśa m.\ \artha{Erwägung}}

\dhatu{mṛṣ} \pres{mṛṣyate; -ti \gan{IV ĀP}}, \pppp{mṛṣita} \artha{ertragen, vernachlässigen} 
\forms{
\PF mamṛṣe
\PS mṛṣyate
\GDV marṣaṇīya
\ABS \mbox{-mṛṣya}
\CS marṣayati; -te 
}

\dhatu{mṛṣṭ} \vw{mṛj}

\dhatu{mokṣ(ay)} \pres{mokṣayati \gan{X PĀ}}, \pppp{mokṣita} \artha{befreien; schleudern} 
\forms{
\GDV mokṣayitavya
\INFIN mokṣayitum
\ABS mokṣayitvā
}
\Zusatz{v.\ mokṣa m.\ \artha{Befreiung}}\\
\vw{muc}

\dhatu{mnā} \pres{(ā-)manati \gan{I P}}, \pppp{(ā-)mnāta} \artha{erwähnen, überliefern} 
\forms{
\AO amnāsīt, amnāsiṣuḥ
\PS (ā-)mnāyate
\INFIN (ā-)mnātum
}
\Zusatz{ā-mnāya m.\ \artha{Überlieferung}}

\dhatu{mlā/mlai} \pres{mlāyati; -te \gan{I P}}, \pppp{mlāna} \artha{welken, hinschwinden} 
\forms{
\IND mlāyati, mlāyanti/mlānti; mlāyate
\PF mamlau
\INJ mā mlāsīḥ
\CS ml\shortlonga{}payati
}
\Zusatz{mlāni f.\ \artha{Welken; Erschöpfung}}

\medskip

\dhatu{yacch} \vw{yam}

\dhatu{yaj} \pres{yajati; -te \gan{I PĀ}}, \pppp{iṣṭa} \artha{opfern, verehren} 
\forms{
\PF iyāja; īje
\AO ayākṣīt; ayaṣṭa
\FT yakṣyati; -te
\PS ijyate
\GDV ijya, yaṣṭavya, yajanīya
\INFIN yaṣṭum
\ABS iṣṭvā
\CS yājayati; -te \CS \GDV yājanīya
\DS yiyakṣati; -te
}
\Zusatz{yajamāna m.\ \artha{Opferpatron}; yajña m.\ \artha{Opfer}}

\dhatu{yat} \pres{yatate; -ti \gan{I Ā}}, \pppp{yat(i)ta} \artha{sich bemühen, streben} 
\forms{
\FT yatiṣyate; -ti
\PS yatyate
\GDV yatitavya, yatanīya
\INFIN yatitum
\ABS \mbox{-yatya}
\CS yātayati; -te \artha{quälen} 
}
\Zusatz{(pra-)yatna m.\ \artha{Mühe}}

\dhatu{yantr} \pres{yantrayati \gan{X PĀ}}, \pppp{yantrita} \artha{binden, fesseln} 
\forms{
\PS yantryate
\INFIN yantrayitum
\ABS yantrayitvā, \mbox{-yantrya}
}
\Zusatz{yantra n.\ \artha{Maschine}}

\dhatu{yam} \pres{yacchati; -te \gan{I P} | yamati; -te}, \pppp{yata} \artha{zügeln; darreichen} 
\forms{
\PF yayāma,  yayantha,  yemuḥ; yeme
\FT yamiṣyati
\PS yamyate
\GDV \mbox{-yamya}, yantavya/yamitavya, \mbox{-yamanīya}
\INFIN yantum/yamitum
\ABS ya(mi)tvā, \mbox{-yamya}
\CS y\shortlonga{}mayati; yamayate 
}
\Zusatz{yama m.\ \artha{Observanz; Todesgott}}

\dhatu{yas} \pres{yasyati \gan{IV P}}, \pppp{yas(i)ta} \artha{sich anstrengen} 
\forms{
\AO ayasat
\CS yāsayati
}
\Zusatz{ā-yāsa, pra-yāsa m.\ \artha{Anstrengung}}

\dhatu{yā} \pres{yāti \gan{II P}}, \pppp{yāta} \artha{gehen} 
\forms{
\OP yāyāt
\IPV yātu
\IMP ayāt, ayān/ayuḥ
\PF yayau; yaye
\AO ayāsīt
\FT yāsyati; -te
\PFT yātā
\PS yāyate
\GDV yātavya
\INFIN yātum
\ABS yātvā, \mbox{-yāya}
\CS yāpayati
\DS yiyāsati
}
\Zusatz{pra-yāṇa n.\ \artha{Aufbruch, Reise}}

\dhatu{yāc} \pres{yācati; -te \gan{I PĀ}}, \pppp{yācita} \artha{flehen, (er)bitten} 
\forms{
\PF yayāce
\AO ayācīt
\FT yāciṣyate
\PS yācyate
\GDV yācya, yācitavya, yācanīya
\INFIN yācitum
\ABS yācitvā, \mbox{-yācya}
\CS yācayati; -te
}
\Zusatz{yācaka m.\ \artha{Bittsteller}; yācñā f.\ \artha{Bitten; Bitte}}

\dhatu{yu} \pres{yauti \gan{II P}}, \pppp{yuta} \artha{befestigen; in Besitz nehmen} 
\forms{
\IND yauti, yuvanti
\OP yuyāt
\IPV yautu, yuvantu
\IMP ayaut, ayuvan
\ABS \mbox{-yūya}
}

\dhatu{yuj} \pres{yunakti; yuṅkte \gan{VII PĀ} | yuñjati; -te}, \pppp{yukta} \artha{zusammenfügen; anwenden; den Geist auf einen Punkt richten, beauftragen} 
\forms{
\IND yunajmi, yunakṣi, yunakti, yujmaḥ, yuṅktha, yuñjanti | yuñjati; -te
\IMP ayunajam, ayunak, ayunak, ayuñjva | ayuñjata  
\PF yuyoja; yuyuje
\AO ayujat; ayukta
\FT yokṣyati; -te
\PS yujyate 
\PS \AO ayoji/ayukṣi
\GDV yojya, yoktavya, yojanīya
\INFIN yoktum
\ABS yuktvā, \mbox{-yujya}
\CS yojayati; -te  \CS \AO ayūyujat \CS \PS yojyate \CS \GDV yojayitavya
\DS yuyukṣati
}
\Zusatz{yoga m.\ Name eines philosophischen Systems}

\dhatu{yudh} \pres{yudhyate; -ti \gan{IV Ā}}, \pppp{yuddha} \artha{(be)kämpfen} 
\forms{
\PF yuyudhe; yuyodha
\AO ayuddha; ayodhīt
\INJ mā yotsīḥ
\FT yotsyati; -te
\PFT yoddhā
\PS yudhyate
\GDV yodhya, yoddhavya, yodhanīya
\INFIN  yoddhum
\ABS yuddhvā, \mbox{-yudhya}
\CS yodhayati; -te
\DS yuyutsati; -te
}
\Zusatz{yudh m.\ \artha{Kämpfer}, f.\ \artha{Kampf, Schlacht}}

\medskip

\dhatu{rakṣ} \pres{rakṣati; -te \gan{I P}}, \pppp{rakṣita} \artha{schützen} 
\forms{
\PF rarakṣa
\AO arakṣīt
\FT rakṣiṣyati
\PFT rakṣitā
\PS rakṣyate
\GDV rakṣya, rakṣitavya, rakṣaṇīya
\INFIN rakṣitum
\ABS rakṣitvā, \mbox{-rakṣya}
\CS rakṣayati
\DS rirakṣiṣati
}
\Zusatz{raksaṇa n., rakṣā f.\ \artha{Schützen, Bewachen}}

\dhatu{rac} \pres{racayati \gan{X PĀ}}, \pppp{racita} \artha{verfertigen} 
\forms{
\PPF racayām āsa
\AO arīracat
\PS racyate \PS \AO araci
\INFIN  racayitum
\ABS racayitvā
}
\Zusatz{racanā f.\ \artha{Erzeugen; literarisches Werk}}

\dhatu{raj/rañj} \pres{rajyati; -te \gan{IV PĀ} | rañjati \gan{I PĀ}}, \pppp{rakta} \artha{sich färben; in Aufregung geraten, verliebt sein} 
\forms{
\PS rajyate
\GDV rañjya, ra(ñ)janīya
\ABS \mbox{-rajya}
\CS rañjayati; -te
}\Zusatz{rāga m.\ \artha{Leidenschaft}, rajaka m.\ \artha{Färber, Wäscher}}

\dhatu{raṭ} \pres{raṭati \gan{I P}}, \pppp{raṭita} \artha{heulen, krächzen} 
\forms{
\IS rāraṭīti
}

\dhatu{raṇ/ran} \pres{raṇati \gan{I P}}, \pppp{raṇita} \artha{klingen, tönen} 
\forms{
\CS raṇayati
}

\dhatu{rabh/rambh} \pres{-ra(m)bhate; -ti \gan{I Ā}}, \pppp{-rabdha} \artha{fassen, ergreifen, unternehmen, beginnen} 
\forms{
\PF \mbox{-rebhe}
\FT \mbox{-rapsyate}; -ti
\PS \mbox{-rabhyate} \PS \AO \mbox{-arambhi}
\GDV \mbox{-rabhya}, \mbox{-rabdhavya}
\INFIN \mbox{-rabdhum}
\ABS \mbox{-rabhya}
\CS \mbox{-rambhayati}; -te
\DS \mbox{-ripsate}
}\Zusatz{ā-rambha, prā-rambha m.\ \artha{Anfang; Unternehmung}}

\dhatu{ram} \pres{ramate; -ti \gan{I Ā}}, \pppp{rata} \artha{sich vergnügen (mit)} 
\forms{
\PF reme; rarāma, remuḥ
\AO araṃsīt; araṃsta
\FT raṃsyate; -ti
\GDV ramya, rantavya, ramaṇīya
\INFIN rantum/ramitum
\ABS ra(n)tvā, \mbox{-ramya}
\PS ramyate
\CS r\shortlonga{}mayati; ramayate
\DS riraṃsate
}\Zusatz{P wenn transitiv; °rata \artha{beschäftigt mit \dots}, °vi-rata \artha{aufgehört zu \dots}}

\dhatu{ras [1]} \pres{rasati; -te \gan{I P}}, \pppp{rasita} \artha{brüllen, heulen} 
\forms{
\PF rarāsa, resuḥ
}

\dhatu{ras [2]} \pres{rasayati; -te \gan{X PĀ} | rasyati}, \pppp{---} \artha{schmecken; empfinden} 

\dhatu{rah} \pres{rahayati \gan{X PĀ}}, \pppp{rahita} \artha{verlassen} 
\forms{
\INFIN rahitum
}
\Zusatz{rahas n.\ \artha{Einsamkeit; Geheimnis}}

\dhatu{rā} \pres{rāti \gan{II P}}, \pppp{rāta} \artha{geben} 
\forms{
\PFT rātā
}

\dhatu{rāj} \pres{rājati; -te \gan{I PĀ}}, \pppp{rājita} \artha{herrschen; glänzen, (er)scheinen} 
\forms{
\PF rarāja; reje
\PS rājyate
\CS rājayati
}

\dhatu{rādh} \pres{rādhnoti \gan{V P} | rādhyati; -te \gan{IV P}}, \pppp{rāddha} \artha{zu Stande bringen} 
\forms{
\PS rādhyate
\GDV \mbox{-rādhya}, \mbox{-rādhanīya}
\INFIN rāddhum
\ABS \mbox{-rādhya}
\CS rādhayati \CS \PS rādhyate
}
\Zusatz{aparādha m.\ \artha{Vergehen}}

\dhatu{rās} \pres{rāsate; -ti \gan{I Ā}}, \pppp{---} \artha{heulen, schreien} 
\forms{
\PF rarāse
\IS rārāsyate
}

\dhatu{ric} \pres{ricyate; -ti \gan{VII PĀ}}, \pppp{rikta} \artha{leer werden} 
\forms{
\AO aricat
\GDV \mbox{-recya}, \mbox{-recanīya}
\ABS \mbox{-ricya}
\CS recayati 
}
\Zusatz{°(vi-)rikta \artha{ohne \dots}}

\dhatu{riṣ} \pres{riṣyati; -te \gan{I P}}, \pppp{riṣṭa} \artha{Schaden nehmen, schädigen} 
\forms{
\CS reṣayati
}

\dhatu{ru} \pres{rauti \gan{II P} | ruvati | ravati; -te}, \pppp{ruta} \artha{brüllen} 
\forms{
\IND rauti, rutaḥ, ruvanti | ruvati | ravati; -te
\PF rurāva,  ruruvuḥ
\INFIN rotum
\CS rāvayati \CS \AO arūruvat
\IS roravīti; rorūyate
}
\Zusatz{rava, rāva m.\ \artha{Gebrüll, Ton}}

\dhatu{ruc} \pres{rocate; -ti \gan{I Ā}}, \pppp{rucita} \artha{leuchten; jmnd.\ gefallen} 
\forms{
\PF ruruce
\AO arociṣṭa; arucat
\FT rociṣyate
\INFIN rocitum
\CS rocayati; -te \CS \PS rocyate
}
\Zusatz{idaṃ me rocate \artha{dies gefällt mir}; ruci f.\ \artha{Gefallen, Appetit; Pracht}}

\dhatu{ruj} \pres{rujati; -te \gan{VI P}}, \pppp{rugṇa} \artha{zerbrechen; schmerzen} 
\forms{
\PF ruroja, rurujuḥ
\PS rujyate
\ABS \mbox{-rujya}
\CS \AO arūrujat
}
\Zusatz{roga m., ruj f.\ \artha{Krankheit}}

\dhatu{rud} \pres{roditi \gan{II P} | rudati; -te | rodati; -te}, \pppp{rudita} \artha{(be)weinen} 
\forms{
\IND rodimi, rodiṣi, roditi, rudima, ruditha, rudanti | rudati; -te | rodati; -te
\OP rudyāt
\IPV rodāni, rudihi, roditu, rodāva, rudantu
\IMP arodam, arodaḥ/arodīḥ, arodat/arodīt, arudiva, arudan
\PF ruroda, ruruduḥ; rurude
\AO araudiṣīt
\FT rodiṣyati
\PS rudyate
\GDV roditavya
\INFIN roditum
\ABS ruditvā/roditvā, \mbox{-rudya}
\CS rodayati
\DS rurudiṣati
\IS rorudyate
}

\dhatu{rudh} \pres{ruṇaddhi; runddhe \gan{VII PĀ} | rodhati | rundhati; -te}, \pppp{ruddha} \artha{zurückhalten; verdecken, belagern; anhalten} 
\forms{
\IND ruṇaddhi, rundhanti; runddhe, rundhate | rodhati | rundhati; -te
\PF rurodha; rurudhe
\AO arudhat/arautsīt
\FT rotsyati; -te
\PS rudhyate
\GDV rodhya, roddhavya/-rodhitavya
\INFIN roddhum/rodhitum
\ABS ru(n)ddhvā, \mbox{-rudhya}
\CS rodhayati/rundhayati \CS \PS rodhyate
\DS rurutsate; -ti
}
\Zusatz{avarodha m.\ \artha{Harem}}

\dhatu{ruṣ} \pres{ruṣyati; -te \gan{IV P} | ruṣati}, \pppp{ruṣita/ruṣṭa} \artha{zürnen} 
\forms{
\CS roṣayati; -te 
}
\Zusatz{ruṣ f.\ \artha{Zorn}}

\dhatu{ruh} \pres{rohati; -te \gan{I P} | ruhati; -te}, \pppp{rūḍha} \artha{wachsen; ersteigen} 
\forms{
\PF ruroha, ruruhuḥ; ruruhe
\AO aruhat
\FT rokṣyati; rokṣyate/rohiṣyate
\PS ruhyate
\GDV \mbox{-roḍhavya}, rohaṇīya
\INFIN roḍhum/rohitum
\ABS \mbox{-ruhya}
\CS rohayati/ropayati; -te
\CS \PS ropyate
\CS \PS \AO aropi
\CS \GDV ropaṇīya
\DS rurukṣati
}

\dhatu{rūp} \pres{rūpayati \gan{X PĀ}}, \pppp{rūpita} \artha{gestalten, darstellen} 
\Zusatz{nirūpaṇa n.\ \artha{Bestimmung}}

\dhatu{rūṣ} \pres{(saṃ-)roṣayati | --- \gan{I P}}, \pppp{rūṣita} \artha{bestäubt, bestreut mit} 

\medskip

\dhatu{lakṣ} \pres{lakṣayati; -te \gan{X PĀ} | lakṣate; -ti}, \pppp{lakṣita} \artha{bezeichnen; bemerken}  
\forms{
\PS lakṣyate
\GDV lakṣya, lakṣ(ay)itavya, lakṣaṇīya
\ABS lakṣayitvā, \mbox{-lakṣya}
}
\Zusatz{lakṣaṇa n.\ \artha{Definition}}

\dhatu{lag} \pres{lagati \gan{I P}}, \pppp{lagna} \artha{haften} 
\forms{
\FT lagiṣyati
\GDV laganīya
\ABS lagitvā, \mbox{-lagya}
\CS l\shortlonga{}gayati
}

\dhatu{laṅgh} \pres{laṅghati; -te \gan{I P}}, \pppp{laṅghita} \artha{überspringen} 
\forms{
\PS laṅghyate
\GDV laṅghya
\INFIN laṅghitum
\ABS laṅghitvā, \mbox{-laṅghya}
\CS laṅghayati 
}

\dhatu{lajj} \pres{lajjate; -ti \gan{VI Ā}}, \pppp{lajjita} \artha{sich schämen} 
\forms{
\PF lalajje, lalajjire
\INFIN lajjitum
\CS lajjayati
}
\Zusatz{lajjā f.\ \artha{Scham}}

\dhatu{lap} \pres{lapati; -te \gan{I P}}, \pppp{lap(i)ta} \artha{schwatzen, klagen} 
\forms{
\PF lalāpa, lepuḥ
\FT lapiṣyati
\PS lapyate
\GDV \mbox{-lāpya}, \mbox{-lapitavya}, lapanīya
\INFIN lap(i)tum
\ABS \mbox{-lapya}
\CS lāpayati \CS \GDV \mbox{-lāpya}
\IS lālapyate
}
\Zusatz{pra-lāpa \artha{Geschwätz; Wehklage}}

\dhatu{labh} \pres{labhate \gan{I Ā}}, \pppp{labdha} \artha{ergreifen, erlangen, wahrnehmen} 
\forms{
\PF lebhe; lalābha
\AO alabdha
\FT lapsyate; lapsyati/labhiṣyati
\PFT labdhā
\PS labhyate
\GDV labhya, labdhavya, la(m)bhanīya
\ABS labdhvā, \mbox{-labhya}
\CS lambhayati
\DS lipsate; -ti \DS \GDV lipsitavya
}
\Zusatz{su-labha \artha{leicht zu erlangen}, dur-labha \artha{schwer zu erlangen}}

\dhatu{lamb} \pres{lambate; -ti \gan{I Ā}}, \pppp{lambita} \artha{(herab)hängen; zögern} 
\forms{
\PF lalambe
\FT lambiṣyati; -te
\PS lambyate
\GDV lambya, lambitavya, lambanīya
\INFIN lambitum
\ABS \mbox{-lambya}
\CS lambayati; -te
}\Zusatz{ava-lamba m.\ \artha{Halt, Stütze}, ava-lambya \artha{mittels}}

\dhatu{lambh} \vw{labh}

\dhatu{lal} \pres{lalati; -te \gan{I P}}, \pppp{lalita} \artha{scherzen, spielen} 
\forms{
  \CS lālayati; -te \artha{hätscheln} 
  \CS (ul-)lalayati
\CS \PS lālyate
\CS \GDV lālya, lālayitavya, lālanīya
}
\Zusatz{lalanā f.\ \artha{(tändelnde) Frau}}

\dhatu{laṣ} \pres{-laṣati; -te \gan{I PĀ}}, \pppp{-laṣita} \artha{begehren} 
\forms{
\PF \mbox{-lalāṣa}, \mbox{-leṣuḥ}
\FT \mbox{-laṣiṣyati}
\GDV \mbox{-laṣya}, \mbox{-laṣaṇīya}
}
\Zusatz{abhi-lāṣa m.\ \artha{Verlangen}}

\dhatu{las} \pres{lasati; -te \gan{I PĀ}}, \pppp{lasita} \artha{strahlen, (er)scheinen; lustig sein} 
\forms{
\PF lalāsa
\CS lāsayati
\CS \PS lāsyate \CS \GDV lāsya
}
\Zusatz{ul-lāsa m.\ \artha{Erscheinen; Freude}}

\dhatu{likh} \pres{likhati \gan{VI P}}, \pppp{likhita} \artha{schreiben, ritzen} 
\forms{
\PF lilekha
\FT likhiṣyati
\PS likhyate
\GDV likhya/lekhya, lekhanīya
\INFIN likhitum
\ABS likhitvā/lekhitvā, \mbox{-likhya}
\CS lekhayati/likhāpayati \CS \AO alīlikhat
}
\Zusatz{lekha m., lekhā f.\ \artha{Strich}; ul-lekha m.\ \artha{Schilderung}}

\dhatu{lip/limp} \pres{limpati; -te \gan{VI PĀ}}, \pppp{lipta} \artha{beschmieren} 
\forms{
\PF lilepa, lilipuḥ
\AO alipta
\PS lipyate \PS \AO alepi
\GDV lepya
\ABS liptvā, \mbox{-lipya}
\CS lepayati
}
\Zusatz{lipi f.\ \artha{Schrift}}

\dhatu{lih} \pres{leḍhi; līḍhe \gan{II PĀ} | lihati}, \pppp{līḍha} \artha{lecken} 
\forms{
\PF lileha; lilihe
\PS lihyate
\GDV lehya
\ABS \mbox{-lihya}
\CS lehayati
\IS lelihyate \IS \PPA lelihāna
}

\dhatu{lī} \pres{līyate; -ti \gan{IV Ā}}, \pppp{līna} \artha{sich anschmiegen; kauern, verschwinden} 
\forms{
\PF lilye; lilāya, lilyuḥ
\ABS \mbox{-līya}
\CS lāyayati/lāpayati 
}

\dhatu{līlāy} \pres{līlāyate; -ti} \gand{Denom.} \pppp{līlāyita} \artha{spielen}
\Zusatz{v.\ līlā f.\ \artha{Spiel}}

\dhatu{luñc} \pres{luñcati \gan{I P}}, \pppp{luñcita} \artha{rupfen} 
\forms{
\PF luluñcuḥ; luluñce
\ABS luñcitvā
}

\dhatu{luṭh [1]} \pres{luṭhati; -te \gan{VI P}}, \pppp{luṭhita} \artha{sich wälzen}
\forms{
\PF luloṭha
\CS loṭhayati
\DS luluṭhiṣate
\IS loluṭhīti
}

\dhatu{luṭh [2]/luṇṭh} \pres{luṇṭhayati/loṭhayati \gan{X PĀ}}, \pppp{luṇṭhita} \artha{plündern} 
\forms{
\FT luṇṭhiṣyati
\PS luṇṭhyate
\INFIN luṇṭh(ay)itum
\ABS luṇṭhayitvā, \mbox{-luṇṭhya}
}

\dhatu{luḍ} \pres{loḍayati | --- \gan{I P}}, \pppp{loḍita} \artha{aufrühren, beunruhigen}  
\forms{
\PS loḍyate
\ABS \mbox{-loḍya}
}

\dhatu{lup/lump} \pres{lumpati; -te \gan{VI PĀ}}, \pppp{lupta} \artha{beschädigen, rauben} 
\forms{
\PF lulopa; lulupe
\PS lupyate
\GDV lopya
\INFIN loptum
\ABS luptvā, \mbox{-lupya}
\CS lopayati; \mbox{-te} \CS \AO alūlupat \CS \PS lopyate
\IS lolupyate
}
\Zusatz{lopa m.\ \artha{Schwund bestimmter gramm.\ Morpheme; Mangel}}

\dhatu{lubh} \pres{lubhyati \gan{IV P}}, \pppp{lubdha} \artha{verlangen nach; anlocken} 
\forms{
\PF lulobha; lulubhe
\GDV \mbox{-lobhya}, lobhanīya
\INFIN lobdhum
\CS lobhayati; -te \CS \PS lobhyate \CS \ABS \mbox{-lobhya}
\IS lolubhyate
}
\Zusatz{lobha m.\ \artha{Gier}}

\dhatu{lump} \vw{lup/lump}

\dhatu{lul} \pres{lolati \gan{I P}}, \pppp{lulita} \artha{sich hin und her bewegen} 
\forms{
\CS lolayati
}
\Zusatz{lola \artha{schwankend; begehrend}}

\dhatu{lū} \pres{lunāti; lunīte \gan{IX PĀ}}, \pppp{lūna} \artha{(ab)schneiden} 
\forms{
\PF lulāva; luluve
\GDV lavya
}

\dhatu{lok} \pres{lokayati; -te \gan{X PĀ} | lokate \gan{I Ā}}, \pppp{lokita} \artha{erblicken, erkennen} 
\forms{
\PF luloke
\PPF -lokayām āsa
\PS lokyate \PS \AO aloki
\GDV \mbox{-lokya}, \mbox{-lokayitavya}, \mbox{-lokanīya}
\INFIN lokitum
\ABS \mbox{-lokya}
}
\Zusatz{loka m.\ \artha{Welt}; ā-loka m.\ \artha{Anblick}}

\dhatu{loc} \pres{-locayati; -te \gan{X PĀ} | --- \gan{I Ā}}, \pppp{-locita} \artha{betrachten, erwägen} 
\forms{
\PF \mbox{-luloce}
\PS \mbox{-locyate}
\GDV \mbox{-locya}, \mbox{-locanīya}
\INFIN \mbox{-locitum}
\ABS \mbox{-lokya}
}
\Zusatz{locana n.\ \artha{Auge}}

\medskip

\dhatu{vac} \pres{vakti \gan{II P}}, \pppp{ukta} \artha{sprechen} 
\forms{
\IND vacmi, vakṣi, vakti, vacvaḥ, vakthaḥ, vaktaḥ, vacmaḥ, vaktha, [vadanti, ergänzt v.\ \vw{vad}]
\OP vacyāt
\IPV vacāni, vagdhi, vaktu
\IMP avacam, avak, avak, avacva, avaktam, avaktām, avacma, avakta, [avadan, ergänzt v.\ \vw{vad}]
\PF uvāca, ūcuḥ; ūce, ūcire
\AO avocat; avocata
\FT vakṣyati; -te
\PFT vaktā
\PS ucyate \PS \AO avoci/avāci
\GDV vācya, vaktavya, vacanīya
\INFIN vaktum
\ABS uktvā, \mbox{-ucya}
\CS vācayati; -te  \CS \PS vācyate \CS \GDV vācanīya
\DS vivakṣati; -te \DS \PS vivakṣyate
}
\Zusatz{vāc f.\ \artha{Rede}; vacana n.\ \artha{Reden}, eka°, dvi°, bahu-vacana n.\ \artha{Singular, Dual, Plural}}

\dhatu{vañc} \pres{vañcayati \gan{X Ā}}, \pppp{vañcita} \artha{entkommen; täuschen} 
\forms{
\PS va(ñ)cyate
\GDV vañcayitavya, vañcanīya
\INFIN vañcitum
}
\Zusatz{vañcana n.\ \artha{Betrug}}

\dhatu{vad} \pres{vadati; -te \gan{I P}}, \pppp{udita} \artha{reden} 
\forms{
\PF uvāda,  ūduḥ
\AO avādīt
\FT vadiṣyati
\PS udyate \PS \AO avādi
\GDV vādya, vaditavya
\INFIN vaditum
\ABS uditvā, \mbox{-udya}
\CS vādayati; -te 
\DS vivadiṣati
}
\Zusatz{vāda m.\ \artha{Rede; Lehre}, vādin \artha{redend}, °vādin m.\ \artha{Vertreter einer Lehre}}

\dhatu{vadh} \pres{[hanti \gan{II P}}, \pppp{hata]} \artha{töten, schlagen}
\forms{
\AO avadhīt
\FT vadhiṣyati; -te
\PS vadhyate \PS \AO avadhi
\GDV vadhya
\CS vadhayati
}
\Zusatz{unvollständiges, präsentische Formen und PPP werden von \vw{han} ergänzt}

\dhatu{vand} \pres{vandate; -ti \gan{I Ā}}, \pppp{vandita} \artha{ehren, begrüßen} 
\forms{
\PF vavande
\PS vandyate
\GDV vandya, vanditavya, vandanīya
\INFIN vanditum
\ABS vanditvā, \mbox{-vandya}
\CS vandayati
}
\Zusatz{pāda-(abhi-)vandana n.\ \artha{Verehrung der Füße = ehrfurchtsvolle Verehrung}}

\dhatu{vap} \pres{vapati; -te \gan{I PĀ}}, \pppp{upta} \artha{säen} 
\forms{
\PF uvāpa, uvapitha/uvaptha, ūpuḥ
\AO avāpsīt
\FT vapsyati/vapiṣyati
\PS upyate
\GDV vāpya, vaptavya, vapanīya
\ABS uptvā, \mbox{-upya}
\CS vāpayati
}

\dhatu{vam} \pres{vamati \gan{I P} | vamiti}, \pppp{vānta} \artha{erbrechen} 
\forms{
\IND vamiti | vamati, vamanti
\IMP avamat
\PF vavāma, vemuḥ
\GDV vamitavya
\INFIN vamitum
\ABS vamitvā
\CS v\shortlonga{}mayati \CS \GDV vāmanīya
}

\dhatu{varṇ} \pres{varṇayati \gan{X PĀ}}, \pppp{varṇita} \artha{bemalen; beschreiben; erzählen} 
\forms{
\PPF varṇayām āsa/cakāra
\PS varṇyate
\GDV varṇayitavya, varṇanīya
\INFIN varṇ(ay)itum
}
\Zusatz{varṇana n.\ \artha{Schilderung}}


\dhatu{val} \pres{valate; -ti \gan{I Ā}}, \pppp{valita} \artha{sich wenden; sich äußern} 
\forms{
\PF vavale
\ABS valitvā
\CS valayati
}

\dhatu{valg} \pres{valgati; -te \gan{I PĀ}}, \pppp{valgita} \artha{hüpfen} 
\forms{
\PF vavalga
}

\dhatu{vaś} \pres{vaṣṭi \gan{II P}}, \pppp{---} \artha{verlangen} 
\forms{
\IND vaśmi, vakṣi, vaṣṭi, uśvaḥ, uṣṭhaḥ, uṣṭaḥ, uśmaḥ, uṣṭha, uśanti
\OP uśyāt
\IPV vaśāni,  uḍḍhi,  vaṣṭu, vaśāva, uṣṭam, uṣṭām, vaśāma, uṣṭa, uśantu
\IMP avaśam, avaṭ, avaṭ, auśva, auṣṭam, auṣṭām, auśma, auṣṭa, auśan
\INJ mā vaśīḥ
\CS vaśayati
}
\Zusatz{vaśa m.\ \artha{Wille, Wunsch}}

\dhatu{vas [1]} \pres{vaste \gan{II Ā}}, \pppp{vasita} \artha{anziehen, sich kleiden} 
\forms{
\IND vaste, vasate
\PF vavase
\GDV vasitavya
\INFIN vasitum
\ABS vasitvā, \mbox{-vasya}
\CS vāsayati; \mbox{-te} \CS \PPP vāsita \CS \GDV vāsya
}
\Zusatz{carma-vasana \artha{in Fell gekleidet}}

\dhatu{vas [2]} \pres{vasati; -te \gan{I P}}, \pppp{uṣita/uṣṭa/vasita} \artha{wohnen} 
\forms{
\PF uvāsa, ūṣuḥ
\AO avātsīt
\FT vatsyati/vasiṣyati; vatsyate
\PS uṣyate
\GDV \mbox{-uṣya}, vastavya/\mbox{-uṣitavya}
\INFIN vas(i)tum
\ABS uṣitvā/uṣṭvā, \mbox{-uṣya}
\CS vāsayati; \mbox{-te}
\CS \PS vāsyate
\CS \PPP vāsita
\CS \GDV vāsya, vāsayitavya
}
\Zusatz{ni-v\shortlonga{}sana n.\ \artha{Behausung}}

\dhatu{vah} \pres{vahati; -te \gan{I PĀ}}, \pppp{ūḍha} \artha{fahren; etw.\ befördern; erdulden} 
\forms{
\PF uvāha, ūhuḥ
\AO avākṣīt
\FT vakṣyati/vahiṣyati; vakṣyate
\PS uhyate \PS \AO avāhi
\GDV vāhya
\INFIN voḍhum
\ABS \mbox{-uhya}
\CS vāhayati; -te \CS \PS vāhyate \CS \GDV vāhanīya
\IS vāvahīti
}
\Zusatz{vāhana n.\ \artha{Gefährt}, ati-vāhana n.\ \artha{Verbringen}}

\dhatu{vā [1]} \pres{vāti \gan{II P} | vāyati; -te}, \pppp{vāna} \artha{wehen, sich verbreiten; erschöpft werden} 
\forms{
\PF vavau
\FT vāsyati
\INFIN vātum
\CS vāpayati \CS \GDV vāpayitavya
}
\Zusatz{nir-vāṇa n.\ \artha{Befreiung (rel.)}}

\dhatu{vā [2]} \vw{ve/vā/vi}

\dhatu{vāñch} \pres{vāñchati \gan{I P}}, \pppp{vāñchita} \artha{begehren} 
\forms{
\PS vāñchyate
\CS vāñchayati
}
\Zusatz{vāñchā f.\ \artha{Begehren}}

\dhatu{vāś} \pres{vāśyate; -ti \gan{IV Ā} | vāśati; -te}, \pppp{vāśita} \artha{brüllen} 
\forms{
\PF vavāśe
\PS vāśyate
\ABS vāśitvā, \mbox{-vāśya}
\IS vāvāśyate
}

\dhatu{vāh} \pres{vāhate \gan{I Ā}}, \pppp{vāhita} \artha{drängen, drücken} 
\forms{
\CS vāhayati
}

\dhatu{vi} \vw{ve/vā/vi}

\dhatu{vighnay} \pres{vighnayati} \gand{Denom.} \pppp{vighnita} \artha{hindern}
\Zusatz{vighna m.\ \artha{Hindernis}}

\dhatu{viḍamb} \vw{ḍamb}

\dhatu{vic} \pres{-vinakti \gan{VII PĀ}}, \pppp{-vikta} \artha{sondern; erwägen} 
\forms{
\FT \mbox{-vekṣyati}
\PS \mbox{-vicyate}
\GDV \mbox{-vekya}, \mbox{-vektavya}
\INFIN  \mbox{-vektum}
\ABS \mbox{-vicya}
\CS \mbox{-vecayati}
}
\Zusatz{vi-veka m.\ \artha{Untersuchung; Verstand}}

\dhatu{vij} \pres{vijate; -ti \gan{VI Ā} | vejate}, \pppp{vigna} \artha{wanken, zittern} 
\forms{
\PF vivije
\FT vijiṣyati/vejiṣyati
\ABS \mbox{-vijya}
\CS vejayati; -te
}
\Zusatz{vega m.\ \artha{Hast, Kraft}; ud-vega m.\ \artha{Zittern}}

\dhatu{vid [1]} \pres{vetti \gan{II P} | veda}, \pppp{vidita} \artha{wissen, erkennen}
\forms{
\IND vedmi, vetsi, vetti, vidvaḥ, vitthaḥ, vittaḥ, vidmaḥ, vittha, vidanti
\abbrev{präs.\ pf} veda, vettha, veda, vikṣa, vidathuḥ, vidatuḥ, vidma, vida, viduḥ
\OP vidyāt
\IPV vedāni, viddhi, vettu, vedāva, vittam, vittām, vedāma, vitta, vidantu
\IMP avedam, aveḥ/avet, avet, avidva, avittam, avittām, avidma, avitta, aviduḥ
\PF viveda
\PPF vidāṃ cakāra
\AO avedīt
\FT vetsyati; -te
\PFT vettā
\PS vidyate
\GDV vedya, veditavya, vedanīya
\INFIN veditum/vettum
\ABS viditvā
\CS vedayati; -te \CS \PPF \mbox{-vedayām} āsa \CS \PS vedyate
\DS vividiṣati/vivitsati
}
\Zusatz{veda m.\ \artha{der Veda, (heiliges) Wissen}}

\dhatu{vid [2]/vind} \pres{vindati; -te \gan{VI PĀ}}, \pppp{vitta/vinna} \artha{finden} 
\forms{
\PF viveda; vivide
\AO avidat; avidata
\FT vetsyati; -te
\PS vidyate \artha{es gibt} 
\GDV vedya, vettavya, vedanīya
\INFIN vettum
\ABS vittvā, \mbox{-vidya}
\CS vedayati
\DS vivitsati
}
\Zusatz{vitta n.\ \artha{Vermögen}}

\dhatu{vidh} \vw{ vyadh}

\dhatu{vip/vep} \pres{vepate; -ti \gan{I Ā}}, \pppp{---} \artha{zittern, beben} 
\forms{
\CS vepayati
}

\dhatu{viś} \pres{viśati; -te \gan{VI P}}, \pppp{viṣṭa} \artha{eintreten} 
\forms{
\PF viveśa, viviśuḥ; viviśe
\AO avikṣat
\FT vekṣyati; -te
\PS viśyate \PS \AO aveśi
\GDV veśya, veṣṭavya, \mbox{-veśanīya}
\INFIN veṣṭum
\ABS \mbox{-viśya}
\CS veśayati; -te \CS \PPF \mbox{-veśayām} āsa \CS \AO avīviśat \CS \PS veśyati \CS \GDV \mbox{-veśayitavya}
\DS vivikṣati
}
\Zusatz{pra-veśa m.\ \artha{Eintritt; das Auftreten (im Schauspiel)}}

\dhatu{vī} \vw{vyā/vye/vī}

\dhatu{vīj} \pres{vījati; -te \gan{I PĀ}}, \pppp{---} \artha{befächeln}
\forms{
\PF vivyajuḥ
\CS vījayati \CS \PS vījyate
}

\dhatu{vṛ [1]} \pres{vṛṇoti; vṛṇute \gan{V PĀ}}, \pppp{vṛta} \artha{verhüllen, abwehren} 
\forms{
\PF vavāra, vavartha, vavṛva,  vavruḥ;  vavre
\PS vriyate
\GDV vārya, varitavya, varaṇīya
\INFIN var\shortlongi{}tum
\ABS \mbox{-vṛtya}
\CS vārayati , vārayitavya
}
\Zusatz{ā-varaṇa n.\ \artha{Verhüllung, Gewand}}

\dhatu{vṛ [2]/vṝ} \pres{vṛṇīte; vṛṇāti \gan{IX Ā} | vṛṇoti; vṛṇute  \gan{V PĀ}}, \pppp{vṛta} \artha{wählen} 
\forms{
\PF vavre
\PS vriyate
\GDV vṛtya/vārya, varītavya, varaṇīya
\INFIN varītum
\ABS \mbox{-vṛtya}
\CS v\shortlonga{}rayati; varayate  \CS \GDV varayitavya
}
\Zusatz{vara \artha{vorzüglich}, m.\ \artha{Segenswunsch; Geliebter}}

\dhatu{vṛṃh} \vw{bṛṃh}

\dhatu{vṛj} \pres{varjayati; -te \gan{X PĀ}}, \pppp{varjita/vṛkta} \artha{beseitigen, verzichten auf} 
\forms{
\PPF varjayām āsa
\PS varjyate
\GDV varjya, varjayitavya, varjanīya
\ABS varjayitvā
}
\Zusatz{oft nur: \artha{ohne, außer}; °varjita \artha{ohne \dots}}

\dhatu{vṛt} \pres{vartate; -ti \gan{I Ā}}, \pppp{vṛtta} \artha{sich drehen, existieren} 
\forms{
\PF vavṛte; vavarta, vavṛtuḥ
\AO avṛtat
\FT vartsyate/vartiṣyate; -ti
\PFT vartitā
\GDV \mbox{-vartya}, vart(i)tavya, vartanīya
\INFIN vartitum
\ABS vartitvā, \mbox{-vṛtya}
\CS vartayati; -te
\CS \GDV \mbox{-vartayitavya}
\IS varīvṛtyate
}
\Zusatz{vṛtti f.\ \artha{Geschehen; Lebensunterhalt; Kommentar zu einem Sūtra}; vārttā f.\ \artha{Kunde, Nachricht}}

\dhatu{vṛdh} \pres{vardhati; -te \gan{I P}}, \pppp{vṛddha} \artha{wachsen (lassen)} 
\forms{
\PF vavardha; vavṛdhe
\AO avṛdhat; avardhiṣṭa
\FT vartsyati
\GDV vardhitavya, vardhanīya
\INFIN vardhitum
\CS vardhayati/vardhāpayati; vardhayate
\CS \AO avīvṛdhat
}
\Zusatz{vṛddha \artha{alt}}

\dhatu{vṛśc} \vw{vraśc}

\dhatu{vṛṣ} \pres{varṣati; -te \gan{I P}}, \pppp{vṛṣṭa} \artha{(be)regnen} 
\forms{
\PF vavarṣa, vavṛṣuḥ/vavarṣuḥ; vavṛṣe
\FT varṣiṣyate
\GDV vṛṣya/varṣya
\INFIN  varṣitum
\ABS \mbox{-vṛṣya}
\CS varṣayati  \CS \AO avīvṛṣat
}
\Zusatz{varṣa m.\altern n., vṛṣṭi f.\ \artha{Regen}}

\dhatu{vṛh} \vw{bṛh}

\dhatu{vṝ} \vw{vṛ [2]/vṝ}

\dhatu{ve/vā/vi} \pres{vayati; -te \gan{I PĀ}}, \pppp{uta} \artha{weben} 
\forms{
\PS ūyate
\GDV vātavya
}

\dhatu{vep} \vw{vip/vep}

\dhatu{vell} \pres{vellati \gan{I P}}, \pppp{vellita} \artha{schwanken, taumeln} 
\forms{
}

\dhatu{veṣṭ} \pres{veṣṭate \gan{I Ā}}, \pppp{veṣṭita} \artha{sich winden, sich hängen an} 
\forms{
\PS veṣṭyate
\GDV \mbox{-veṣṭanīya}
\CS veṣṭayati; -te \artha{umhüllen; kleiden} \CS \INFIN veṣṭitum
}

\dhatu{vyath} \pres{vyathate; -ti \gan{I Ā}}, \pppp{vyathita} \artha{schwanken} 
\forms{
\PF vivyathe; vivyathuḥ
\INJ mā vyathiṣṭhāḥ
\INFIN vyathitum
\CS vyathayati \CS \PS vyathyate
}
\Zusatz{vyathā f.\ \artha{Schmerz, Fehlschlag}}

\dhatu{vyadh} \pres{vidhyati; -te \gan{IV P}}, \pppp{viddha} \artha{durchbohren} 
\forms{
\PF vivyādha, vividhuḥ; vivyadhe
\PS vidhyate
\GDV vyadhya/vedhya, veddhavya,
\INFIN veddhum
\ABS viddhvā, \mbox{-vidhya}
\CS vy\shortlonga{}dhayati/vedhayati \CS \AO avīvidhat
}\Zusatz{°viddha \artha{behaftet mit \dots}; vyādha(ka) m.\ \artha{Jäger}}

\dhatu{vyay} \pres{vyayati; -te \gan{I PĀ}}, \pppp{vyayita} \artha{verausgaben} 

\dhatu{vyā/vye/vī} \pres{vyayati; -te \gan{I PĀ}}, \pppp{vīta} \artha{umhüllen, bergen} 

\dhatu{vyāpar/vyāpṛ/vyāpri} \vw{pṛ}

\dhatu{vraj} \pres{vrajati; -te \gan{I P}}, \pppp{vrajita} \artha{wandern} 
\forms{
\PF vavrāja, vavrajitha
\AO avrājīt
\FT vrajiṣyati
\PS vrajyate
\GDV \mbox{-vrajya}
\INFIN vrajitum
\ABS vrajitvā, \mbox{-vrajya}
\CS vrājayati  \CS \PS vrājyate
}
\Zusatz{pari-vrājaka m.\ \artha{Wanderasket}}

\dhatu{vraśc} \pres{vṛścati \gan{VI P}}, \pppp{vṛkṇa} \artha{abhauen, fällen}
\forms{
\PS vṛścyate
\ABS \mbox{-vṛścya}
}

\dhatu{vrīḍ} \pres{vrīḍate | --- \gan{IV P}}, \pppp{vrīḍita} \artha{sich schämen} 
\forms{
\CS vrīḍayati
}
\Zusatz{vrīḍā f.\ \artha{Scham}}

\medskip

\dhatu{śaṃs} \pres{śaṃsati; -te \gan{I P}}, \pppp{śasta/śaṃsita} \artha{hersagen, preisen} 
\forms{
\PF śaśaṃsa; śaśaṃse
\AO aśaṃsīt
\FT śaṃsiṣyati
\PS śasyate
\GDV śa(ṃ)sya, śastavya/śaṃsitavya, śaṃsanīya
\INFIN śaṃsitum
\ABS śastvā, \mbox{-śasya}
\CS śaṃsayati; -te
}
\Zusatz{pra-śaṃsā f.\ \artha{Lob}}

\dhatu{śak} \pres{śaknoti \gan{V P}}, \pppp{śak(i)ta} \artha{vermögen, können} 
\forms{
\IND śaknoti, śaknuvanti
\PF śaśāka, śekuḥ
\AO aśakat
\FT śakṣyati; -te
\PS śakyate
\GDV śakya
\DS śikṣati \DS \GDV śikṣaṇīya
}
\Zusatz{śakti f.\ \artha{Kraft}}\\
\vw{śikṣ}

\dhatu{śakalay} \pres{śakalayati} \gand{Denom.} \pppp{śakalita} \artha{zerstückeln}
\Zusatz{v.\ śakala m.\altern n.\ \artha{Splitter}}

\dhatu{śaṅk} \pres{śaṅkate; -ti \gan{I Ā}}, \pppp{śaṅkita} \artha{zweifeln, vermuten} 
\forms{
\PF śaśaṅke
\INJ mā śaṅkīḥ
\PS śaṅkyate
\GDV \mbox{-śaṅkya}, śaṅkitavya, śaṅkanīya
\INFIN  śaṅkitum
\ABS \mbox{-śaṅkya}
\CS śaṅkayati
}
\Zusatz{śaṅkā f.\ \artha{Zweifel, Sorge, Vermutung}}

\dhatu{śat} \pres{śātayati; -te}, \pppp{śātita} \artha{vernichten}

\dhatu{śap} \pres{śapati; -te \gan{I PĀ}}, \pppp{śap(i)ta} \artha{fluchen; schwören} 
\forms{
\PF śaśāpa, śepuḥ; śepe
\FT śapiṣyate
\PS śapyate
\INFIN śap(i)tum
\ABS śap(i)tvā, \mbox{-śapya}
\CS śāpayati
}
\Zusatz{śāpa m.\ \artha{Fluch}}

\dhatu{śabday/śabdāy} \pres{śabdayati; śabdāyate \gan{X PĀ}}, \pppp{śabdita} \artha{rufen, tönen}

\forms{
\CS śabdāpayati; -te \artha{herbeirufen, nennen}
}
\Zusatz{śabda m.\ \artha{Laut}}

\dhatu{śam [1]} \pres{śāmyati; -te \gan{IV P}}, \pppp{śānta} \artha{ruhig werden} 
\forms{
\PF śaśāma, śemuḥ
\GDV śamanīya
\ABS śamitvā/śāntvā, \mbox{-śamya}
\CS ś\shortlonga{}mayati \CS \AO aśīśamat \CS \GDV śāmya, śamayitavya
}
\Zusatz{śānti f.\ \artha{Ruhe}}

\dhatu{śam [2]} \pres{(ni-)śāmayati \gan{X Ā}}, \pppp{(ni-)śānta} \artha{gewahr werden} 
\forms{
\PS \mbox{(ni-)ś\shortlonga{}myate}
\GDV \mbox{-śāmayitavya}
}

\dhatu{śaś} \pres{śaśati \gan{I P}}, \pppp{śaśita} \artha{springen} 
\Zusatz{śaśa m.\ \artha{Hase}; śaśin m.\ \artha{Mond}}

\dhatu{śas} \pres{śasati \gan{I P}}, \pppp{śasta} \artha{schneiden, schlachten} 
\forms{
\IMP aśasat
\PF śaśāsa
\ABS \mbox{-śasya}
}
\Zusatz{śastra n.\ \artha{Klinge}}

\dhatu{śā/śi} \pres{śiśāti; śiśīte | --- \gan{IV P | V PĀ}}, \pppp{śita/śāta} \artha{schärfen, wetzen} 
\forms{
}

\dhatu{śāt} \vw{śat}

\dhatu{śās} \pres{śāsti \gan{II P} | śāsati; -te}, \pppp{śās(i)ta/śiṣṭa} \artha{anweisen; unterweisen; züchtigen} 
\forms{
\IND śāsmi, śāsti, śāsti, śiṣvaḥ, śiṣmaḥ, śiṣṭha, śāsati | śāsati; -te
\OP śiṣyāt
\IPV śāsāni, śādhi, śāstu, śāsāva, śiṣṭam, śiṣṭām, śāsāma, śiṣṭa, śāsatu
\IMP aśāsam, aśāḥ/aśāt, aśāt, aśiṣva, aśiṣma, aśiṣṭa, aśāsuḥ
\PF śaśāsa, śaśāsuḥ
\AO aśiṣat
\FT śāsiṣyati
\PS śāsyate/śiṣyate
\GDV śāsya/śiṣya, śāsitavya, śāsanīya
\INFIN śāstum
\ABS śāsitvā, \mbox{-śāsya}
}
\Zusatz{śāstra n.\ \artha{Lehrbuch}}

\dhatu{śikṣ} \pres{śikṣate \gan{I Ā}}, \pppp{śikṣita} \artha{lernen, versuchen} 
\forms{
\FT śikṣiṣyate
\PS śikṣyate
\GDV śikṣaṇīya
\INFIN śikṣitum
\ABS śikṣitvā, \mbox{-śikṣya}
\CS śikṣayati
}
\Zusatz{śikṣā f.\ \artha{Unterweisung}}\\
\vw{śak}

\dhatu{śithilay} \pres{śithilayati} \gand{Denom.} \pppp{śithilita} \artha{lockern, lösen}
\Zusatz{v.\ śithila \artha{locker}}

\dhatu{śiṣ} \pres{śinaṣṭi \gan{VII P}}, \pppp{śiṣṭa} \artha{übrig lassen} 
\forms{
\IND śinaṣṭi, śiṃṣvaḥ, śiṃṣanti
\IPV śinaṣāṇi, śiṅki, śinaṣṭu
\PS śiṣyate
\GDV śeṣya
\ABS śiṣṭvā, \mbox{-śiṣya}
\CS śeṣayati  \CS \PS śeṣyate
}
\Zusatz{śeṣa m.\altern n.\ \artha{Rest}}

\dhatu{śī} \pres{śete \gan{II Ā} | śayate; -ti}, \pppp{śayita} \artha{liegen, (ein)schlafen} 
\forms{
\IND śaye, śeṣe, śete, śevahe, śayāthe, śayāte, śemahe, śedhve, śerate | śayate; -ti
\OP śayīta
\IPV śayai, śeṣva, śetām, śayāvahai, śayāthām, śayātām, śayāmahai, śedhvam, śeratām
\IMP aśayi, aśethāḥ, aśeta, aśevahi, aśayāthām, aśayātām, aśemahi, aśedhvam, aśerata
\PF śiśye, śiśyire
\AO aśayiṣṭa
\FT śayiṣyate/śeṣyate; -ti
\PFT śayitā
\GDV śayitavya, śayanīya
\INFIN śayitum
\ABS śayitvā, \mbox{-śayya}
\CS śāyayati; -te \CS \PPP śāyita \CS \AO aśīśayat
\DS śiśayiṣate
}
\Zusatz{śayana n.\ \artha{Bett}}

\dhatu{śī} \vw{śyā/śyai/śī}

\dhatu{śīl} \pres{śīlayati \gan{X P}}, \pppp{śīlita} \artha{üben, pflegen zu tun; bewohnen; besorgen} 
\forms{
\OP śīlayet
\GDV śīlayitavya, śīlanīya
\INFIN śīlayitum
\ABS śīlayitvā
}
\Zusatz{śīla n.\ \artha{(guter) Charakter}}

\dhatu{śuc} \pres{śocati; -te \gan{I P}}, \pppp{śucita} \artha{Schmerz empfinden; (be)trauern} 
\forms{
\PF śuśoca
\AO aśucat
\FT śociṣyati
\GDV śocya, śocitavya, śocanīya
\INFIN śocitum
\ABS śocitvā
\CS śocayati \CS \PS śocyate
}
\Zusatz{śoka m.\ \artha{Schmerz, Kummer}}

\dhatu{śudh} \pres{śudhyati; -te \gan{IV P}}, \pppp{śuddha} \artha{reinigen} 
\forms{
\GDV śodhya, śodhanīya
\CS śodhayati \CS \PS śodhyate \CS \GDV śodhayitavya
}
\Zusatz{śuddhi f.\ \artha{Reinigung, Läuterung}}

\dhatu{śubh} \pres{śobhate; -ti \gan{I ĀP} | śumbhati \gan{VI P}}, \pppp{śubhita} \artha{schmücken; erscheinen} 
\forms{
\PF śuśubhe; śuśobha
\FT śobhiṣyati
\PS \AO aśobhi
\CS śobhayati
\DS śuśobhiṣate
\IS śośubhyate
}
\Zusatz{śobhā f.\ \artha{Schönheit}}

\dhatu{śuṣ} \pres{śuṣyati; -te \gan{IV P}}, \pppp{śuṣka} \artha{eintrocknen}
\forms{
\GDV śoṣya, śoṣaṇīya
\ABS \mbox{-śuṣya}
\CS śoṣayati; -te \CS \PS śoṣyate
}
\Zusatz{śuṣka meist nur adj.\ \artha{trocken}}

\dhatu{śū} \vw{śvi}

\dhatu{śṛ/śṝ} \pres{śṛṇāti \gan{IX P}}, \pppp{śīrṇa} \artha{zerbrechen} 
\forms{
\PS śīryate
}

\dhatu{śc(y)ut} \pres{śc(y)otati \gan{I P}}, \pppp{śc(y)utita} \artha{träufeln} 
\forms{
}

\dhatu{śyā/śyai/śī} \pres{śyāyate \gan{I Ā}}, \pppp{śīta/śyāna/śīna} \artha{gefrieren, gerinnen lassen} 
\forms{
\GDV \mbox{-śyāya}
}

\dhatu{śram} \pres{śrāmyati \gan{IV P} | śramati; -te}, \pppp{śrānta} \artha{müde werden} 
\forms{
\PF śaśrāma, śaśramuḥ
\PS śramyate
\GDV śrāntavya, śramaṇīya
\INFIN śramitum
\ABS \mbox{-śramya}
\CS śr\shortlonga{}mayati
}
\Zusatz{śrama m.\ \artha{Anstrengung, Ermüdung}; vi-śrama m.\ \artha{Ruhe, Erholung}}

\dhatu{śrambh} \pres{-śrambhate \gan{I Ā}}, \pppp{-śrabdha} \artha{vertrauen} 
\forms{
\GDV \mbox{-śrambhaṇīya}
\INFIN \mbox{-śrambhitum}
\ABS \mbox{-śrabhya}
\CS \mbox{-śrambhayati} \CS \PPP \mbox{-śrambhita}
}
\Zusatz{vi-śrambha m.\ \artha{Vertrauen}}

\dhatu{śri} \pres{śrayati; -te \gan{I PĀ}}, \pppp{śrita} \artha{lehnen an; gelangen zu} 
\forms{
\PF śiśrāya; śiśriye
\AO aśiśriyat
\FT śrayiṣyati; -te
\PS śrīyate
\GDV \mbox{-śrayitavya}, śrayaṇīya
\INFIN śrayitum
\ABS śrayitvā, \mbox{-śritya}
}\Zusatz{ā-śritya \artha{mittels}; °ā-śrita \artha{in \dots\ lebend}}

\dhatu{śru} \pres{śṛṇoti; śṛṇute | --- \gan{I P}}, \pppp{śruta} \artha{hören}
\forms{
\IND śṛṇoti, śṛṇutaḥ, śṛṇvanti; śṛṇute
\PF śuśrāva, śuśrotha, śuśrāva, śuśruva, śuśruvathuḥ, śuśruvatuḥ, śuśruma, śuśruva, śuśruvuḥ; śuśruve
\AO aśrauṣīt
\FT śroṣyati; \mbox{-te}
\PFT śrotā
\PS śrūyate \PS \AO aśrāvi
\GDV śravya/śrāvya, śrotavya, śravaṇīya
\INFIN śrotum
\ABS śrutvā, \mbox{-śrutya}
\CS śrāvayati; -te \CS \GDV śrāvya, śrāvayitavya, śrāvaṇīya
\DS śuśrūṣate; \mbox{-ti} \DS \GDV śuśrūṣya, śuśrūṣitavya
}
\Zusatz{śruti f.\ \artha{orthodoxe Überlieferung, der Veda; Ohr, Hören}; śravaṇa n.\ \artha{Hören}; śrotṛ m.\ \artha{Hörer}}

\dhatu{ślath} \pres{ślathati; -te \gan{I PĀ}}, \pppp{ślathita} \artha{lösen, locker werden}

\forms{
\PS ślathyate
\CS ślathayati; slathāyate
}

\dhatu{ślāgh} \pres{ślāghate \gan{I Ā}}, \pppp{ślāghita} \artha{prahlen, preisen, schmeicheln} 
\forms{
\PF śaślāghe, śaślaghire
\PS ślāghyate
\GDV ślāghya, ślāghanīya
\CS ślāghayati
}
\Zusatz{ślāghā f.\ \artha{Prahlerei}}

\dhatu{śliṣ} \pres{śliṣyati \gan{IV P}}, \pppp{śliṣṭa} \artha{sich anhängen, umarmen} 
\forms{
\PF śiśleṣa
\AO aśliṣat
\PS śliṣyate
\INFIN śleṣṭum
\ABS śliṣṭvā, \mbox{-śliṣya}
\CS śleṣayati; -te 
}

\dhatu{śvas} \pres{śvasiti \gan{II P} | śvasati; -te}, \pppp{śvas(i)ta} \artha{atmen} 
\forms{
\PF śaśvāsa
\INJ mā śvasīḥ
\FT śvasiṣyati
\GDV \mbox{-śvāsya}, \mbox{-śvasitavya}, \mbox{-śvasanīya}
\INFIN śvasitum
\ABS \mbox{-śvasya}
\CS śvāsayati \CS \GDV \mbox{-śvāsanīya}
}
\Zusatz{ā-śvāsa m.\ \artha{Trost, Erholung}}

\dhatu{śvā} \vw{śvi}

\dhatu{śvi} \pres{śvayati \gan{I P}}, \pppp{śūna} \artha{anschwellen} 
\forms{
\PS śūyate
}

\medskip

\dhatu{ṣṭhiv} \pres{ṣṭhīvati \gan{I P}}, \pppp{ṣṭhyūta/ṣṭhīvita} \artha{ausspeien} 
\forms{
\PF tiṣṭheva
\ABS \mbox{-ṣṭhīvya}
}

\medskip

\dhatu{sajjay} \pres{sajjayati} \gand{Denom.} \pppp{sajjita} \artha{besehnen; bereit machen}
\Zusatz{v.\ sajja \artha{bereit}}

\dhatu{sañj} \pres{sajati \gan{I P}}, \pppp{sakta} \artha{haften} 
\forms{
\PF sasañja
\AO asāṅkṣīt
\PS sajyate
\GDV sajya, \mbox{-sa(ṅ)ktavya}, \mbox{-sañjanīya}
\INFIN saktum
\ABS \mbox{-sajya}
\CS sañjayati/sajjayati; -te
}
\Zusatz{saṅga m.\ \artha{Umgang, Verkehr; Hang, Neigung}}

\dhatu{sad} \pres{sīdati; -te \gan{I | VI P}}, \pppp{sanna} \artha{sitzen; zusammensinken} 
\forms{
\PF sasāda, seditha/sasattha, seduḥ
\AO asadat
\FT sīdiṣyati
\PS sadyate
\GDV \mbox{-sādya}, \mbox{-sadanīya}
\INFIN sattum/sīditum
\ABS \mbox{-sadya}
\CS sādayati; -te 
}

\dhatu{sah} \pres{sahate; -ti \gan{I Ā}}, \pppp{soḍha} \artha{ertragen, bewältigen, vermögen} 
\forms{
\FT sahiṣyate/sakṣyati; sahiṣyati
\PFT soḍhā
\PS sahyate
\GDV sahya, soḍhavya/sahitavya, sahanīya
\INFIN soḍhum/sahitum
\ABS \mbox{-sahya}
\CS sāhayati 
}

\dhatu{sā/si} \pres{-syati; -te | --- \gan{V | IX PĀ}}, \pppp{-sita} \artha{binden}
\forms{
\PF \mbox{-sasau}
\PS \mbox{-sīyate}
\GDV \mbox{-seya}, \mbox{-setavya}
\INFIN \mbox{-situm}
\ABS \mbox{-sāya}
\CS \mbox{-sāyayati}
}

\dhatu{sādh} \pres{sādhayati; -te | --- \gan{V P}}, \pppp{sādhita} \artha{in Ordnung bringen; ausführen; erlangen}
\forms{
\PS sādhyate
\GDV sādhya, sādhayitavya, sādhanīya
\INFIN sādhitum
\DS siṣādhayiṣati/sisādhayiṣati
}
\Zusatz{sādhana n.\ \artha{rel.\ Praxis; Mittel, Beweis (wörtl.: zum Ziele führend)}}

\dhatu{sāntv} \pres{sāntvayati; -te \gan{X PĀ}}, \pppp{sāntvita} \artha{besänftigen, trösten} 
\forms{
\PPF sāntvayām āsa/cakāra
}

\dhatu{si} \vw{sā/si}

\dhatu{sic} \pres{siñcati; -te \gan{VI PĀ}}, \pppp{sikta} \artha{ausgießen, besprengen} 
\forms{
\PF siṣeca; siṣice
\AO asicat; -ta
\FT sekṣyati; -te
\PS sicyate
\GDV secya, sektavya, secanīya
\INFIN sektum
\ABS siktvā/siñcitvā, \mbox{-sicya}
\CS secayati; -te
}
\Zusatz{abhi-ṣeka m.\ \artha{Königsweihe}}

\dhatu{sidh [1]} \pres{sedhati; -te \gan{I P}}, \pppp{siddha} \artha{vertreiben} 
\forms{
\PF siṣedha; siṣidhe
\AO asedhīt
\FT sedhiṣyati
\PS sidhyate \PS \AO asedhi
\PPA
\PPFA
\GDV
\INFIN seddhum
\ABS \mbox{-ṣidhya}
\CS sedhayati
\DS
\IS
}
\Zusatz{prati-ṣedha m.\ \artha{Verbot}}

\dhatu{sidh [2]} \pres{sidhyati; -te \gan{IV P}}, \pppp{siddha} \artha{zum Ziele gelangen} 
\forms{
\FT setsyati; -te
}
\Zusatz{siddhi f.\ \artha{Gelingen, Vollkommenheit; mystische Kraft}}

\dhatu{siv/syū} \pres{sīvyati \gan{IV P}}, \pppp{syūta} \artha{nähen} 
\forms{
\GDV sīvya
\CS sīvayati
}

\dhatu{su} \pres{sunoti; sunute \gan{V PĀ}}, \pppp{suta} \artha{(aus)pressen} 
\forms{
\PF suṣāva; suṣuve
\FT soṣyati
\PS sūyate
\GDV s\shortlonga{}vya
\ABS \mbox{-sutya}
\CS sāvayati
}
\Zusatz{savana n.\ \artha{Somafest; Opfer}}

\dhatu{sukh} \pres{sukhayati \gan{X PĀ}}, \pppp{sukhita} \artha{erfreuen} 
\Zusatz{sukha \artha{bequem}, n.\ \artha{Behagen, Glück}}

\dhatu{sū} \pres{sūte; sauti \gan{II Ā} | sūyate; -ti \gan{IV Ā} | savati}, \pppp{s\shortlongu{}ta} \artha{zeugen, gebären} 
\forms{
\IND sūte, suvate | sūyate; -ti | savati
\OP suvīta
\IPV suvai, sūṣva, sūtām
\IMP asūta
\PF suṣuve; suṣāva
\FT saviṣyati/soṣyati; -te
\PS sūyate
\ABS sūtvā, \mbox{-sūya}
}
\Zusatz{\mbox{(pra-)}sūti f., pra-sava m.\ \artha{Geburt}}

\dhatu{sūc} \pres{sūcayati \gan{X PĀ}}, \pppp{sūcita} \artha{andeuten, verkünden} 

\dhatu{sūtr} \pres{sūtrayati \gan{X PĀ}}, \pppp{sūtrita} \artha{aneinander reihen, hervorbringen, als Sūtra darstellen}


\dhatu{sūd} \pres{sūdayati; -te \gan{X PĀ}}, \pppp{sūdita} \artha{töten} 
\forms{}

\dhatu{sṛ} \pres{sarati; -te \gan{I P}}, \pppp{sṛta} \artha{rennen, fließen} 
\forms{
\PF sasāra, sasartha, sasāra, sasruḥ; sasre
\FT sariṣyati
\GDV \mbox{-sārya}, \mbox{-sartavya}, \mbox{-saraṇīya}
\INFIN sartum
\ABS sṛtvā, \mbox{-sṛtya}
\CS sārayati; -te \CS \PS sāryate \CS \GDV \mbox{-sāraṇīya}
}
\Zusatz{ava-sara m.\ \artha{Gelegenheit, rechte Zeit}}

\dhatu{sṛj} \pres{sṛjati; -te \gan{VI P}}, \pppp{sṛṣṭa} \artha{erschaffen; werfen; ausgiessen} 
\forms{
\PF sasarja; sasṛje
\AO asrākṣīt
\FT srakṣyati; -te
\PS sṛjyate
\GDV sṛjya, sraṣṭavya
\INFIN  sraṣṭum
\ABS sṛṣṭvā, \mbox{-sṛjya}
\CS sarjayati \CS \GDV sarjya, sarjayitavya
\DS sisṛkṣati
}
\Zusatz{sarga m.\ \artha{Schöpfung, Geschöpf, Natur (Charakter)}}

\dhatu{sṛp} \pres{sarpati; -te \gan{I P}}, \pppp{sṛpta} \artha{schleichen, kriechen} 
\forms{
\PF sasarpa, sasṛpiva
\FT srapsyati
\PS sṛpyate
\GDV \mbox{-sṛpya}
\INFIN sarpitum
\CS sarpayati
\DS sisṛpsati
\IS sarīsṛpyate
}
\Zusatz{sarpa m.\ \artha{Schlange}}

\dhatu{sev} \pres{sevate; -ti \gan{I Ā}}, \pppp{sevita} \artha{besuchen, sich begeben zu, (be)dienen; genießen} 
\forms{
\PF siṣeve; siṣeva
\FT seviṣyate; -ti
\PS sevyate
\GDV sevya, sevitavya, sevanīya
\INFIN  sevitum
\ABS sevitvā, \mbox{-sevya}
\CS sevayati
}
\Zusatz{sevā f.\ \artha{Dienst, Verehrung}}

\dhatu{skand} \pres{skandati; -te \gan{I P}}, \pppp{skanna} \artha{springen} 
\forms{
\PF caskanda; caskande
\ABS \mbox{-skandya}
\CS skandayati \CS \PPP skandita
}

\dhatu{skhal} \pres{skhalati; -te \gan{I P}}, \pppp{skhalita} \artha{stolpern; irren} 
\forms{
\PF caskhāla, caskhaluḥ
\CS skhalayati
}

\dhatu{stan} \pres{stanayati \gan{X PĀ}}, \pppp{stanita} \artha{donnern} 
\forms{
\ABS stanitvā
}

\dhatu{stabh/stambh} \pres{stabhnāti \gan{IX P} | stambhate \gan{I Ā}}, \pppp{stabdha} \artha{stützen; erstarren} 
\forms{
\IPV stabhnāni, stabhāna, stabhnātu
\PF tastambha
\PS stabhyate \PS \AO astambhi
\GDV stambhanīya
\INFIN stabdhum
\ABS stabdhvā/stambhitvā, \mbox{-stabhya}
\CS stambhayati; -te
}
\Zusatz{stambha m.\ \artha{Säule}, ava-ṣṭambha m.\ \artha{Zuversicht, Mut}}

\dhatu{stu} \pres{stauti/stavīti; stute \gan{II PĀ}}, \pppp{stuta} \artha{preisen} 
\forms{
\IND staumi, stauṣi, stauti/stavīti, stuvanti; stute, stuvate
\OP stuyāt; stavīta
\IPV stavāni, stuhi, stautu/stavītu
\IMP astavam, astauḥ, astaut/astavīt, astuvan
\PF tuṣṭāva, tuṣṭuvuḥ; tuṣṭuve
\AO astāvīt/astauṣīt; austoṣṭa
\FT stoṣyati
\PS stūyate
\GDV stavya/stutya, stotavya, stavanīya
\INFIN stotum
\ABS stutvā, \mbox{-stutya}/\mbox{-stūya}
\CS stāvayati; -te
\DS tuṣṭūṣati
}
\Zusatz{\mbox{(pra-)}stuti f., stotra n.\ \artha{Lobeshymne}}

\dhatu{stṛ/stṝ} \pres{stṛṇoti; stṛṇute \gan{V PĀ} | stṛṇāti; stṛṇīte \gan{IX PĀ} | starati}, \pppp{stīrṇa/stṛta} \artha{(be)streuen} 
\forms{
\PF tastāra, tastaruḥ; tastare
\FT stariṣyati; \mbox{-te}
\PS stīryate
\ABS stṛtvā, \mbox{-stīrya}/\mbox{-stṛtya}
\CS stārayati
}
\Zusatz{vi-stāra m.\ \artha{Ausdehung, Weitläufigkeit}}

\dhatu{styā} \pres{styāyate \gan{IV Ā}}, \pppp{styāna} \artha{gerinnen, sich verhärten}
\forms{
}

\dhatu{sthā} \pres{tiṣṭhati; -te \gan{I P}}, \pppp{sthita} \artha{stehen, verweilen} 
\forms{
\PF tasthau; tasthe
\AO asthāt; 
\FT sthāsyati; te
\PFT sthātā
\PS sthīyate \PS \AO asthāyi
\GDV stheya, sthātavya
\INFIN sthātum/\mbox{-sthitum}
\ABS sthitvā, \mbox{-sthāya}
\CS sthāpayati; -te \CS \PPF sthāpayām āsa \CS \PS sthāpyate \CS \GDV sthāpya, sthāpayitavya, sthāpanīya
\DS tiṣṭhāsati
}\Zusatz{adhi-ṣṭhāya, ā-sthāya \artha{mittels}; sthiti f.\ \artha{Verweilen, Zustand, Erhaltung}}

\dhatu{snā} \pres{snāti \gan{II P} | snāyate}, \pppp{snāta} \artha{baden} 
\forms{
\PF sasnau, sasnuḥ
\FT snāsyati; -te
\PS \AO asnāyi
\GDV sneya, snātavya
\INFIN snātum
\ABS snātvā, \mbox{-snāya}
\CS sn\shortlonga{}payati; snāpayate 
}
\Zusatz{snāna n.\ \artha{Bad, Baden}}

\dhatu{snih} \pres{snihyati; -te \gan{IV P}}, \pppp{snigdha} \artha{klebrig, feucht werden; lieben} 
\forms{
\ABS snigdhvā, \mbox{-snehya}
\CS snehayati  \CS \GDV snehayitavya
}
\Zusatz{sneha m.\ \artha{Klebrigkeit; Liebe}}

\dhatu{spand} \pres{spandate; -ti \gan{I Ā}}, \pppp{spandita} \artha{zucken} 
\forms{
\INFIN spanditum
\CS spandayati; -te
}
\Zusatz{spanda m.\ \artha{Zucken}}

\dhatu{spardh} \pres{spardhate; -ti \gan{I Ā}}, \pppp{spardhita} \artha{wetteifern} 
\forms{
\PF paspardha
\GDV spardhya, spardhanīya
}

\dhatu{spṛś} \pres{spṛśati; -te \gan{VI P}}, \pppp{spṛṣṭa} \artha{berühren} 
\forms{
\PF pasparśa, paspṛśuḥ; paspṛśe
\AO asprākṣīt
\FT sprakṣyati
\PS spṛśyate
\GDV spṛśya, spraṣṭavya, sparśanīya
\INFIN  spraṣṭum
\ABS spṛṣṭvā, \mbox{-spṛśya}
\CS sparśayati; -te \CS \GDV sparśayitavya
\DS pispṛkṣati
}
\Zusatz{sparśa m.\ \artha{Berührung, taktiles Gefühl}}

\dhatu{spṛh} \pres{spṛhayati; -te \gan{X PĀ}}, \pppp{spṛhita} \artha{eifern um} 
\forms{
\GDV spṛhaṇīya
}
\Zusatz{spṛhā f.\ \artha{Begehren}}

\dhatu{sphar} \pres{spharati \gan{VI P}}, \pppp{---} \artha{weit öffnen, aufreissen} 
\forms{
\ABS spharitvā
\CS sphārayati \CS \PPP sphārita
}

\dhatu{sphuṭ} \pres{sphuṭati \gan{VI P}}, \pppp{sphuṭita} \artha{bersten} 
\forms{
\PF pusphoṭa
\FT sphuṭiṣyati
\ABS sphuṭitvā
\CS sphoṭayati 
}
\Zusatz{sphoṭa m.\ \artha{Platzen; unteilbarer Bedeutungsträger in der einheimischen Semasiologie}}

\dhatu{sphuṭay} \pres{sphuṭayati} \gand{Denom.} \pppp{sphuṭita} \artha{platzen, deutlich werden}
\Zusatz{v.\ sphuṭa \artha{offen, deutlich}}

\dhatu{sphur} \pres{sphurati; -te \gan{VI P}}, \pppp{sphurita} \artha{schnellen, zittern, funkeln, hervorbrechen}

\forms{
\FT sphuriṣyati
\INFIN sphuritum
\ABS sphuritvā/spharitvā
\CS sphurayati
}

\dhatu{sphūrj} \pres{sphūrjati \gan{I P}}, \pppp{sphūrjita} \artha{dröhnen} 
\forms{
\ABS \mbox{-sphūrjya}
\CS sphūrjayati
}

\dhatu{smi} \pres{smayate; -ti \gan{I Ā}}, \pppp{smita} \artha{lächeln; erröten} 
\forms{
\PF sismiye, sismāya
\PPF (vi-)smayām āsa
\AO asmayiṣta
\GDV \mbox{-smayanīya}
\ABS sm(ay)itvā, \mbox{-smitya}
\CS smāpayati/smāyayati \CS \GDV \mbox{-smāpanīya}/\mbox{-smāpayanīya}
}
\Zusatz{vi-smaya m.\ \artha{Erstaunen}}

\dhatu{smṛ} \pres{smarati; -te \gan{I P}}, \pppp{smṛta} \artha{sich erinnern, vergegenwärtigen; überliefern} 
\forms{
\PF sasmāra, sasmaruḥ
\FT smariṣyati
\PS smaryate
\GDV smarya, smartavya, smaraṇīya
\INFIN  smartum
\ABS smṛtvā/smaritvā, \mbox{-smṛtya}
\CS sm\shortlonga{}rayati; smārayate \CS \PS smāryate
}
\Zusatz{smṛti f.\ \artha{Erinnerung, Tradition}}

\dhatu{sya} \vw{sā/si}

\dhatu{syand} \pres{syandate; -ti \gan{I Ā}}, \pppp{syanna} \artha{fließen, strömen} 
\forms{
\PF sasyande
\PS syandyate
\CS syandayati
}

\dhatu{sraṃs/sras} \pres{sraṃsate; -ti \gan{I Ā}}, \pppp{srasta} \artha{abfallen, erschlaffen} 
\forms{
\AO asrasat
\CS sraṃsayati \CS \PS sraṃsyate
}

\dhatu{srambh} \vw{śrambh}

\dhatu{sru} \pres{sravati; -te \gan{I P}}, \pppp{sruta} \artha{(aus)strömen} 
\forms{
\PF susrāva, susruvuḥ; susruve
\FT sraviṣyati
\CS sr\shortlonga{}vayati; srāvayate \CS \GDV srāvya
}

\dhatu{svaj/svañj} \pres{svajate; -ti \gan{I Ā}}, \pppp{svakta} \artha{umarmen} 
\forms{
\PF sasvaje; sasvajuḥ
\FT svajiṣyate
\INFIN svaktum
\ABS svajitvā, \mbox{-svajya}
}

\dhatu{svad} \pres{svadate; -ti \gan{I Ā}}, \pppp{svātta} \artha{schmackhaft machen; schmecken} 
\forms{
\PF sasvade
\GDV svādya, svadanīya
\CS svādayati; -te  \CS \PPP svādyate \CS \GDV svādanīya
}
\Zusatz{svādu \artha{süß}}

\dhatu{svan} \pres{svanati; -te \gan{I P}}, \pppp{svanita} \artha{schallen} 
\forms{
\PF sasvāna, sasvanuḥ
\CS svanayati
}

\dhatu{svap} \pres{svapiti \gan{II P} | svapati; -te}, \pppp{supta} \artha{(ein)schlafen} 
\forms{
\IND svapiti, svapanti | svapati; -te
\PF suṣvāpa, suṣupuḥ
\AO asvāpsīt
\FT svapsyati; svapsyate/svapiṣyate
\PS supyate \PS \AO asvāpi
\GDV svaptavya, svapanīya
\INFIN svaptum
\ABS suptvā
\CS svāpayati
\DS suṣupsati
}


\dhatu{svapnāy} \pres{svapnāyate} \gand{Denom.} \pppp{---} \artha{schläfrig sein, träumen}
\Zusatz{v.\ svapna m.\ \artha{Schlaf, Traum}}

\dhatu{svid} \pres{svidyati; -te \gan{IV P}}, \pppp{svinna} \artha{schwitzen} 
\forms{
\PF siṣvide
\GDV svedya
\CS svedayati 
}
\Zusatz{sveda m.\ \artha{Schweiß}}

\medskip

\dhatu{han} \pres{hanti; hate \gan{II P}}, \pppp{hata} \artha{(er)schlagen} 
\forms{
\IND hanmi, haṃsi, hanti, hanvaḥ, hathaḥ, hataḥ, hanmaḥ, hatha, ghnanti; hate, ghnate
\OP hanyāt
\IPV hanāni, jahi, hantu, hanāma, hata, ghnantu
\IMP ahanam, ahan, ahan, ahanva, ahatam, ahatām, ahanma, ahata, aghnan
\PF jaghāna, jaghnuḥ; jaghne
\FT haniṣyati; -te
\PFT hantā
\PS hanyate
\GDV ghātya, hantavya
\INFIN hantum
\ABS hatvā, \mbox{-hatya}/\mbox{-hanya}
\CS ghātayati; -te  \CS \AO ajīghanat \CS \GDV ghātya
\DS jighāṃsati; -te
} \Zusatz{vertritt präsentische Formen von \vw{vadh}}

\dhatu{has} \pres{hasati; -te \gan{I P}}, \pppp{hasita} \artha{(ver)lachen} 
\forms{
\PF jahāsa, jahasuḥ; jahase
\FT hasiṣyati
\PS hasyate \PS \AO ahāsi
\GDV hāsya, hasanīya
\INFIN hasitum
\ABS hasitvā, \mbox{-hasya}
\CS hāsayati
\IS jāhasyate
}
\Zusatz{(upa-)hāsa m.\ \artha{Gelächter}}

\dhatu{hā} \pres{jahāti \gan{III P}}, \pppp{hīna/hāta/jahita} \artha{verlassen} 
\forms{
\IND jahāmi, jahāsi, jahāti, jah\shortlongi{}maḥ, jah\shortlongi{}tha, jahati
\OP jahyāt
\IPV jahāni, jah\shortlongu{}hi/jahāhi, jahātu, jahāma, jahīta, jahatu
\IMP ajahām, ajahāḥ, ajahāt, ajahuḥ
\PF jahau, jahitha/jahātha, jahuḥ; jahe
\AO ahāsīt/ahāt
\FT hāsyati/jahiṣyati; hāsyate
\PS hīyate
\GDV heya, hātavya
\INFIN  hātum
\ABS hitvā, \mbox{-hāya}
\CS hāpayati; -te \CS \AO ajījahat \CS \GDV \mbox{-hāpanīya}
\DS jihāsati
}
\Zusatz{hāni f.\ \artha{Abnahme, Verlust}}

\dhatu{hi} \pres{hinoti \gan{V P} | hinvati}, \pppp{hita} \artha{antreiben; schicken} 
\forms{
\IND hinoti, hinvanti | hinvati
\PF jighāya, jighyuḥ
\FT heṣyati
\PS hīyate
\GDV hetavya
}

\dhatu{hiṃs} \pres{hinasti \gan{VII P} | hiṃsati; -te}, \pppp{hiṃsita} \artha{verletzen, töten} 
\forms{
\IND hinasmi, hinassi, hinasti, hiṃsmaḥ, hiṃstha, hiṃsanti | hiṃsati; -te
\OP hiṃsyāt
\IPV hinasāni, hindhi, hinastu
\IMP ahinasam, ahinaḥ/ahinat, ahinat, ahiṃsan
\PF jihiṃsa
\AO ahiṃsīt
\FT hiṃsiṣyati; -te
\PS hiṃsyate
\GDV hiṃsya, hiṃsitavya, hiṃsanīya
\INFIN hiṃsitum
\CS hiṃsayati 
}
\Zusatz{a-hiṃsā f.\ \artha{Gewaltlosigkeit}}

\dhatu{hikk} \pres{hikkati; -te \gan{I P}}, \pppp{hikkita} \artha{schlucksen} 

\dhatu{hu} \pres{juhoti; juhute \gan{III P}}, \pppp{huta} \artha{darbringen, opfern} 
\forms{
\IND juhoti, juhvati; juhute, juhvate
\PF juhāva, juhuvuḥ; juhuve
\PPF juhavāṃ cakāra
\AO ahauṣīt
\FT hoṣyati; -te
\PS hūyate
\PPA juhvant
\GDV havya, hotavya
\INFIN hotum
\ABS hutvā
\CS hāvayati
\DS juhūṣati
\IS johavīti
}
\Zusatz{homa m.\ \artha{Opferguss}}

\dhatu{hū} \vw{hvā/hve}

\dhatu{hṛ} \pres{harati; -te \gan{I PĀ}}, \pppp{hṛta} \artha{(weg)nehmen} 
\forms{
\PF jahāra, jahartha, jahruḥ; jahre
\AO ahārṣīt; ahṛta
\FT hariṣyati; -te
\PFT hartā
\PS hriyate \PS \AO ahāri
\GDV hārya, hartavya, \mbox{-haraṇīya}
\INFIN hartum
\ABS hṛtvā, \mbox{-hṛtya}
\CS hārayati \CS \PS hāryate \CS \GDV \mbox{-hārayitavya}
\DS jihīrṣati; jihīrṣyate
\IS jarīharti
}
\Zusatz{hara \artha{raubend}, m.\ \artha{Räuber}; hāra m.\ \artha{Raub}}

\dhatu{hṛṣ} \pres{hṛṣyati; -te \gan{IV P}}, \pppp{hṛṣṭa/hṛṣita} \artha{erregt werden} 
\forms{
\PF jaharṣa, jahṛṣuḥ
\AO ahṛṣat
\GDV harṣitavya, harṣaṇīya
\INFIN harṣitum
\ABS \mbox{-hṛṣya}
\CS harṣayati; -te \CS \PPP harṣita
}
\Zusatz{harṣa m.\ \artha{Freude, Erregung}}

\dhatu{heṣ/hreṣ} \pres{h(r)eṣate; -ti \gan{I Ā}}, \pppp{h(r)eṣita} \artha{wiehern} 
\forms{
\PF jiheṣire
}

\dhatu{hnu} \pres{hnute; hnauti \gan{II Ā}}, \pppp{hnuta} \artha{beseitigen, leugnen}

\forms{
\INFIN hnotum
\ABS \mbox{-hnutya}
}

\dhatu{hras} \pres{hrasati; -te \gan{I P}}, \pppp{hrasita} \artha{abnehmen} 
\forms{
\CS hrāsayati \CS \GDV hrāsanīya
}

\dhatu{hrād} \pres{hrādate \gan{I Ā}}, \pppp{hrādita} \artha{tönen} 
\forms{
}
\Zusatz{hrāda m.\ \artha{Geräusch}}

\dhatu{hrī} \pres{jihreti \gan{III P}}, \pppp{hrīṇa/hrīta} \artha{sich schämen} 
\forms{
\IND jihreti, jihrītaḥ, jihriyati
\OP jihrīyāt
\IPV jihretu
\IMP ajihret
\PF jihrāya, jihriyuḥ
\CS hrepayati
\IS jehrīyate
}

\dhatu{hreṣ} \vw{heṣ}

\dhatu{hlād} \pres{hlādate \gan{I Ā}}, \pppp{hlādita/\mbox{-hlanna}} \artha{erquicken} 
\forms{
\CS hlādayati; -te \CS \AO ahlādayiṣata
}
\Zusatz{(ā-)hlāda m.\ \artha{Erquickung}}

\dhatu{hval} \pres{hvalati; -te \gan{I P}}, \pppp{hvalita} \artha{fallen, verunglücken} 
\forms{
}

\dhatu{hvā/hve} \pres{hvayati; -te \gan{I PĀ}}, \pppp{hūta} \artha{(herbei)rufen} 
\forms{
\PF juhāva, juhuvuḥ
\PPF hvayām āsa/cakre
\FT hvāsyate
\PS hūyate
\GDV havya, hvātavya
\INFIN hvātum
\ABS hūtvā, \mbox{-hūya}
\CS hvāyayati \CS \GDV \mbox{-hvāyayitavya}
\DS juhūṣati
\IS johavīti
}
\Zusatz{ā-hvāna n.\ \artha{Anrufen; gerichtliche Vorladung}}
